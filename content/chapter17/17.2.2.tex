\setlength{\columnsep}{3pt}
\begin{flushleft}
	Synchronized method using object lock:
	\begin{itemize}
		\item \textbf{Step1: Choose an Object}
		\begin{itemize}
			\item Select the Java object of your choice.
			\item Every object in Java has a unique lock. 
		\end{itemize}
		\item \textbf{Step 2: Acquire the Lock}
		\begin{itemize}
			\item A thread first gets lock of that object and enters the critical section.
			\item Only one thread can hold the object lock at a time.
		\end{itemize}
		\item \textbf{Step 3: Execute Critical Section}
		\begin{itemize}
			\item The thread executes code inside synchronized block/method.
			\item This code needs to be protected from concurrent access.
			\item Other threads are not allowed to execute synchronized method on the given object. 
			\item Other threads can execute only non-synchronised methods on the same object.
		\end{itemize}
		\item \textbf{Step 4: Release the Lock}
		\begin{itemize}
			\item The thread releases the lock, once method completes execution.
		\end{itemize}
	\end{itemize}

	\newimage{0.5}{content/chapter17/images/six.png}
	\newpage
	Eg: Below code updates Customer object 2 times, notice the output without synchronised method set()
	\codeblockfull{Demo.java}{
		class Customer \{ \\
		\s String name; \\
		\s	Customer(String name) \{ \\
		\s \s		this.name = name; \\
		\s	\} \\
		\s	public void set(String name) \{ \\
		\s \s		this.name = name; \\
		\s \s		try \{ \\
		\s \s \s			Thread.sleep(2000); \\
		\s \s		\} \\
		\s \s		catch (InterruptedException e) \{\} \\
		\s \s		System.out.println(Thread.currentThread().getName() + " Updated the Customer"); \\
		\s	\} \\
		\s	public void get()\{ \\
		\s \s		System.out.println("Customer name: " + this.name); \\
		\s	\} \\
		\} \\
		\\
		class Demo extends Thread \{ \\
		\s	Customer c; \\
		\s	String name; \\
		\s	Demo(Customer c, String name) \{ \\
		\s \s		this.c = c; \\
		\s \s		this.name = name; \\
		\s	\} 
	}
	\bigskip
	\codecontinue{
		\s	public void run() \{ \\
		\s \s		c.set(name); \\
		\s \s		c.get(); \\
		\s	\} \\
		\s	public static void main(String[] args) throws InterruptedException\{ \\
		\s \s		Customer c1 = new Customer("Ramesh"); \\
		\s \s		Demo t1 = new Demo(c1,"Ram"); \\
		\s \s		Demo t2 = new Demo(c1,"Sham"); \\
		\s \s		t1.start(); \\
		\s \s		t2.start(); \\
		\s \s		t1.join(); \\
		\s \s		t2.join(); \\
		\s	\} \\
		\}	 
	}
	\outputblock{
		Thread-1 Updated the Customer \\
		Thread-0 Updated the Customer \\
		Customer name: Sham \\
		Customer name: Sham
	}
	
	\bigskip
	
	Eg: Below code updates Customer object 2 times, notice the output with synchronised method set().
	
	\codeblockfull{Demo.java}{
		class Customer \{ \\
		\s String name; \\
		\s Customer(String name) \{ \\
		\s \s this.name = name; \\
		\s	\} 
	}

	\codecontinue{
		\s	synchronized public void set(String name) \{ \\
		\s \s	this.name = name; \\
		\s \s		try \{ \\
		\s \s \s			Thread.sleep(2000); \\
		\s \s		\}  \\
		\s \s		catch (InterruptedException e) \{\} \\
		\s \s		System.out.println(Thread.currentThread().getName() + " Updated the Customer"); \\
		\s	\} \\
		\s	public void get()\{ \\
		\s	\s	System.out.println("Customer name: " + this.name); \\
		\s	\} \\
		\} \\
		\\
		class Demo extends Thread \{ \\
		\s	Customer c; \\
		\s	String name; \\
		\s	Demo(Customer c, String name) \{ \\
		\s \s		this.c = c; \\
		\s \s		this.name = name; \\
		\s	\} \\
		\s	public void run() \{ \\
		\s \s		c.set(name); \\
		\s \s		c.get(); \\
		\s	\} \\
		\s	public static void main(String[] args) throws InterruptedException\{ \\
		\s \s		Customer c1 = new Customer("Ramesh"); \\
		\s \s		Demo t1 = new Demo(c1,"Ram"); \\
		\s \s		Demo t2 = new Demo(c1,"Sham"); 
	}
	\bigskip
	\codecontinue{
		\s \s		t1.start(); \\
		\s \s		t2.start(); \\
		\s \s		t1.join(); \\
		\s \s		t2.join(); \\
		\s	\} \\
		\}		
	}
	\bigskip
	\outputblock{
		Thread-0 Updated the Customer \\
		Customer name: Ram \\
		Thread-1 Updated the Customer \\
		Customer name: Sham
	}

	\textbf{Notice}
	\begin{itemize}
		\item If set() is not declared as synchnosied, then both the threads will be executed simultaneously and hence, we will get irregular output.	
		\item If set() is synchronised, then at a time, only one thread is allowed to execute set()
	\end{itemize}
	
\end{flushleft}
