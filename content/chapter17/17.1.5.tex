\setlength{\columnsep}{3pt}
\begin{flushleft}

		\textbf{start()}:
		\begin{itemize}
			\item start() comes from Thread class \& is heart of multi-threading.
			\item \textbf{start() creates new thread} and \textbf{executes run()}.
			\item start() can be overridden, however \textbf{it is not recommended}.
			\item Role of start():
			\begin{itemize}
				\item \textbf{Register the thread} with thread schedular
				\item Perform related mandatory activities
				\item \textbf{Invoke run()}
			\end{itemize}
		\end{itemize}
	
		\textbf{run()}:
		\begin{itemize}
			\item run() contains the code executed by start().
			\item By default, Thread class's run() is executed i.e empty.
			\item \textbf{Override this} method when you extend the Thread class or implement the \textbf{Runnable} interface.
			\item Overloaded run() will need to be call explicitly.
			\item start() always invoke no-argument run().
		
			\bigskip
			\noteblock{
				\textbf{Difference between start() and run()}:
				\newline
				\textbf{start() creates new thread \& calls run()}. \newline
				\textbf{run() will not create} \textbf{new thread}. Calling run() explicitly will execute like a normal method call.
			}

		\end{itemize}	


	\newpage	
		Eg: Below code shows overridden start() not executing run() explicitly.
		\bigskip
		\codeblockfull{demo.java}{
			class DemoThread extends Thread \{ \\
			\s	public void start() \{ \\
			\s \s	System.out.println("Overridden Start"); \\
			\s	\} \\
			\s	public void run() \{ \\
			\s \s		System.out.println("This wont execute"); \\
			\s	\} \\
			\} \\
			public class demo \{ \\
			\s	public static void main(String[] args) \{ \\
			\s \s		DemoThread demoThread = new DemoThread(); \\
			\s \s		demoThread.start(); \\
			\s	\} \\
			\}
		}
		\bigskip
		\outputblock{
			Overridden Start
		}
		
		\bigskip
		
		Eg: Below code shows overridden run() and no-arg run().
		\bigskip
		\codeblockfull{demo.java}{
			class DemoThread extends Thread \{ \\
			\s public void run() \{ \\
			\s System.out.println("No-argument run executed"); \\
			\s	\}   \\
			\s	public void run(int i) \{ \\
			\s \s		System.out.println("run with argument executed"); 
		}
		\newpage
		\codecontinue{
			\s	\} \\
			\} \\
			public class demo \{ \\
			\s	public static void main(String[] args) \{ \\
			\s \s		DemoThread demoThread = new  DemoThread(); \\
			\s \s		demoThread.start(); \\
			\s	\} \\
			\} 
		}
		\bigskip
		\outputblock{
			No-argument run executed
		}
		
		You can call super.start() in overridden start() to create new thread as usual:
		\newline
		Eg:
		\codeblockfull{demo.java}{
			class DemoThread extends Thread \{ \\
			\s public void start() \{ \\
			\s \s	super.start(); \\
			\s \s	System.out.println("Overridden Start"); \\
			\s \} \\
			\s 	public void run() \{ \\
			\s \s	System.out.println("This will execute"); \\
			\s	\} 
			\} \\
			public class demo \{ \\
			\s	public static void main(String[] args) \{ \\
			\s \s	DemoThread demoThread = new DemoThread(); \\
			\s \s	demoThread.start(); \\
			\s	\} 
			\}
		}
		\outputblock{
			Overridden Start \\
			This will execute
		}
			
		After starting a thread, if you are trying to restart the same thread, then it will result in runtime exception saying: \textbf{IllegalThreadState} exception.
		\newline
		Eg:
		\codeblockfull{demo.java}{
			class DemoThread extends Thread \{\} \\
			public class demo \{ \\
			\s	public static void main(String[] args) \{ \\
			\s	\s	DemoThread demoThread = new DemoThread(); \\
			\s \s		demoThread.start(); \\
			\s \s		demoThread.start(); \\
			\s	\} \\
			\}
		}
		\outputblock{
			Exception in thread "main" \\ java.lang.IllegalThreadStateException
			at \\ java.base/java.lang.Thread.start(Thread.java:794) \\
			at demo.main(demo.java:7)
		}
		
\end{flushleft}
\newpage