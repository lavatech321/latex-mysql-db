\setlength{\columnsep}{3pt}
\begin{flushleft}

	\begin{itemize}
		\item Used if number of lines of the code to be synchronised is less.
		\item Pros of synchronised block over synchronised method:
		\begin{itemize}
			\item Reduces thread waiting time.
			\item Improves application performance.
		\end{itemize}
		\item Ways of creating synchronised blocks:
		\begin{itemize}
			\item Get lock of particular object:
			\bigskip
			\syntaxblock{
				synchronised(object) {}
			}
			\item Get lock of class-level object:
			\bigskip
			\syntaxblock{
					synchronised(Class) {}
			}
		\end{itemize}
		\item Eg: Consider below code that uses synchronized block to obtain object lock and update custom name using multiple threads.
		\bigskip
		\codeblockfull{demo.java}{
			class Customer \{ \\
			\s	String name; \\
			\s	Customer(String name) \{ \\
			\s \s		this.name = name; \\
			\s	\} \\
			\s	public void set(String name) \{ \\
			\s \s		this.name = name; \\
			\s	\} \\
			\s	public void get()\{ \\
			\s		System.out.println("Customer name: " + this.name); \\
			\s	\} \\
			\} 
		}
		\newpage
		\codecontinue{
			class Demo extends Thread \{ \\
			\s	Customer c; \\
			\s	String name; \\
			\s	Demo(Customer c, String name) \{ \\
			\s \s		this.c = c; \\
			\s \s		this.name = name; \\
			\s	\} \\
			\s	public void run() \{ \\
			\s \s		synchronized (c) \{ \\
			\s \s \s			c.set(name); \\
			\s \s \s			try \{ \\
			\s \s \s				Thread.sleep(2000); \\
			\s \s \s			\} \\
			\s \s \s			catch (InterruptedException e) \{\} \\
			\s			System.out.println(Thread.currentThread().getName() + " Updated the Customer"); \\
			\s \s 		\} \\
			\s	\s	c.get(); \\
			\s	\} \\
			\s	public static void main(String[] args)  throws InterruptedException\{  \\
			\s \s		Customer c1 = new Customer("Ramesh"); \\
			\s \s		Demo t1 = new Demo(c1,"Ram"); \\
			\s \s		Demo t2 = new Demo(c1,"Sham"); \\
			\s \s		t1.start(); \\
			\s \s		t2.start(); \\
			\s \s		t1.join(); \\
			\s \s		t2.join(); \\
			\s	\} \\
			\}
		}
		\bigskip
		\outputblock{
			Thread-0 Updated the Customer \\
			Customer name: Ram \\
			Thread-1 Updated the Customer \\
			Customer name: Sham
		}
	
	
 		\item Eg: Consider below code that uses synchronized block to obtain class lock and update a variable using multiple threads.
		\bigskip	
		\codeblockfull{demo.java}{
			class MyClass \{ \\
			\s private static int count = 0; \\
			\s \s	public static void incrementCount() \{ \\
			\s \s \s		synchronized (MyClass.class) \{ \\
			\s \s \s \s			count++; \\
			\s \s \s		\} \\
			\s \s	\}  \\
			\s	public static int getCount() \{ \\
			\s \s		synchronized (MyClass.class) \{  \\
			\s \s \s			return count; \\
			\s \s		\} \\
			\s	\} \\
			\} \\
			class ClassLevel implements Runnable \{ \\
			\s public void run() \{ \\
			\s \s		for (int i = 0; i < 5; i++) \{ \\
			\s \s \s			MyClass.incrementCount(); \\
			\s 	System.out.println(Thread.currentThread().getName() + ": Count is " + MyClass.getCount()); 
		} 
		\newpage
		
		\codecontinue{
			\s \s \s			try \{  \\
			\s \s \s \s				Thread.sleep(100); \\
			\s \s \s			\} \\
			\s \s \s			catch (InterruptedException e) \{  \\
			\s \s \s \s				e.printStackTrace(); \\
			\s \s \s			\} \\
			\s \s		\} \} \} \\ 
			class Main \{ \\
			\s	public static void main(String[] args)  throws InterruptedException \{ \\
			\s \s		ClassLevel c1 = new ClassLevel(); \\
			\s \s		Thread thread1 = new Thread(c1); \\
			\s \s		Thread thread2 = new Thread(c1); \\
			\s \s		thread1.start(); \\
			\s \s		thread2.start(); \\
			\s \s		thread1.join(); \\
			\s \s		thread2.join(); \\
			\s		System.out.println("Final Count: " + MyClass.getCount()); \\
			\s 	\} \\
			\}	
		}
		\bigskip
		\outputblock{
			Thread-0: Count is 2 \\
			Thread-1: Count is 3 \\
			Thread-0: Count is 4 \\
			.. \\
			Thread-1: Count is 8 \\
			Thread-1: Count is 10 \\
			Thread-0: Count is 10 \\
			Final Count: 10 
		}
	\end{itemize}	
	
	
\end{flushleft}

\newpage