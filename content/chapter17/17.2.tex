\setlength{\columnsep}{3pt}
\begin{flushleft}

	Prevent thread execution using below methods:
	\begin{itemize}
		\item yield()
		\item join()
		\item sleep()
		\item interrupt()
		\item stop() (deprecated)
		\item suspend() \& resume() (deprecated)
	\end{itemize}	

	\textbf{yield()}:
		\begin{itemize}
			\item yield() \textbf{pauses current thread} \& \textbf{runs same priority waiting threads}.
			\item Low priority waiting threads causes same thread to run.
			\bigskip
			\syntaxblock{
				public static native void yield()
			}
			\item Here, native means method is in non-java language.
			\item yield() \& thread life cycle - 
			\newimage{0.37}{content/chapter17/images/three.png}
			\item Eg: In below code, if you uncomment "Line 1", then both threads (main \& DemoThread) will be executed. As "Line 1" is commented, it first executes main thread and then DemoThread.
			\bigskip
			\codeblockfull{demo.java}{
			   class DemoThread extends Thread \{ \\
			   \s	public void run() \{ \\
			    \s \s	for (int i = 0; i <5; i++) \{ \\
				\s \s		System.out.println("Child Thread"); \\
				\s \s		Thread.yield();   // Line 1 \\
				\s	\} \\
			   	\s 	\} \\
				\}  \\
				public class demo \{ \\
				\s	public static void main(String[] args) \{ \\
				\s \s		DemoThread t1 = new DemoThread(); \\
				\s \s		t1.start(); \\
				\s \s		for(int i = 0; i < 5; i++) \{ \\
				\s \s			System.out.println("Main Thread"); \\
				\s \s		\} \\
				\s	\} \\
				\}
			}
			\bigskip
			\outputblock{
				Main Thread \\
				..\\
				Child Thread \\
				..
			}
			\noteblock{
			Some platforms wont provide proper support for yield().
			}
		\end{itemize}
	
		\newpage
	\textbf{sleep()}:
		\begin{itemize}
			\item \textbf{Pauses a thread for given time}.
			\bigskip
			\syntaxblock{
				public static native void sleep(long ms) throws InterruptedException \\
				\\
				public static native void sleep(long millisec, int nanosec) throws InterruptedException
			}
			\item sleep() \& thread life-cycle:
			\newimage{0.5}{content/chapter17/images/four.png}
			\newpage
			\item Eg: Below code causes thread to sleep for 2000 seconds:
			\bigskip
			\codeblockfull{demo.java}{
				public class demo \{ \\
				\s public static void main(String[] args) throws InterruptedException\{ \\
				\s \s for(int i=0; i<5; i++) \{ \\
				\s \s System.out.println("Main-Thread"); \\
				\s \s \s			Thread.sleep(2000); \\
				\s \s		\} \\
				\s	\} \\
				\}	
			}
			\bigskip
			\outputblock{
			Main-Thread \\
			.. 
			}
		\end{itemize}
	
		\bigskip
		
	\textbf{join()}: 
			\begin{itemize}
				\item \textbf{Cause a thread to wait until completing another thread}.
				\item ThreadA calls ThreadB.join(), if it wants to wait until ThreadB completes.
			\end{itemize}
			\bigskip
			\syntaxblock{
				public final void join() throws InterruptedException \\ 
				\\
				public final void join(long milliseconds) throws InterruptedException \\
				\\
				public final void join(long milliseconds, int nanoseconds) throws InterruptedException
			}
			\newpage
			Impact of join() in thread lifecycle:
			\newimage{0.4}{content/chapter17/images/five.png}
						
			\bigskip
			Eg: In below code, main \& new thread are not in sync. There is no guarantee when the new thread's run() method will complete. Hence count is showing 0.

			\codeblockfull{demo.java}{
				class ThreadX extends Thread \{ \\
				\s int count; \\
				\s	public void run() \{ \\
				\s \s		count=10; \\	
				\s	\} \} \\
				public class demo \{ \\
				\s	public static void main(String[] args) throws InterruptedException \{ \\
				\s \s 		ThreadX t1 = new ThreadX(); \\
				\s \s		t1.start(); \\
				\s \s	System.out.println(t1.count); // Output: 0 \\
				\s	\} 	\}
			}
			\bigskip
			Eg2: join() ensures that main thread waits for the new thread to finish.
			\bigskip
			\codeblockfull{demo.java}{
				class ThreadX extends Thread \{ \\
				\s	int count; \\
				\s	public void run() \{ \\
				\s \s		count=10; \\
				\s	\} 	\} 				\\
				public class demo \{ \\
				\s	public static void main(String[] args) throws InterruptedException \{ \\
				\s \s		ThreadX t1 = new ThreadX(); \\
				\s \s		t1.start(); \\
				\s \s		t1.join(); \\
				\s \s		System.out.println(t1.count); // Output: 10 \\
				\s	\} \}	
			}
		\noteblock{
		\textbf{Thread Deadlock}: ThreadA calls join() on ThreadB \& vice-a-versa results in wait forever. This is called \textbf{thread deadlock}.
		}

		\textbf{stop()}:
	\begin{itemize}
		\item \textbf{stop() causes thread to enter dead state}.
		\item stop() is \textbf{deprecated}. Eg:
		\codeblockfull{demo.java}{
			class NewThread extends Thread \{\} \\
			class demo \{ \\
			\s public static void main(String[] args) \{ \\
			\s \s		NewThread t1 = new NewThread(); \\
			\s \s		t1.start(); t1.stop(); \\
			\s	\} \}
		}
	\end{itemize}

	\textbf{interrupt()}:
		\begin{itemize}
			\item Interrupt a sleeping/waiting thread using interrupt().
			\item It works only on sleeping/waiting thread.
			\bigskip
			\syntaxblock{
				public void interrupt()
			}
			\item Eg: In below code, if we comment "Line 1", then main thread wont interrupt child thread. Child thread will execute for loop 10 times. 
			If we are not commenting "Line 1", then main thread interrupts child thread. 
			\bigskip
			\codeblockfull{demo.java}{
				class DemoThread extends Thread \{ \\
				\s	public void run() \{ \\
				\s \s		try\{ \\
				\s \s \s			for (int i = 0; i<5; i++) \{ \\
				\s \s \s				System.out.println("Child Thread-1");   \\
				\s \s \s				Thread.sleep(2000); \\
				\s \s \s	\} 	\} \\
				\s	\s	catch(InterruptedException e)\{ \\
				\s \s \s			System.out.println("OOps thread interrupted"); \\
				\s \s		\} \} \} \\
				public class new3 \{ \\
				\s 	public static void main(String[] args) throws InterruptedException\{ \\
				\s \s		DemoThread t1 = new DemoThread(); \\
				\s \s		t1.start(); \\
				\s \s		t1.interrupt();  // Line 1 \\
				\s \s		System.out.println("End of Main Thread"); \\
				\s 	\} \}
			}
			\bigskip
			\outputblock{
				End of Main Thread \\
				Child Thread-1 \\
				OOps thread interrupted
			}
		\end{itemize}
		
		\textbf{suspend() \& resume()} 
		\begin{itemize}
			\item You can suspend a thread by using suspend() from Thread class.
			\item The thread will be entered into suspended state.
			\item You can resume a suspended thread using resume() from Thread class.
			\syntaxblock{
				public void suspend() \\
				public void resume()
			}
			\item These methods are deprecated and not recommended to use.
		\end{itemize}
		




	\newpage
	\textbf{Comparison tables for yield(), join() and sleep()}
	
	\tablefour{
		\textbf{Property} & \textbf{yield()} & \textbf{join()} & \textbf{sleep()} \\
		\hline
		\textbf{Purpose} &  If a thread wants to pass it’s execution to give the chance for remaining threads of same priority, then we should go for yield() & If a thread wants to wait until completing some other thread, then we should go for join() & If a thread dont want to perform any operation, for a particular amount of time, then we should go for sleep()  \\
		\hline
		\textbf{  Is it overloaded()} & No & yes & yes \\
		\hline
		\textbf{Is it final()} & No & yes & No \\
		\hline
		\textbf{It it throws IE (interrupted exception)?} & No & yes & yes \\
		\hline
		\textbf{Is it native} & yes & no & sleep(long m1) → native
		sleep(long milliseconds, int nanoseconds) → non-native \\
		\hline
		\textbf{It is static} & yes & no & yes \\
	}
	
\end{flushleft}

\newpage