\setlength{\columnsep}{3pt}
\begin{flushleft}

	\begin{itemize}
		\item Attributes represents the \textbf{characteristics} of an object.
		\item They are declared \textbf{inside a class}.
		\item Each object has its own copy of attributes.
		\bigskip
		\syntaxblock{
			access-modifier type identifier;
		}
		where,
		\begin{itemize}
			\item \textbf{access modifiers} can be public, private, protected, default, static, final, transient, volatile.
			\item \textbf{type} can be data-type, classname
			\item \textbf{identifier} is name of attribute
		\end{itemize}
		
	\end{itemize}	
	
	Access modifier in detail:
	\begin{itemize}
		\item \textbf{public attributes:}
		\begin{itemize}
			\item Can be accessed from any other class or package.
			\item Eg 1: Accessing from other class -
			\bigskip
			\codeblock{
				class A \{ \\
				\s	\textbf{public int no;} \\
				\} \\
				\\
				public class Test3 \{ \\
				\s	public static void main(String[] args) \{  \\
				\s \s		A a1 = new A(); \\
				\s \s		a1.no = 1; \\
				\s \s		System.out.println(a1.no);  // Output: 1\\
				\s	\} \\
				\}
			}

			\newpage
			\item Eg 2: Accessing from other package -
			\bigskip
			\codeblockfull{Test1.java}{
				package com.lavatech.www; \\
				public class Test1 \{ \\
				\s	\textbf{public int code;} \\
				\} 
			}
			
			\bigskip
			\codeblockfull{Test2.java}{
				package com.lavatech.info; \\
				import com.lavatech.www.Test1;			 \\
				class Test2 \{ \\
				\s	public static void main(String[] args) \{  \\
				\s \s		Test1 test = new Test1(); \\
				\s \s		test.code = 12345678; \\
				\s \s		System.out.println(test.code); \\
				\s	\} \\
				\}	
			}
			
			\bigskip
			\commandblock{
				\$ javac -d . Test1.java  \\
				\$ javac -d . Test2.java  \\
				\$ java com.lavatech.info.Test2  \\
				12345678
			}		
		\end{itemize}
	
		\newpage
		\item \textbf{private attributes}:
		\begin{itemize}
			\item  Can only be accessed within the same class. 
			\item Not visible to other classes or packages.
			\item Eg: Accessing with the same class:
			\bigskip
			\codeblockfull{A.java} { 
				public class A \{ \\
				\s	private int no=100; \\
				\s	public static void main(String[] args) \{ \\
				\s \s	A a1 = new A(); \\
				\s \s	System.out.println(a1.no); // Output: 100 \\
				\s	\} \\
				\}
			}
		\end{itemize}
	
		\item \textbf{protected attribute}:
		\begin{itemize}
			\item Accessible within the same package or subclass. 
			\item Accessed from subclasses even in different package.
			
			\item Eg 1: Accessing from within subclass: 
			\bigskip
			\codeblockfull{Test.java}{
				class A \{ \\
				\s	protected int no; \\
				\} \\
				public class Test extends A \{  \\
				\s	public static void main(String[] args) \{ \\
				\s \s		Test3 t1 = new Test3(); \\
				\s \s		t1.no = 120; \\
				\s \s		System.out.println(t1.no); // Output: 120 \\
				\s	\}  \\
				\}
			}
		\newpage
			
			\item Eg 2: Accessing from other package:
			\bigskip
			\codeblockfull{Test1.java}{
				package com.lavatech.www; \\
				public class Test1 \{  \\
				\s	protected int no; \\
				\}	
			}
			\bigskip
			\codeblockfull{Test2.java}{
				package com.lavatech.info; \\
				import com.lavatech.www.Test1; \\
				class Test2 extends Test1 \{  \\
				\s	public static void main(String[] args) \{ \\
				\s \s		Test2 test = new Test2(); \\
				\s \s		test.no = 150; \\
				\s \s		System.out.println(test.no); \\
				\s	\} \\
				\}
			}
			
			\bigskip
			\commandblock{
				\$ javac -d . Test1.java  \\
				\$ javac -d . Test2.java  \\
				\$ java com.lavatech.info.Test2  \\
				150
			}	
		\end{itemize}
	
		\bigskip
		\item \textbf{default (package-private):} 
		\begin{itemize}
			\item No access modifier specified is default attribute. 
			\item Can be accessed within the same package but not from other packages.
			\item Eg: Accessing from same class -
			\bigskip
			\codeblockfull{Test.java}{ 
				class A \{ \s	\\
				\s \textbf{int no;}  \\
				\} \\
				public class Test3 \{  \\
				\s public static void main(String[] args) \{  \\
				\s \s		A a1 = new A(); \\
				\s \s		System.out.println(a1.no); // Output: 0 \\
				\s	\}  \\
				\}	
			}
		\end{itemize}
	
		\item \textbf{static}:
		\begin{itemize}
			\item A static attribute belongs to the class rather than object.
			\item It is shared among all instances of the class.
			\item Eg: Accessing varaible using class name -
			\bigskip
			\codeblockfull{Test.java}{
				class A \{ \\
				\s static int count; \\
				\} \\
				public class Test3 \{ \\
				\s	public static void main(String[] args) \{ \\
				\s \s		System.out.println(A.count); // Output: 0 \\
				\s	\} \\
				\}
			}
		\end{itemize}
		\newpage
		
		\item \textbf{final}:
		\begin{itemize}
			\item Can be assigned a value once, and its value cannot be changed. 
			\item Key points:
			\begin{itemize}
				\item \textbf{Initialization:} Declared \& initialized together or within the \textbf{constructor} of the class.
				\item \textbf{Naming convention}: Should be in \textbf{uppercase}, separted with underscores (e.g., FINAL\_VARIABLE).
				\item \textbf{For primitive type:} Value assigned at initialization cannot be modified. Eg:
				\bigskip
				\codeblockfull{A.java}{
					public class A \{ \\
					\s	final int count=100; \\
					\s	public static void main(String[] args) \{ \\
					\s \s		A a1 = new A(); \\
					\s	System.out.println(a1.count);   // Output: 100 \\
					\s	\} \\
					\}
				}
				\bigskip
				\item \textbf{For reference type:} Reference itself cannot be changed, but the state of the object it refers to can be modified. Eg:
				\bigskip
				\codeblockfull{Test.java}{
					public class A \{ \\
					\s	final StringBuffer b1=new StringBuffer("Lava"); \\
					\s	public static void main(String[] args) \{ \\
					\s \s		A a1 = new A(); \\
					\s \s		a1.b1.append("tech"); \\
					\s		System.out.println(a1.b1); // Output: Lavatech \\
					\s	\} \\ \}
				}
				\newpage
				\item \textbf{final static:} Creates class-level constant accessible without an object. Eg:
				\bigskip
				\codeblockfull{A.java}{
					public class A \{ \\
					\s	static final int count=100; \\
					\s	public static void main(String[] args) \{  \\
					\s		System.out.println(count); // Output: 100 \\
					\s	\} \\
					\}
				}
				\bigskip				
				\item \textbf{protected final:} Creates read-only attribute accessible within the same package or subclass.
				\bigskip
				\codeblockfull{Test.java}{
					class Constant \{ \\
					\s	protected final int MAX\_VALUE = 500; \\
					\} 
				}
			\end{itemize}
		\end{itemize}
	
		\item \textbf{transient}: 
		\begin{itemize}
			\item Use transient keyword in serialisation context.
			\item More on this is in serialisation chapter.
		\end{itemize}
		\bigskip
		\item \textbf{volatile attributes}: 
		\begin{itemize}
			\item A volatile instance variable is used in multithreaded programs.
			\item More on this in multi-threading chapter.		
		\end{itemize}
	\end{itemize}
	
\end{flushleft}

\newpage

