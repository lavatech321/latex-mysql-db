\setlength{\columnsep}{3pt}
\begin{flushleft}
	
	\begin{itemize}
		\item Using access modifier, class can provide more information to the JVM like:
		\begin{itemize}
			\item Whether the class is accessible from anywhere or not
			\item Whether child class creation is possible or not
			\item Whether object creation is possible or not
		\end{itemize}
		\item The only applicable modifiers for top-level classes are:
		\begin{itemize}
			\item public
			\item default
			\item final
			\item abstract
			\item strictfp
		\end{itemize}
		\item But, for inner classes, the applicable modifiers are:
		\begin{itemize}
			\item private
			\item protected 
			\item static
		\end{itemize}
	
	\end{itemize}

	\newimage{0.4}{content/chapter11/images/class.png}
	
	\noteblock{
	In Java, there are only access modifiers, there is no word like access specifier.
	
	}
	
	Let's see these class level access modifiers in detail.
	
	\newpage
	
	\textbf{public classes}:
	\begin{itemize}
		\item  If a class is declared as public then we can access that class from anywhere.
		\item Eg:
		\bigskip
		\codeblockfull{Test1.java}{
			package com.lavatech.www; \\
			\textbf{public class Test1} \{  \\
			\s	public void message()\{ \\
			\s \s		System.out.println("Hard work pays off!"); \\
			\s	\} \\
			\}
		}
		\bigskip
		\codeblockfull{Test2.java}{
			package com.lavatech.info; \\
			\textbf{import com.lavatech.www.Test1;} \\
			public class Test2 \{ \\
			\s public static void main(String[] args) \{ \\
			\s \s		\textbf{Test1 t1 = new Test1();} \\
			\s \s		t1.message(); \\
			\s	\} \\
			\}	
		}
		\bigskip
		\commandblock{
		\$ javac -d . Test1.java  \\
		\$ javac -d . Test2.java  \\
		\$ java com.lavatech.info.Test2  \\
		Hard work pays off!
		}
	\end{itemize}
	
	\newpage
	\textbf{default classes}:
	
	\begin{itemize}
		\item A default class \textbf{does not have an access modifier} specified. 
		\item Also known as a \textbf{package-private class} 
		\item It is accessible \textbf{only within the same package}.
		\item If no access modifier is specified, the class have default access.
		\item Eg:
		\bigskip
		\codeblockfull{Test1.java}{
			package com.lavatech.www; \\
			class Test1 \{ \\
			\s public void message()\{ \\
			\s \s		System.out.println("Hard work pays off!"); \\
			\s	\} \\
			\}
		}
		\bigskip
		\codeblockfull{Test2.java}{
			package com.lavatech.info; \\
			import com.lavatech.www.Test1;
		}
		\bigskip
		\commandblock{
			\$ javac -d . Test2.java  \\
			\color{red}
			Test2.java:2: error: Test1 is not public in com.lavatech.www; cannot be accessed from outside package \\
			import com.lavatech.www.Test1; \\
			1 error 
		}
	\end{itemize}

	\textbf{final class}:
	\begin{itemize}
		\item Final class can’t be inherited. 
		\item Final class method are always final ie. method cannot be overridden.
		\item Final class attributes need not be final.
		\newpage
		\item Eg:
		\bigskip
		\codeblockfull{Test.java}{
			final class A \{\} \\
			class Test \textbf{extends} A \{\}  \xmark
		}
		\item Disadvantage of final keyword:
		\begin{itemize}
			\item Missing inheritance (due to final classes)
			\item Polymorphism (due to final methods)
		\end{itemize}
		
	\end{itemize}	

	\textbf{abstract class}
	\begin{itemize}
		\item Abstract classes cannot have any objects.
		\item More on abstract class is mentioned in Abstract class chapter.
	\end{itemize}

	\textbf{strictfp class}
	\begin{itemize}
		\item Introduced in Java1.2 version, strictfp class ensures all methods in the class \& its subclasses provide same result for the floating points operations on any platform. 
		\item Eg:
		\codeblockfull{Test.java}{
		\textbf{strictfp class A} \{ \\
		\s	double num1 = 10e+102; \\
		\s	double num2 = 6e+08; \\
		\s	double calculate() \{ \\
		\s \s		return num1 + num2; \\
		\s	\} \} \\
		public class Test \{  \\
		\s	public static void main(String[] args) \{  \\
		\s \s		A a1 = new A();      \\
		\s 	System.out.println(a1.calculate()); // Output: 1.0E103 \\
		\s	\}  \}
		}
		
	\end{itemize}
	
\end{flushleft}

\newpage
