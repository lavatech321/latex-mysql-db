\setlength{\columnsep}{3pt}
\begin{flushleft}
	
	\begin{itemize}
		\item Static initializer is used to initialize static attributes.
		\item It is executed only once when the class is loaded into memory. 
		\item Multiple static blocks are allowed \& will execute from top to bottom.
		\bigskip
		\syntaxblock{
			class Classname \{ \\
			\s static \{ \\
			\s \s // codehere \\
			\s\} \\
			\}
		}
		\item Eg:
		\bigskip
		\codeblockfull{Test.java}{
			class Test \{ \\
			\s	static int count; \\
			\s	\textbf{static \{}  \\
			\s \s		\textbf{count = 10;} \\
			\s \s		\textbf{System.out.println("Count: " + count);} \\
			\s	\textbf{\}} \\
			\s	public static void main(String[] args) \{ \\	
			\s	\s   System.out.println("End"); \\
			\s	\} \\
			\}
		}
		\bigskip
		\outputblock{
			Count: 10 \\
			End
		}
		
	\end{itemize}	
	\newpage
	\textbf{Did you know?}
	\quest{Without main method is it possible to print some statements to console?}{
		\begin{itemize}
			\item Before Java 1.7, yes by using static blocks.
			\item Eg:
			\codeblock{
				class Test \{ \\
				\s static int count;  \\
				\s	static \{ \\
				\s \s	count = 10; \\
				\s \s		System.out.println("Count: " + count); \\
				\s	\} \\
				\}
			}
			\item After Java 1.7, this code will result in runtime error.
		\end{itemize}		
	}
	
\end{flushleft}
