\setlength{\columnsep}{3pt}
\begin{flushleft}

	A table is in 4NF if:
	\begin{itemize}
		\item It is already in BCNF.
		\item 4NF helps when you have multiple, independent pieces of information about the same entity that don't depend on each other but are stored together. 
		\item It separates these into different tables to avoid redundancy.
		
		\item Example (without 4NF): Consider a table where we store information about a student's courses and their hobbies:
		
		\newimage{0.36}{content/chapter11/images/4nf1.png}
		
		\item This table has a multi-valued dependency because the courses and hobbies are unrelated, but they are being stored together.
		
		\item To bring it into 4NF, we need to split the table:

		\newimage{0.36}{content/chapter11/images/4nf2.png}		
		
		
	\end{itemize}	
	
\end{flushleft}

\newpage

