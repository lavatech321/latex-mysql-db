\setlength{\columnsep}{3pt}
\begin{flushleft}

	\textbf{private constructor}
	\tablethree{
		\hline
		Syntax & Accessibility & Uses \\
		\hline
		private Classname \{\} & Only accessible within the same class & Used to \textbf{prevent direct instantiation} of the class \\
		\hline
	}
	\newline
	Eg:
		\codeblockfull{A.java}{
				class A \{ \\
				\s \textbf{private A()} \{ \\
				\s \s		System.out.println("Working"); \\
				\s	\} \\
				\s 	public static void main(String[] args) \{ \\
				\s \s 		\textbf{A a1 = new A();}  // Output: Working \\
				\s	\} \} \\
				public class B extends A \{ \\
				\s 	public static void main(String[] args) \{ \\
				\s \s	\textbf{B b1 = new B();} \xmark \s // Cannot instantiate \\
				\s \}
				\} 
		}
		
		\newpage
		\textbf{protected constructor}		
		\tabletwo{
			Syntax & Accessibility \\
			\hline
			protected Classname \{\} & \begin{itemize}
				\item Can be access from same class \& subclasses or classes within the same package
				\item Cannot be accessed from different package where class is not subclass.
			\end{itemize} 
			 \\
		}
		Eg:
		\codeblockfull{B.java}{
				class A \{ \\
				\s \textbf{protected A()} \{ \\
				\s \s		System.out.print("One"); \\
				\s	\}  \\
				\} \\
				public class B extends A \{ \\
				\s	public B() \{ \\
				\s \s		System.out.print("Two"); \\
				\s	\} \\
				\s	public static void main(String[] args) \{ \\
				\s \s		\textbf{B b1 = new B();} // Output: OneTwo \\
				\s	\} 	 \\
				\}
			}
		\newpage
		\textbf{public constructor}

		\tabletwo{
			Syntax & Accessibility \\
			\hline
			public Classname \{\} & Accessible from: \begin{itemize}
				\item Other classes
				\item Subclasses
				\item Different packages
			\end{itemize} \\
		}
			
		\begin{itemize}
			\item Eg 1:
			\codeblockfull{B.java}{
				class A \{ \\
				\s \textbf{public A()} \{ \\
				\s \s	System.out.print("A"); \\
				\s	\} \\
				\} \\
				public class B extends A \{ \\
				\s	public B() \{ \\
				\s \s		System.out.print("B"); \\
				\s	\} \\
				\s	public static void main(String[] args) \{ \\
				\s \s		\textbf{B b1 = new B();} // Output: AB \\
				\s	\} 	\\
				\}
			}
		\end{itemize}
	
		\newpage
		\textbf{default constructor}: Means no access modifier specified.
		\newline
		\tablethree{
			\hline
			Syntax & Accessibility & Uses \\
			\hline
			Classname \{\} & Accessible within the same package but not accessible from classes outside the package.
			& 
			If no constructor is defined, a default constructor is automatically provided by the compiler. \\
			\hline
		}
		
		\bigskip
		Eg:
			\codeblockfull{B.java}{
				class A \{ \\
				\s \textbf{A()} \{ \\
				\s \s		System.out.println("A constructor called"); \\
				\s	\} \\
				\} \\
				public class B extends A \{ \\
				\s	public B() \{ \\
				\s \s		System.out.println("B constructor called"); \\
				\s	\} \\
				\s	public static void main(String[] args) \{ \\
				\s \s		\textbf{B b1 = new B();} // Output: AB \\
				\s	\} 	 \\
				\}
			}
	
	
\end{flushleft}

\newpage
