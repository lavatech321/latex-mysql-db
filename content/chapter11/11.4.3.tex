\setlength{\columnsep}{3pt}
\begin{flushleft}
	
	\begin{itemize}
		\item Method overriding means method in the child class has the \textbf{same name, return type, arguments} as a method in the parent class.
		\item Key points:
		\begin{itemize}
			\item \textbf{Signature:} \textbf{Same signature} (name, return type, arguments) is required for overriding \& overridden method.
			\item \textbf{Access modifier:} Overriding method \textbf{cannot have a more restrictive access modifier} than overridden method.
			\item \textbf{@Override annotation:} Optionally, use \textbf{@Override} annotation.
		\end{itemize}
		\item Eg:
		\bigskip
		\codeblockfull{Test.java}{
			class Animal \{ \\
			\s	public void makeSound() \{ \\
			\s \s		System.out.println("Animal makes a sound"); \\
			\s	\} \\
			\} \\
			class Cat extends Animal \{ \\
			\s	\textbf{@Override} \\
			\s 	\textbf{public void makeSound()} \{ \\
			\s \s		System.out.println("Cat meows"); \\
			\s	\} \\
			\} \\
			class Test \{ \\
			\s	public static void main(String[] args) \{ \\
			\s \s		Animal animal1 = new Cat(); \\
			\s \s		\textbf{animal1.makeSound();} // Output: Cat meows \\
			\s	\} \\
			\}
		}
	\end{itemize}
	
\end{flushleft}

\newpage
