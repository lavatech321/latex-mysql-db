\setlength{\columnsep}{3pt}
\begin{flushleft}
	
	Method hiding is same overriding except:
	\tabletwo{
		\hline
		Method hiding & Method overriding \\
		\hline
		It is like sticking poster on black board & It is like erasing board and writing new content \\
		\hline
		Require static parent and child method & Require non-static parent and child method \\
		\hline
		Compiler does method resolution & JVM does method resolution \\
		\hline
		Known as compile-time/static polymorphism or early binding & Known as runtime/dynamic polymorphism or late binding \\
		\hline
	}
	Eg:
	\codeblockfull{Test.java}{
		class Parent \{ \\
		\s public static void m1() \{ \\
		\s \s	System.out.println("Parent"); \\
		\s	\} \} \\
		class Child extends Parent \{ \\
		\s	public static void m1() \{ \\
		\s \s		System.out.println("Child"); \\
		\s 	\} \}  \\
		class Test \{ \\
		\s	public static void main(String[] args) \{ \\
		\s \s		Parent p1 = new Parent(); \\
		\s \s		p1.m1();  \s // Output: Parent \\
		\s \s		Child c1 = new Child();  \\
		\s \s		c1.m1();  \s // Output: Child \\
		\s \s		Parent p2 = new Child(); \\
		\s \s		p2.m1();  \s  // Output: Parent \\
		\s	\} \\
		\}
	}	
	
\end{flushleft}

\newpage
