\setlength{\columnsep}{3pt}
\begin{flushleft}

	\begin{itemize}
		\item HashMap class implements Map interface.
		\item The underlying data structure is \textbf{hash table}.
		\newimage{0.2}{content/chapter14/images/hash.png}		
		\item Features:
		\begin{itemize}
			\item It is \textbf{resizable array}.
			\item \textbf{Duplicate keys not allowed}, but duplicate values are allowed.
			\item \textbf{Insertion} order is not \textbf{preserved} and it is based on hashCode of objects.
			\item \textbf{Hetrogenous keys and values are allowed.}
			\item \textbf{Null key is allowed (only once) , null value is allowed any number of times}.
		\end{itemize}
		\item Best choice for frequent \textbf{search operation}.		
	\end{itemize}
	
	\textbf{Ways of initializing HashMap}
	
	\begin{itemize}
		\item Creating am empty HashMap:
		\bigskip
		\syntaxblock{
			HashMap s1 = new HashMap(); \\
		}
		\bigskip
		\item \textbf{HashMap capacity \& fill ratio}:
		\begin{itemize}
			\item Above syntax will create HashMap with \textbf{default capacity of 16}.
			\item Default fill ratio is \textbf{0.75}: Fill ratio is after filling 16th element, when 17th element is added, it’s new capacity should be increased by 75\%. Fill ratio is also called Load factor			
		\end{itemize}		
		\item Creating am empty HashMap with initial capacity:
		\bigskip
		\syntaxblock{
			HashMap h = new HashMap(int initialcapacity);
		}
		
		\item Creating an empty HashMap with initial capacity and fill ratio:
		\bigskip
		
		\syntaxblock{
			HashMap h = new HashMap(int initialcapacity, float fillratio);
		}
		Eg:
		\codeblock{
			HashMap h = new HashMap(1000, 0.9);
		}
	\end{itemize}
	
	\textbf{HashMap specific methods}
	
	\begin{itemize}
		\item Return a collection view of the mappings contained in this map.
		\syntaxblock{
			Set entrySet()	
		}
		\bigskip
		\item Return a set view of the keys contained in this map.
		\syntaxblock{
			Set keySet()	
		}
		\bigskip
		\item Return a set view of the keys contained in this map.
		\syntaxblock{
			Set keySet()	
		}
		\bigskip
		\item Returns a collection view of the values contained in the map.
		\syntaxblock{
			Collection<V> values()	
		}
		\bigskip
		\item Replaces the specified value for a specified key.
		\syntaxblock{
			Value replace(K key, V value)	
		}
	\end{itemize}

	\textbf{Example Program}: Display HashMap operations:
	\codeblockfull{Test.java}{
		import java.util.*; \\
		class one \{ \\
		\s public static void main(String[] args) \{ \\
		\s \s		\textbf{HashMap map = new HashMap();} \\
		\s \s		\textbf{map.put("Raman",101);} \\
		\s \s		map.put("Ravi",102); \\
		\s \s		map.put("Kavi",103); \\
		\s \s		System.out.println(map); \\
		\s \s		\textbf{Set s = map.keySet()}; \\
		\s \s		System.out.println(s); \\
		\s \s		\textbf{Collection c = map.values();}  \\
		\s \s		System.out.println(c); \\
		\s \s		\textbf{Set s2 = map.entrySet();} \\
		\s \s		System.out.println(s2); \\
		\s \s		System.out.println(\textbf{map.replace("Ravi",106)}); \\
		\s \s		System.out.println(map);		 \\
		\s	\} \}	
	}

	\outputblock{
		\{Ravi=102, Raman=101, Kavi=103\} \\
		\textbf{[}Ravi, Raman, Kavi\textbf{]} \\
		\textbf{[}102, 101, 103\textbf{]} \\
		\textbf{[}Ravi=102, Raman=101, Kavi=103\textbf{]} \\
		102 \\
		\{Ravi=106, Raman=101, Kavi=103\}
	}
	
\end{flushleft}
\newpage


