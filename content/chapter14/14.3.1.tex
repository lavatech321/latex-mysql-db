\setlength{\columnsep}{3pt}
\begin{flushleft}

	\begin{itemize}
		\item HashSet class is \textbf{unordered} collection of \textbf{heterogeneous} objects wherein \textbf{duplicates are not allowed} \& \textbf{null insertion is possible only once}.
		\item  HashSet class implements Set interface.
		\item Best choice for \textbf{search} operation.		
		\item Below is an example of \textbf{hash table}.
		\newimage{0.2}{content/chapter14/images/hash.png}		
	\end{itemize}
	
	\textbf{Ways of initializing HashSet}
	
	\begin{itemize}
		\item Creating an empty HashSet:
		\bigskip
		\syntaxblock{
			HashSet s1 = new HashSet(); 
		}
		\bigskip
		\item \textbf{HashSet capacity \& fill ratio}:
		\begin{itemize}
			\item Above syntax will create HashSet with \textbf{default capacity of 16}.
			\item Default fill ratio is \textbf{0.75}: Fill ratio is after filling 16th element, when 17th element is added, it’s new capacity should be increased by 75\%. Fill ratio is also called Load factor			
		\end{itemize}		
		\item Creating am empty HashSet with initial capacity:
		\bigskip
		
		\syntaxblock{
			HashSet h = new HashSet(int initialcapacity);
		}
		\newpage
		\item Creating an empty HashSet with initial capacity and fill ratio:
		\bigskip
		
		\syntaxblock{
			HashSet h = new HashSet(int initialcapacity, float fillratio);
		}
		Eg:
		\codeblock{
			HashSet h = new HashSet(1000, 0.9);
		}
	\end{itemize}
	
	\textbf{Example Program} \newline
	Java program to display HashSet operations:
	\codeblockfull{Test.java}{
		import java.util.*; \\
		class one \{  \\
		\s	public static void main(String[] args) \{ \\
		\s \s		\textbf{HashSet set = new HashSet()}; \\
		\s \s		\textbf{set.add("Apple")}; \\
		\s \s		set.add("Mango"); \\
		\s \s		set.add("Apple"); \\
		\s \s		System.out.println(set); \\
		\s \s		\textbf{set.remove("Mango")}; \\
		\s \s		System.out.println(set); \\
		\s	\} \\
		\}
	}
	\bigskip
	\outputblock{
		\textbf{[}Apple, Mango\textbf{]} \\
		\textbf{[}Apple\textbf{]}
	}
\end{flushleft}
\newpage

