\setlength{\columnsep}{3pt}
\begin{flushleft}
	
	\begin{itemize}
		\item Enumeration interface is used to \textbf{enumerate} (or iterate over) elements in \textbf{certain legacy collection classes} like:
		\begin{itemize}
			\item Vector
			\item Stack
			\item Hashtable
			\item Properties
		\end{itemize}
		\item It is part of the \textbf{java.util} package.		
	\end{itemize}
	
	\textbf{Enumeration specific methods:}
	
	\begin{itemize}
		\item Check whether collection is empty or not:
		\syntaxblock{
			boolean hasMoreElements()
		}
		\item Check for the next element:
		\syntaxblock{
			Object nextElement()
		}
	\end{itemize}

	\noteblock{
		\begin{itemize}
			\item Enumeration concept is applicable only for legacy classes.
			\item Only read access is allowed, and you cant perform remove operation. 
			\item To overcome these limitations, go for iterator
		\end{itemize}
	}
	
	\newpage
	\textbf{Example:}
	\newline
	Java program to iterate over vector object and display even numbers only:
	\codeblockfull{Test.java}{
		import java.util.*; \\
		class Test\{ \\
		\s public static void main(String[] args) \{ \\
		\s \s		Vector v1 = new Vector(); \\
		\s \s		v1.add(340);  \\
		\s \s		v1.add(123);  \\ 
		\s \s		v1.add(342);  \\
		\s \s		v1.add(541);  \\
		\s \s		Enumeration e = v1.elements(); \\
		\s \s		while (e.hasMoreElements()) \{ \\
		\s \s \s			Integer i = (Integer) e.nextElement(); \\
		\s \s \s			if(i\%2 == 0) \\
		\s \s \s \s			System.out.println(i); \\
		\s \s \s		\} \\
		\s	\} \\
		\}	
	}
	\bigskip
	\outputblock{
		340 \\
		342 
	}

	\quest{If Enumeration, Iterator and ListIterator is an interface, then how we are creating it’s object?}{
	Vector class internally implements Enumeration, Iterator and ListIterator as \textbf{anonymous class}.}
	
	
\end{flushleft}

\newpage

