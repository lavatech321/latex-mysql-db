\setlength{\columnsep}{3pt}
\begin{flushleft}
	
	\begin{itemize}
		\item LinkedList are \textbf{ordered \& indexed} collection of \textbf{hetrogenous objects} wherein \textbf{duplicate objects \& null} is allowed. 
		\newimage{0.28}{content/chapter14/images/newlist.png}

		\item LinkedList class implements List interface.
		\item LinkedList class is used to \textbf{develop stacks and queues}.

	\end{itemize}

	\textbf{Difference between ArrayList and LinkedList}
	\tablethree{
		& \textbf{ArrayList} & \textbf{LinkedList} \\
		\hline
		Best choice for & Retrieval operation & Insertion/Deletion operation in the middle \\
		\hline
		Worst choice for & Insertion/Deletion operation in the middle & Retrieval operation \\
		\hline
		Object storage \newline order & Consecutive memory location & Not stored in consecutive memory locations \\
		\hline
		Capacity & Allows mentioning capacity & Does not allow mentioning capacity \\
	}
	
	\newpage
	\textbf{Ways of initializing LinkedList}

	\begin{itemize}
		\item Creating an empty LinkedList:
		\bigskip
		
		\syntaxblock{
			// To create LinkedList of hetrogenous elements: \\
			LinkedList numbers = new LinkedList(); \\
			\\
			// To create LinkedList of homogenous elements: \\
			LinkedList<ClassName> list = new LinkedList<ClassName>(); 
		}
		Eg:
		\codeblock{
			LinkedList numbers = new LinkedList(); \\
			LinkedList<Integer> list = new LinkedList<Integer>(); 
		}
		\bigskip
		\item Initialization with data:
		\begin{itemize}
			\item Using \textbf{Arrays.asList()} method:
			\bigskip
			\codeblock{
				import java.util.*; \\
				String[] namesArray = \{"Jim", "Jane", "Alice"\}; \\
				LinkedList<String> namesList = new LinkedList<>(Arrays.asList(namesArray)); 
			}
			
			\item Using \textbf{List.of} method:
			\bigskip
			\codeblock{
				LinkedList<String> colors = new LinkedList<>(List.of("Red", "Green", "Blue"));
			}
			
			\item Using \textbf{"var"} and \textbf{"List.of"}: 
			\bigskip
			\codeblock{
				var fruits = new LinkedList<>(List.of("Apple", "Banana", "Orange"));
			}
		
		\end{itemize}
	
	\end{itemize}
	

	
	\bigskip
	\textbf{Methods specific to LinkedList}:
	\begin{itemize}
		\item Add first member to LinkedList:
		\syntaxblock{
			void addFirst(Object o)
		}
		\bigskip
		\item Add last member to LinkedList:
		\syntaxblock{
			void lastFirst(Object o)
		}
		\bigskip
		\item Get first member of LinkedList:
		\syntaxblock{
			Object getFirst()
		}
		\bigskip
		\item Get last member of LinkedList:
		\syntaxblock{
			Object getLast()
		}
		\bigskip
		\item Remove first member of LinkedList:
		\syntaxblock{
			Object removeFirst()
		}
		\bigskip
		\item Remove last member of LinkedList:
		\syntaxblock{
			Object removeLast()
		}

	\end{itemize}
	
	\newpage
	
	\textbf{Example}
	
	Create an empty LinkedList and perform add, remove operation and display it's size:
	
	\codeblockfull{Test.java}{
		import java.util.LinkedList; \\
		class Test \{  \\
		\s public static void main(String[] args) \{ \\
		\s \s		LinkedList l1 = new LinkedList(); \\
		\s \s		\textbf{l1.add("Jack")};  \\
		\s \s		\textbf{l1.add(null)}; \\
		\s \s		\textbf{l1.remove("Jack")};  \\
		\s \s		\textbf{l1.add(2,"Jimmy")};  \\
		\s \s		int count = \textbf{l1.size()};  \\
		\s \s		\textbf{l1.set(0,"Jack Sparrow")};  \\
		\s \s		\textbf{l1.addFirst("First")}; \\
		\s \s		\textbf{l1.addLast("Last")}; \\
		\s \s		\textbf{l1.removeFirst()}; \\
		\s \s		\textbf{l1.removeLast()}; \\
		\s	\} 	\\
		\}
	}
	
\end{flushleft}
\newpage


