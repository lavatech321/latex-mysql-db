\setlength{\columnsep}{3pt}
\begin{flushleft}

	\begin{itemize}
		\item A Pandas series is like a column in a table.
		\item It is a one-dimensional array holding data of any type.
		\item Eg:
		\bigskip
		\codeblock{
			import pandas as pd \\
			data = ['apple','mango','chikoo'] \\
			var = pd.Series(data) \\
			print(var)
		}
		\bigskip
		\outputblock{
			0  \s \s   apple \\
			1  \s \s  mango \\
			2  \s \s  chikoo \\
			dtype: object
		}	
	\end{itemize}	
	
	\textbf{Labels}
	\begin{itemize}
		\item If nothing else is specified, the values are labeled with their index number. 
		\item First value has index 0, second value has index 1 etc.
		\bigskip
		\codeblock{
			import pandas as pd \\
			data = ['apple','mango','chikoo'] \\
			var = pd.Series(data) \\
			print(var[0]) \\
			print(var[1]) \\
			print(var[2]) 
		}
		\outputblock{
			apple \\
			mango \\
			chikoo
		}
	
		\item \textbf{Create your own labels}
		\bigskip
		\codeblock{
			import pandas as pd \\
			a = [1, 7, 2] \\
			myvar = pd.Series(a, index = ["x", "y", "z"]) \\
			print(myvar) \\
			print(myvar['x']) \\
			print(myvar['y'])
		}
		\bigskip
		\outputblock{
			x  \s \s   1 \\
			y  \s \s  7 \\
			z  \s \s  2 \\
			dtype: int64 \\
			1  \\
			7
		}

		\item \textbf{Key/Value Objects as Series:}
		\bigskip
		\codeblock{
			import pandas as pd \\
			a = {'name':'Ravi','age':56,'country':'India'} \\
			myvar = pd.Series(a) \\
			print(myvar)
		}	
		\bigskip
		\outputblock{
			name   \s  \s    Ravi \\
			age    \s  \s      56 \\
			country  \s \s  India \\
			dtype: object
		}
		
	\end{itemize}
	
	

	
	
	
	
	
\end{flushleft}

\newpage

