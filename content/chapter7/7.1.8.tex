\setlength{\columnsep}{3pt}
\begin{flushleft}
	\begin{itemize}
		\item The \textbf{"|"} character is pipe operator.
		\item A pipe is a technique for passing output of one command to another command:
		\bigskip
		\bigskip
		\begin{figure}[h!]
			\centering
			\includegraphics[scale=0.5]{content/chapter7/images/pipe.png}
			\caption{Pipe operator}
			\label{fig:pipe_operator}
		\end{figure}

		\bigskip
		\begin{tcolorbox}[breakable,notitle,boxrule=0pt,colback=pink,colframe=pink]
			\color{black}
			\fontdimen2\font=9pt
			Syntax: command1 | command2 | command3
			\fontdimen2\font=4pt
		\end{tcolorbox}
	
		Eg: 
		\bigskip
		\begin{tcolorbox}[breakable,notitle,boxrule=-0pt,colback=black,colframe=black]
			\color{green}
			\fontdimen2\font=9pt
			\$ who | wc -l 			\color{yellow} \#[press Enter]
			\newline
			\color{white}
			4
			\fontdimen2\font=4pt
		\end{tcolorbox}
	
	\end{itemize}
	
	\paragraph{Pipelines, redirection, and tee}
	\textbf{tee}: In a pipeline, tee will copy its standard input to its standard output(i.e monitor screen) and will also redirect its standard output to the files named as arguments to the command.
	\begin{tcolorbox}[breakable,notitle,boxrule=0pt,colback=pink,colframe=pink]
		\color{black}
		\fontdimen2\font=9pt
		Syntax: command1 | tee file-name
		\fontdimen2\font=4pt
	\end{tcolorbox}
	Eg:
	\begin{tcolorbox}[breakable,notitle,boxrule=-0pt,colback=black,colframe=black]
		\color{green}
		\fontdimen2\font=9pt
		\$ ls -t | head -n 10 | tee /tmp/ten-last-changed-files
		\newline
		\$ ls -l | tee /tmp/saved-output | less
		\fontdimen2\font=4pt
	\end{tcolorbox}
		
	
	
\end{flushleft}

\newpage

