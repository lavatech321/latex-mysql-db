\setlength{\columnsep}{3pt}
\begin{flushleft}
	\begin{itemize}
		\item JVM can \textbf{save complete \& not part of object} to the file during serialisation.
		\item \textbf{Using externalisation, programmer can save part or complete object to a file.}
		\item Performance wise, externalisation is best choice.
		\item Externalizable is child interface of Serializable.
		
		\newimage{0.4}{content/chapter18/images/obj4.png}

		\item Rules for externalisation:
		\begin{itemize}
			\item \textbf{Override readExternal() \& writeExternal()} as they execute automatically during serialisation \& deserialisation.
			\item Externalisable implemented class should contain “public no-arg” constructor. 
		\end{itemize}
		\item Eg: Creates an object and overrides readExternal() and writeExternal() to write part of object to file "abc.ser":
		\newpage
		\codeblockfull{demo.java}{
			import java.io.*; \\
			\textbf{class External implements Externalizable \{} \\
			\s	String s; \\
			\s	int i, j; \\
			\s	\textbf{public External() \{\}} \\
			\s	\textbf{public External(String s, int i, int j)} \{ \\
			\s \s		\textbf{this.s = s; this.i = i; this.j = j;} \\
			\s	\} \\
			\s	\textbf{public void writeExternal(ObjectOutput out) throws IOException} \{ \\
			\s \s		\textbf{out.writeObject(s);} \\
			\s \s		\textbf{out.writeInt(i);} \\
			\s	\} \\
			\s	\textbf{public void readExternal(ObjectInput in) throws IOException, ClassNotFoundException} \{ \\
			\s \s		\textbf{s = (String)in.readObject();} \\
			\s \s		\textbf{i = in.readInt();} \\
			\s	\}  \\
			public static void main(String[] args) throws Exception \{  \\
			\s		\textbf{External t1 = new External("lavatech",10,20);} \\
			\s		\textbf{FileOutputStream fos = new FileOutputStream("abc.ser");} \\
			\s		\textbf{ObjectOutputStream oos = new ObjectOutputStream(fos);} \\
			\s \s		\textbf{oos.writeObject(t1);} \\
			\s		\textbf{FileInputStream fis = new FileInputStream("abc.ser");} \\
			\s		\textbf{ObjectInputStream ois = new ObjectInputStream(fis);} \\
			\s		\textbf{External t2 = (External)ois.readObject();} \\
			\s \s		System.out.println(t2.s+"..."+t2.i+"..."+t2.j); \\
			\s	\} \}
			
		}
		\bigskip
		\outputblock{
			lavatech...10...0
		}
	\end{itemize}
	
	\textbf{Serialisation V/S Externalisation}
	\bigskip
	\tabletwo{
		\textbf{Serialisation} & \textbf{Externalisation} \\
		\hline
		Meant for \textbf{default} serialisation & Meant for \textbf{customised} serialisation \\
		\hline
		\textbf{JVM} has control & \textbf{Programmer} has control \\
		\hline
		Save \textbf{complete object} to the file & 	Save \textbf{either complete or part of the object} to the file \\
		\hline
		\textbf{Performance is low} & \textbf{Performance is high} \\
		\hline
		It is a marker interface \& does'nt contain any methods & Contains 2 methods: \textbf{writeExternal()} and \textbf{readExternal()} \\
		\hline
		Does not contain  "public no-argument" constructor & Contain public no-argument constructor \\
		\hline
		\textbf{Tranient used} in serialisation & \textbf{Transient cant} be used in externalisation \\	
	}
	
\end{flushleft}

