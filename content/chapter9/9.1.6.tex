\setlength{\columnsep}{3pt}
\begin{flushleft}
	
	\begin{itemize}
		\item A correlated subquery is a subquery which is executed repeatedly once for each row of the main query.
		\item A correlated subquery is a nested query which is executed once for each 'candidate row' selected by the main.
		
		
		\codeblock{
			create table employee( \\
			deptid number,	\\
			name varchar2(20),	\\
			salary number	\\
			);	\\
			insert into employee values(101, 'Ram',34000);	\\
			insert into employee values(102, 'Ravi',56000);	\\
			insert into employee values(103, 'Kavi',16000);	\\
			insert into employee values(104, 'Tanvi',34000);	\\
			insert into employee values(105, 'Juhi',34000); \\
			insert into employee values(106, 'Josh',156000);	\\
			insert into employee values(101, 'Jim',52000);	\\
			insert into employee values(104, 'Jack',89000);	\\
			insert into employee values(102, 'Anny',38000);	
		}
		
		Eg: Below query display all employees who earn salaries greater than the average salary within their own department:
		
		\outputblock{
			SELECT name	\\
			FROM employee e	\\
			WHERE salary > (	\\
			SELECT AVG(salary) FROM employee 	\\
			WHERE deptid = e.deptid	\\
			);	\\
			NAME  \s	SALARY	\\
			Ravi  \s	56000	\\
			Jim	 \s	52000	\\
			Jack \s	89000
		}
		
	\end{itemize}
\end{flushleft}

\newpage

