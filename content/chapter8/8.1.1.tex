\setlength{\columnsep}{3pt}
\begin{flushleft}
	
	Inner join returns all rows from multiple tables where the join condition is met.
	
	\newimage{1.3}{content/chapter8/images/inner.png}
	
	Consider below 2 tables:
	\codeblock{
		create table employee(id number, first\_name varchar2(20), second\_name varchar2(20)); \\
		insert into employee values(101, 'Ravi', 'Sharma');	\\
		insert into employee values(102, 'Kavi', 'Verma');	\\
		insert into employee values(103, 'Jimmy', 'Kumar');	\\
		insert into employee values(104, 'Josh', 'Kumar');	\\
		\\
		create table contractor(id number, first\_name varchar2(20), second\_name varchar2(20)); \\
		insert into contractor values(101, 'Jimmy', 'Kumar');	\\
		insert into contractor values(102, 'Jack', 'Snow');	\\
		insert into contractor values(103, 'Penny', 'Copper');	\\
		insert into contractor values(104, 'Kavi', 'Verma');
	}

		\newpage
		Eg: Inner Join
		\outputblock{
			select * from employee  \\
			inner join contractor \\
			on employee.first\_name = contractor.first\_name; \\
			\\
			ID \s	FIRST\_NAME \s	SECOND\_NAME \s ID	\s FIRST\_NAME \s	SECOND\_NAME \\
			-----------------------------------------------------------\\
			103 \s	Jimmy \s	Kumar \s	101 \s	Jimmy \s	Kumar \\
			102	\s Kavi \s	Verma \s	104 \s	Kavi \s	Verma 
		}
		
		\item INNER JOIN could be rewritten using the older implicit syntax as follows:
		\bigskip
		\outputblock{
			SELECT * \\
			FROM employee, contractor \\
			WHERE employee.first\_name = contractor.first\_name;
		}		
		
	
\end{flushleft}

\newpage



