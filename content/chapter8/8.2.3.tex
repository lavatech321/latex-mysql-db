\setlength{\columnsep}{3pt}
\begin{flushleft}

	\begin{itemize}
		\item If you know number of iterations in advance then for loop is the best choice. 
		\newimage{0.6}{content/chapter8/images/for.png}
		
		\item Curly braces are optional, without curly braces only one statement is allowed, which should not be declarative statement.
		\item All 3 parts of for loop are independent of each other and optional.	
		
		\item Egs:
		\bigskip
		\codeblock{
			for(int i = 0; true; i++) \\
			\s System.out.println("Hello");   \cmark
		}
		\bigskip
		\codeblock{
			for(int i = 0; i < 10; i++) ; \cmark
		}
		\bigskip
		\codeblock{
			for(int i = 0; i < 10; i++) \\
			\s int x = 10; \xmark
		}
		\item Let's see each section of for loop in detail:
		\begin{itemize}
			\item \textbf{Initialisation section:}
			\begin{itemize}
				\item This section will be executed only once.
				\item It declares and initialises local variables.
				\item Any valid Java statement is allowed in this section.
				\bigskip
				\codeblock{
					for(\textbf{int i = 0}; i < 10; i++) \{\} \cmark \\
					for(\textbf{int i = 0, j=0}; i < 10; i++) \{\} \cmark \\
					for(\textbf{int i = 0, String = "a"}; i < 10; i++) \{\} \xmark \\
					for(\textbf{int i = 0, int j = 0}; i < 10; i++) \{\} \xmark 
				}
				

			\end{itemize}
		
			\item \textbf{Conditional section:}
			\begin{itemize}
				\item It contains expression of the type boolean.
				\item If nothing is added here, the compiler will always place true.
				\item Eg:
				\bigskip
				\codeblock{
					for(int i = 0; \textbf{true} ; i++)
				}
				
			\end{itemize}
		
			\item \textbf{Increment/decrement section:}
			\begin{itemize}
				\item It contains any valid Java statement, mostly increment and decrement operation.
				\item Eg:
				\bigskip
				\codeblock{
					for(int i=0; i < 3; i++)  \\
					\s	System.out.print("Hi ");
				}
				\bigskip
				\outputblock{
				Hi Hi Hi
				}
			\end{itemize}
		
		\end{itemize}
		
		\item Infinite loop examples:
		\bigskip
		\codeblock{
			for(;;) \cmark  \\ 
			\s System.out.println("Hello"); \\
			or \\
			for(;;); \cmark
		}
		
		\item Unreachable statement in for loop, results in compile-time error. 
		\newline
		Eg:
		\bigskip
		\codeblock{
			for(int i = 0; \textbf{true} ; i++) \{ \\
			\s System.out.println("Hello"); \\
			\} \\
			System.out.println("Hello");  \xmark // Unreachable statement
		}
				
	\end{itemize}
	
\end{flushleft}

