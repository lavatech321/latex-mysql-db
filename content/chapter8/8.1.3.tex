\setlength{\columnsep}{3pt}
\begin{flushleft}
	
	The self ioin is joining a table by itself. In self join, you have to create the alias for the table name. 
	
	
	Consider below table:	
	\commandblock{
		CREATE TABLE Employee(	\\
		\s EmployeeID INT, FullName VARCHAR(20),	\\
		\s Gender VARCHAR(10), \\
		\s ManagerID INT	\\
		);	 \\
		INSERT INTO Employee VALUES(1, 'Pranaya', 'Male', 3);	\\
		INSERT INTO Employee VALUES(2, 'Priyanka', 'Female', 1);	\\
		INSERT INTO Employee VALUES(3, 'Preety', 'Female', NULL);	\\
		INSERT INTO Employee VALUES(4, 'Anurag', 'Male', 1);	\\
		INSERT INTO Employee VALUES(5, 'Sambit', 'Male', 1);	\\
		INSERT INTO Employee VALUES(6, 'Rajesh', 'Male', 3);	\\
		INSERT INTO Employee VALUES(7, 'Hina', 'Female', 3);
	}
	
	\textbf{Inner Self Join}
	
		\outputblock{
			SELECT E.FullName as Employee, M.FullName as Manager	\\
			FROM Employee E	\\
			INNER JOIN Employee M	\\
			ON E.ManagerId = M.EmployeeId; \\
			EMPLOYEE	\s	MANAGER	 \\
			Priyanka	\s\s	Pranaya	\\
			Anurag	\s\s	Pranaya	
		}
		\newpage
		\outputblock{
			Sambit	\s\s	Pranaya	\\
			Pranaya	\s\s	Preety	\\
			Rajesh	\s\s	Preety	\\
			Hina	\s\s\s	Preety
		}
	
	
\end{flushleft}
