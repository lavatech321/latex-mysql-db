\setlength{\columnsep}{3pt}
\begin{flushleft}
	
	\begin{itemize}
		\item If we want to execute loop body atleast once, then we should go for do-while loop.
		\bigskip
		\syntaxblock{
			do \{ \\
			\s action \\
			\} while(condition); 
		}
		\bigskip
		\item The ";" after while is compulsory.
		\item The condition should be of boolean type.
		\item Curly braces are optional
		\item Without curly braces only one statement is allowed which should not be declarative statement.
		\item Below are some valid and invlaid example of do-while:
		\bigskip
		\codeblock{
			do \\
			\s System.out.println("Hello"); \cmark \\
			while(true);
		}
		\bigskip
		\codeblock{
			do;   \cmark \\
			while(true);
		}
		\bigskip
		\codeblock{
			do \\
			\s int x = 10; \xmark \\
			while(true);
		}
		\bigskip
		\codeblock{
			do \\
			while(true); \xmark
		}
		\bigskip
		\codeblock{
			do \\
			\s while(true) \cmark \\ 
			\s \s System.out.println("Hello"); \\
			while(false);
		}
		\item Unreachable statement in do-while loop also results in compile-time error. Below are some examples showing unreachable statement:
		\bigskip
		\codeblock{
			do \{\\
			\s System.out.println("Hello"); \\
			\} while(true); \\
			System.out.println("Hi");   \xmark // Unreachable statement error
		}	
		\bigskip
		\item Final variable will be evaluated compile-time only. Thus for non-reachable condition, it will result in compile-time error.
		
		\bigskip
		\codeblock{
			final int a = 10, b = 20; \\
			do \{ \\
				\s System.out.println("Hello"); \\
			\} while(a < b); \\
			System.out.println("Hi");  \xmark 
			\s // Unreachable statement
		}
	
	\end{itemize}		
	
\end{flushleft}

\newpage

