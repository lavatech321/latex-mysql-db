\setlength{\columnsep}{3pt}
\begin{flushleft}
	
	Use break statement in the following places:
	\begin{itemize}
		\item Inside switch to stop fall-through:
		\bigskip
		\codeblock{
			int x = 0; \\
			switch(x) \{ \\
			\s case 0: \\
			\s \s System.out.print(0); \\
			\s \s break; \\
			\s case 1: \\
			\s \s System.out.print(1); \\
			\s \s break; \\
			\s case 2: \\
			\s \s System.out.print(2); \\
			\s \s break; 
			\}
		}
		
		\newpage
		\item Inside loop to break loop execution based on some condition.
		\bigskip
		\codeblock{
			for(int i=0; i < 10; i++) \{ \\
			\s if(i==5) \\
			\s \s break; \\
			\s 	System.out.print(i); \\
			\}
		}
		
		\item Inside labeled blocks to break block execution based on some condition:
		\bigskip
		\codeblock{
			int x = 10;  \\
			l1: \{ \\
			\s \s 	System.out.print("begin"); \\ 
			\s \s if(x==10) \\
			\s \s \s	break l1; \\
			\s \s 	System.out.print("end"); \\
			\} \\
			System.out.print("Hello"); 
		}
		
		
	\end{itemize}

\end{flushleft}
