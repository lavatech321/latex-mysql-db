\setlength{\columnsep}{3pt}
\begin{flushleft}
	
	Unary operator uses only ONE operand
	\syntaxblock{
		operator operand
	}
	Below are two unary operators:
	\begin{itemize}
		\item \textbf{+(Unary)} - Makes operand Positive		
		\item \textbf{-(Unary)} - Makes operand Negative
	\end{itemize}
	
	Eg: Consider below table
	\commandblock{
		CREATE TABLE sample(no number); 	\\
		INSERT INTO sample VALUES(101); 	\\
		INSERT INTO sample VALUES(102); 	\\
		INSERT INTO sample VALUES(-103); 	\\
		INSERT INTO sample VALUES(104);		\\
		INSERT INTO sample VALUES(-105);
	}
	\outputblock{
		SELECT +no FROM sample; \\
		NO \\
		101	\\
		102	\\
		-103	\\
		104	\\
		-105
	}
	\newpage
	\outputblock{
		SELECT -no FROM sample; \\
		-NO	\\
		-101	\\
		-102	\\
		103	\\
		-104	\\
		105
	}
	\outputblock{
		SELECT no,+3 FROM sample; \\
		NO \s	+3 \\
		101 \s	3 	\\
		102 \s	3	\\
		-103 \s	3	\\
		104 \s	3	\\
		-105 \s	3
	}
	\outputblock{
		SELECT no,-3 FROM sample; \\
		NO \s	-3	\\
		101 \s	-3	\\
		102	\s -3	\\
		-103 \s	-3	\\
		104	\s -3	\\
		-105 \s	-3
	}
\end{flushleft}

\newpage

