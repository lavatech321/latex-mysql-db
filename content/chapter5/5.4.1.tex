\setlength{\columnsep}{3pt}

\begin{flushleft}

		\begin{itemize}
			\item Used to terminate the while or for loop as per a condition
			\item break with \textbf{while} loop:
			\begin{tcolorbox}[breakable,notitle,boxrule=1pt,colback=pink,colframe=pink]
				\color{black}
				\fontdimen2\font=8pt
				Syntax: 
				\newline
				while condition: \newline
				\hphantom{} \hphantom{}  if condition: \newline
				\hphantom{} \hphantom{} \hphantom{} \hphantom{} break \newline
				\hphantom{} \hphantom{} statements..
				\fontdimen2\font=4pt
			\end{tcolorbox}
		
			\item \textbf{break} with \textbf{for} loop:
			\begin{tcolorbox}[breakable,notitle,boxrule=1pt,colback=pink,colframe=pink]
				\color{black}
				\fontdimen2\font=8pt
				Syntax: 
				\newline
				for x in iterable: \newline
				\hphantom{} \hphantom{}  if condition: \newline
				\hphantom{} \hphantom{} \hphantom{} \hphantom{} break \newline
				\hphantom{} \hphantom{} statements..
				\fontdimen2\font=4pt
			\end{tcolorbox}			
		
			\bigskip
			\item Below is the simple examples of break statement with \textbf{"while"} loop:

			Sample code:
			\begin{tcolorbox}[breakable,notitle,boxrule=-0pt,colback=code,colframe=code]
				\color{white}
				\fontdimen2\font=8pt
				flag=True \newline
				while flag: \newline
				\hphantom{} \hphantom{} ch=input("Do you wish to continue? (y/n): ") \newline
				\hphantom{} \hphantom{} if ch == 'n' or ch == 'no': \newline
				\hphantom{} \hphantom{} \hphantom{} \hphantom{} break
				\fontdimen2\font=4pt
			\end{tcolorbox}
			
			Output:
			\begin{tcolorbox}[breakable,notitle,boxrule=-0pt,colback=output,colframe=output]
				\color{black}
				Do you wish to continue? (y/n): y \newline
				Do you wish to continue? (y/n): y \newline
				Do you wish to continue? (y/n): n
				\fontdimen2\font=4pt
			\end{tcolorbox}
		
			\item Below is the simple examples of break statement with \textbf{"for"} loop:
			
			Sample code:
			\begin{tcolorbox}[breakable,notitle,boxrule=-0pt,colback=code,colframe=code]
				\color{white}
				\fontdimen2\font=8pt
				for x in 'apple': \newline
				\hphantom{} \hphantom{} print(x) \newline
				\hphantom{} \hphantom{} if x == 'p': \newline
				\hphantom{} \hphantom{} \hphantom{} \hphantom{} break
				\fontdimen2\font=4pt
			\end{tcolorbox}
			
			Output:
			\begin{tcolorbox}[breakable,notitle,boxrule=-0pt,colback=output,colframe=output]
				\color{black}
				a \newline
				p
				\fontdimen2\font=4pt
			\end{tcolorbox}
		
			\bigskip
			\item \textbf{break else....}
			\newline
			Sample code:
			\begin{tcolorbox}[breakable,notitle,boxrule=-0pt,colback=code,colframe=code]
				\color{white}
				\fontdimen2\font=8pt
				for x in "apple": \newline
				\hphantom{} \hphantom{} print(x)  \newline
				else: \newline
				\hphantom{} \hphantom{} print('Loop completed successfully')
				\fontdimen2\font=4pt
			\end{tcolorbox}
			
			Output:
			\begin{tcolorbox}[breakable,notitle,boxrule=-0pt,colback=output,colframe=output]
				\color{black}
				a \newline
				p \newline
				p \newline
				l \newline
				e \newline
				Loop completed successfully
				\fontdimen2\font=4pt
			\end{tcolorbox}
			
		\end{itemize}
	
		
	
\end{flushleft}


\newpage


