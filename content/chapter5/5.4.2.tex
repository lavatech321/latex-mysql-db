\setlength{\columnsep}{3pt}


\begin{flushleft}

	
	\begin{itemize}

		\item Used to skip all the remaining statements in

		the loop and move controls back to the top of the loop.

		\item \textbf{continue} with \textbf{for} loop:

		\begin{tcolorbox}[breakable,notitle,boxrule=1pt,colback=pink,colframe=pink]

			\color{black}

			\fontdimen2\font=8pt

			Syntax: 

			\newline

			for x in iterable: \newline
			\hphantom{} \hphantom{}  if condition: \newline
			\hphantom{} \hphantom{} \hphantom{} \hphantom{} continue \newline
			\hphantom{} \hphantom{} statements..

			\fontdimen2\font=4pt

		\end{tcolorbox}			

		
		\bigskip

		
		\item Below is the simple examples of continue statement with \textbf{"for"} loop:

		
		Sample code:

		\begin{tcolorbox}[breakable,notitle,boxrule=-0pt,colback=code,colframe=code]

			\color{white}

			\fontdimen2\font=8pt

			for x in "apple": \newline
			\hphantom{} \hphantom{} if x == "p": \newline
			\hphantom{} \hphantom{} \hphantom{} \hphantom{} continue \newline
			\hphantom{} \hphantom{} print(x)

			\fontdimen2\font=4pt

		\end{tcolorbox}

		
		Output:

		\begin{tcolorbox}[breakable,notitle,boxrule=-0pt,colback=output,colframe=output]

			\color{black}

			a \newline
			l \newline
			e

			\fontdimen2\font=4pt

		\end{tcolorbox}

		
	\end{itemize}

	
\end{flushleft}



\newpage



