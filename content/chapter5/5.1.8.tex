\setlength{\columnsep}{3pt}
\begin{flushleft}
	Consider below 2 tables:
	\codeblock{
		create table employee(id number, first\_name varchar2(20), second\_name varchar2(20)); \\
		insert into employee values(101, 'Ravi', 'Sharma');	\\
		insert into employee values(102, 'Kavi', 'Verma');	\\
		insert into employee values(103, 'Jimmy', 'Kumar');	\\
		insert into employee values(104, 'Josh', 'Kumar');	\\
		\\
		create table contractor(id number, first\_name varchar2(20), second\_name varchar2(20)); \\
		insert into contractor values(101, 'Jimmy', 'Kumar');	\\
		insert into contractor values(102, 'Jack', 'Snow');	\\
		insert into contractor values(103, 'Penny', 'Copper');	\\
		insert into contractor values(104, 'Kavi', 'Verma');
	}
	
	\begin{itemize}
		\item \textbf{UNION}: 
		\begin{itemize}
			\item Combine the result set of two or more queries without duplication.
			\item Sort data in ascending order based on the first column.
			\item Column names of the first query are displayed as column headings of the iutput.
			\item Do not use the query with NULL columns as the first, if so, use aliases for the NULL columns.
			\item Number of columns should match in all the queries.
			\item Use NULL columns, where sufficient number of columns are not present.
			\item Datatypes of the corresponding columns should match.
			
			\newpage
			
			\outputblock{
				SELECT id, first\_name, second\_name \\
				FROM employee \\
				UNION \\
				SELECT id, first\_name, second\_name \\
				FROM contractor; \\
				ID \s	FIRST\_NAME \s	SECOND\_NAME \\
				101 \s	Jimmy	\s	Kumar	\\
				101 \s	Ravi	\s	Sharma	\\
				102 \s	Jack	\s	Snow	\\
				102	\s	Kavi	\s	Verma	\\
				103	\s	Jimmy	\s	Kumar	\\
				103	\s	Penny	\s	Copper	\\
				104	\s	Josh	\s	Kumar	\\
				104 \s	Kavi	\s	Verma 	\\
				\\
				SELECT first\_name, second\_name	\\
				FROM employee	\\
				UNION	\\
				SELECT first\_name, second\_name	\\
				FROM contractor;	\\
				FIRST\_NAME	\s	SECOND\_NAME \\
				Jack	\s	Snow \\
				Jimmy	\s	Kumar	\\
				Josh	\s	Kumar	\\
				Kavi	\s	Verma	\\
				Penny	\s	Copper	\\
				Ravi	\s	Sharma
			}
			
		\end{itemize}
		
		\newpage
		\item \textbf{UNION ALL}: 
		\begin{itemize}
			\item If you want to include duplicate rows in the result set, use \textbf{UNION ALL} operator instead of UNION.
			\bigskip
			\outputblock{
				SELECT first\_name, second\_name	\\
				FROM employee	\\
				\textbf{UNION ALL}	\\
				SELECT first\_name, second\_name	\\
				FROM contractor;	\\
				\\
				FIRST\_NAME	SECOND\_NAME 	\\
				Ravi	\s	Sharma	\\
				Kavi	\s	Verma	\\
				Jimmy	\s	Kumar	\\
				Josh	\s	Kumar	\\
				Jimmy	\s	Kumar	\\
				Jack	\s	Snow	\\
				Penny	\s	Copper	\\
				Kavi	\s	Verma	
			}
			
		\end{itemize}
		
		\newpage
		\item \textbf{INTERSECT}: 
		\begin{itemize}
			\item Common records from two or more queries without duplication.
			\item Sorts data in ascending order based on the first column.
			\item Column names of the first query are displayed as Column headings of the output.
			\item Do not use the query with NULL columns as the first, if so, use aliases for the NULL columns.
			\item Number of columns should match in all the queries.
			\item Must use NULL columns, where sufficient number of columns are not present.
			item Datatypes of the corresponding columns should match.
			\bigskip
			\outputblock{
				SELECT first\_name, second\_name \\
				FROM employee	\\
				\textbf{INTERSECT}	\\
				SELECT first\_name, second\_name	\\
				FROM contractor; \\
				\\
				FIRST\_NAME \s	SECOND\_NAME \\
				Jimmy \s 	Kumar \\
				Kavi \s	Verma
			}
		\end{itemize}
		\newpage
		\item \textbf{MINUS}: 
		\begin{itemize}
			\item Resultant rows in the first query after eliminating the common rows of the second query.
			\item Sorts data in ascending order based on the first column.
			\item Column names of the first query are displayed as column headings of the output.
			\item Do not use the query with NULL columns as the first, if so, use aliases for the NULL columns.
			\item Number of columns should match in all the queries.
			\item Must use NULL columns, where sufficient number of columns are not present.
			\item Datatypes of the corresponding columns should match.
		\end{itemize}

		\bigskip
		\outputblock{
			SELECT first\_name, second\_name \\
			FROM employee	\\
			MINUS	\\
			SELECT first\_name, second\_name	\\
			FROM contractor;	\\
			FIRST\_NAME	\s	SECOND\_NAME	\\
			Josh	\s	Kumar	\\
			Ravi	\s	Sharma
		}
	\end{itemize}
	
\end{flushleft}
\newpage