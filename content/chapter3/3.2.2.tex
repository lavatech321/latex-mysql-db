
\begin{flushleft}
	
	Used to select the attribute based on the condition described by WHERE clause.
	
	\syntaxblock{
		SELECT column1, column2, ... \\
		FROM table\_name; \\
		or \\
		SELECT column1, column2, ... \\
		FROM table\_name \\
		WHERE [condition];
	}
	The [condition] can be any SQL expression, specified using comparison or logical operators like \textbf{>, <, =, <>, >=, <=, LIKE, NOT, IN, BETWEEN} etc.

	Eg:
	\commandblock{
		CREATE TABLE emp(id NUMBER, name VARCHAR2(20)); \\
		INSERT INTO emp VALUES(101, 'Ram'); \\
		INSERT INTO emp VALUES(102, 'Ravi'); \\
		INSERT INTO emp VALUES(103, 'Sham'); 
	}
	\outputblock{
		select id,name   \\
		2  from emp     	\\
		3  where id>102;	\\
		\\
		ID \s NAME		\\
		---------- \s -------------	\\
		103 \s Sham	
	}
	
	\textbf{Variations with select query}:
	
	\begin{itemize}
		\item \textbf{Wildcard character}: \textbf{*} is the wildcard character to select all columns in a table.
		\newline
		Eg:
		\commandblock{
			SELECT * FROM employees;
		}
		\item \textbf{Column Aliases}: 
		\begin{itemize}
			\item Make column names more readable by setting alias for column name.
			\item You can use single quotes ('), double quotes (") and square brackets ([]) to create an alias.
			\newpage
			\item Eg:
			\commandblock{
				SELECT
				FName AS "First Name", \\
				MName AS 'Middle Name', \\
				LName AS [Last Name] \\
				FROM employees;
			}
			\item If the alias has a single word you can write it without quotes or brackets.
			\item Eg:
			\commandblock{
				SELECT \\
				FName AS FirstName, \\
				LName AS LastName \\
				FROM employees;
			}
		\end{itemize}
		\item \textbf{COUNT returns total number of records}:
			\bigskip
			\syntaxblock{
				SELECT Count(*) column\_name  FROM table\_name;
			}
			Eg:
			\codeblock{
				CREATE TABLE corder ( \\
				order\_id NUMBER, \\
				customer\_id NUMBER, \\
				order\_amount NUMBER \\
				); \\
				insert into corder values(101, 201, 3400); \\
				insert into corder values(102, 202, 400); \\
				insert into corder values(103, 201, 100); \\
				insert into corder values(104, 202, 2300);  
			}
			\newpage
			\outputblock{
				SELECT COUNT(*) total  FROM corder; \\
				4
			}
		

		\item \textbf{Selecting with table alias}:
		\bigskip
		\commandblock{
			SELECT e.Fname, e.LName \\
			FROM Employees e
		}
		The Employees table is given the alias 'e' directly after the table name.
		
		\bigskip 
		\item \textbf{Selecting with more than 1 condition}
		\begin{itemize}
			\item \textbf{AND}: Makes sure all condition are true.
			\bigskip
			\commandblock{
				SELECT name FROM persons WHERE gender = 'M' AND age > 20;
			}
			\item \textbf{OR}: Makes sure at least once condition is true.
			\bigskip
			\commandblock{
				SELECT name FROM persons WHERE gender = 'M' OR age < 20;
			}
		
			\item These keywords can be combined to allow for more complex criteria combinations:
			\bigskip
			\commandblock{
				SELECT name \\
				FROM persons \\
				WHERE (gender = 'M' AND age < 20) \\
				OR (gender = 'F' AND age > 20);
			}	
		\end{itemize}
		\item \textbf{Selecting with Aggregate functions}:
		\begin{itemize}
			\item \textbf{AVG()}: Return the average of values selected.
			\bigskip
			\commandblock{
				SELECT AVG(Salary) FROM employees; \\
				SELECT AVG(Salary) FROM employees where DepartmentId = 1;
			}
			\item \textbf{MIN()}: Returns the minimum of values selected.
			\bigskip
			\commandblock{
				SELECT MIN(Salary) FROM employees;
			}
			\item \textbf{MAX()}: Returns the maximum of values selected.
			\bigskip
			\commandblock{
				SELECT MAX(Salary) FROM employees;
			}
			\item \textbf{COUNT()}: Returns the count of values selected.
			\bigskip
			\commandblock{
				SELECT COUNT(*) FROM employees; \\
				SELECT COUNT(*) FROM employees WHERE managerId IS NOT NULL \\
				SELECT COUNT(managerId) FROM employees;
			}

			Count can also be combined with the distinct keyword for a DISTINCT count.
			\bigskip
			\commandblock{
				Select COUNT(DISTINCT DepartmentId) FROM employees;
			}
			\item \textbf{SUM()}: Returns the sum of the values selected for all rows.
			\bigskip
			\commandblock{
				SELECT SUM(Salary) FROM employees;
			}
		\end{itemize}
		
			\item \textbf{Select with condition of multiple values from column}:
			\bigskip
			\commandblock{
				SELECT * FROM Cars WHERE status IN ('Waiting', 'Working') \\
				or \\
				SELECT * FROM Cars WHERE (status = 'Waiting' OR status = 'Working')
			}
			\bigskip
			\noteblock{
				value IN ( <value list> ) is a shorthand for disjunction (logical OR).
			}
			\item \textbf{Selection with sorted results}:
			\bigskip
			\commandblock{
				SELECT * FROM employees ORDER BY name; \\
				SELECT * FROM employees ORDER BY name DESC; \\
				SELECT * FROM employees ORDER BY name ASC; \\
				SELECT * FROM employees ORDER BY lame ASC, name ASC
			}

			To save retyping the column name in the ORDER BY clause, use column's number. Column numbers start from 1.
			\bigskip
			\commandblock{
				SELECT Id, FName, LName, PhoneNumber FROM employees ORDER BY 3
			}
		
			Use ORDER BY with TOP to return the top x rows based on a column's value:
			\bigskip
			\commandblock{
				SELECT TOP 5 DisplayName, Reputation FROM Users ORDER BY Reputation desc
			}
			
			\textbf{Order by Alias}: 
			\bigskip
			\commandblock{
				SELECT DisplayName, JoinDate as jd, Reputation as rep \\
				FROM Users \\
				ORDER BY jd, rep
			}
			
			\textbf{Sorting by multiple columns}:
			\bigskip
			\commandblock{
				SELECT DisplayName, JoinDate, Reputation FROM Users ORDER BY JoinDate, Reputation
			}
			
			
			\item \textbf{Selecting with null}: 
			\bigskip
			\commandblock{
				SELECT Name FROM Customers WHERE PhoneNumber IS NULL
			}
			\bigskip
			\noteblock{
				Selection with nulls take a different syntax. Don't use =, use IS NULL or IS NOT NULL instead.
			}
			\item \textbf{Select distinct (unique values only)}
			\bigskip
			\commandblock{
				SELECT DISTINCT ContinentCode
				FROM Countries;
			}
			\item \textbf{Select rows from multiple tables}:
			\bigskip
			\commandblock{
				SELECT * FROM table1, table2; \\
				SELECT table1.column1, table1.column2, table2.column1  \\
				FROM table1, table2
			}
			\bigskip
			\item \textbf{GROUP BY}
			\begin{itemize}
				\item Results of a SELECT query can be grouped by one or more columns using the GROUP BY statement. 
				\item This generates a table of partial results, instead of one result. 
				\item GROUP BY can be used with aggregation functions using the HAVING statement to define how non-grouped columns are aggregated.
				\item It might be easier if you think of GROUP BY as "for each" for the sake of explanation.

				\item Eg: Consider below table:
				\codecontinue{
					EmpID \s MonthlySalary \\
					1 \s\s 200 			 \\
					1 \s\s 300				 \\
					2 \s\s 300				 
				}
				\bigskip
				\outputblock{
					SELECT EmpID, SUM (MonthlySalary) \\
					FROM Employee \\
					GROUP BY EmpID \\
					------------------------------ \\
					1 \s \s 500 \\
					2 \s \s 300 \\
				}			
			\end{itemize}
			
		
			\item \textbf{Filter GROUP BY results using a HAVING clause}
			\begin{itemize}
				\item A HAVING clause is used in conjunction with the GROUP BY clause to filter the results of a query based on the result of an aggregate function. 
				\item Eg: Consider below table
				\bigskip
				\commandblock{
					CREATE TABLE orders ( \\
					\s order\_id NUMBER, \\
					\s customer\_id NUMBER, \\
					\s order\_amount NUMBER \\
					); \\
					INSERT INTO orders VALUES (1, 101, 500); \\
					INSERT INTO orders VALUES (2, 101, 700); \\
					INSERT INTO orders VALUES (3, 102, 300); \\
					INSERT INTO orders VALUES (4, 103, 900); \\
					INSERT INTO orders VALUES (5, 103, 600);
				}

				Retrieve only those customers who have a total order value greater than 1000:
				\commandblock{
					SELECT customer\_id, SUM(order\_amount) AS total\_order\_value \\
					FROM orders \\
					GROUP BY customer\_id \\
					HAVING SUM(order\_amount) > 1000;
				}
			
			\end{itemize}

	\end{itemize}
\end{flushleft}
\newpage