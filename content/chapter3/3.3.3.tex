
\begin{flushleft}
	
	\begin{itemize}
		\item The SET ROLE statement is used to switch between different database roles for a user's session. 
		\item By setting roles, you can control which privileges a user has during their session.
		\bigskip
		\syntaxblock{
			SET ROLE role\_name
		}
		Eg:
		\commandblock{
			SET ROLE hr\_admin
		}
		\item To disable the currently enabled role(s):
		\commandblock{
			SET ROLE NONE;
		}
		\item To enable all roles for the current user's session:
		\commandblock{
			SET ROLE ALL;
		}
		\item To enable all roles except one like hr\_admin:
		\commandblock{
			SET ROLE ALL EXCEPT hr\_admin;
		}
		\item To set role 'hr\_admin' with password:
		\commandblock{
			SET ROLE hr\_admin identified by Admin12345;
		}
		
	\end{itemize}
	
\end{flushleft}
