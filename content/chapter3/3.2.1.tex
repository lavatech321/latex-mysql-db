
\begin{flushleft}
	
	\begin{itemize}
		\item INSERT statement adds one or more rows of data to an already created table.
		\bigskip
		\syntaxblock{
			INSERT INTO table\_name[(col1, col2, col3...)] VALUES(value1, value2, value3,...);
		}		
		Eg:
		\commandblock{
			CREATE TABLE product(id NUMBER, name VARCHAR2(20)); \\
			INSERT INTO PRODUCT(id,name) VALUES(101,'cadbury'); \\
			INSERT INTO PRODUCT VALUES(102,'munch');
		}
		
		\item \textbf{'\&'} is default substitution variable wherein it will ask user to enter the value.
		\bigskip
		\syntaxblock{
			INSERT INTO table\_name VALUES(\&col1, \&col2, \&col3,...);
		}
		Eg:
		\outputblock{
			\textbf{INSERT INTO PRODUCT VALUES(\&no, '\&name');} \\
			Enter value for no: 101 \\
			Enter value for name: cadbury \\
			old 1: insert into product values(\&no, '\&name') \\
			new 1: insert into product values(101, 'cadbury') \\
			1 row created. 
		}
		\newpage
		\outputblock{
			\textbf{INSERT INTO PRODUCT VALUES(\&a, '\&b');} \\
			Enter value for no: 102 \\
			Enter value for name: munch \\
			old 1: insert into product values(\&a, '\&b') \\
			new 1: insert into product values(102, 'munch') \\
			1 row created. 
		}
		
		\item Insert record using records of another table:
		\syntaxblock{
			INSERT INTO table\_name SELECT col1,col2,.. FROM another\_table\_name; 
		}
		Eg:
		\commandblock{
			INSERT INTO product SELECT * FROM items; \\
			INSERT INTO product(id, name) SELECT id,name FROM items; 
		}	
	\end{itemize}
	
\end{flushleft}
