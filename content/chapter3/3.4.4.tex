
\begin{flushleft}
	
	\begin{itemize}
		\item Set transaction must be executed as beginning of transaction else, you will get error.
		\item Set transcation require commit or rollback to complete the transaction. After completing the transaction, the mode is set back to read write.
		\item Using \textbf{set transaction}, you can set the transaction's mode to:
		\begin{itemize}
			\item \textbf{Read-Only}:
			\bigskip
			\syntaxblock{
				SET TRANSACTION READ ONLY NAME [transaction\_name]
			}
			Using read-only, no DML command are allowed, except \textbf{SELECT}.
			Eg:
			\bigskip
			\outputblock{
				SET TRANSACTION READ ONLY NAME 'test'; \\
				INSERT INTO product VALUES(101); \\
				* \\
				ERROR at line 1:  \\
				ORA-01456: may not perform insert/delete/update operation inside a READ ONLY \\
				transaction
			}
			\bigskip
			\noteblock{
				Executing DDL commands after setting "read-only" will nullify "read-only" clause
			}
			Eg:
			\commandblock{
				SET TRANSACTION READ ONLY NAME 'test'; \\
				CREATE TABLE testtable(no NUMBER); \\
				INSERT INTO testtable VALUES(101); \\
				COMMIT;		
			}
			
			\bigskip
			\item \textbf{Read-Write}:
			\bigskip
			\syntaxblock{
				SET TRANSACTION READ WRITE
			}
			Using read-write, you can perform:
			\begin{itemize}
				\item INSERT
				\item UPDATE
				\item DELETE
				\item MERGE
			\end{itemize}
		\end{itemize}
		
		\item Eg:
		\commandblock{
			SET TRANSACTION READ WRITE NAME 'test'; \\
			CREATE TABLE newtable(no NUMBER); \\
			INSERT INTO newtable VALUES(101); \\
			COMMIT;
		}
	\end{itemize}
	
\end{flushleft}

