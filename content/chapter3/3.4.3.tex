
\begin{flushleft}
	
	\begin{itemize}
		\item Svepoint is temporary saving point within the transaction.
		\item Savepoints store within the buffer only(not in DB).
		\item Number limit for the number of savepoints within the transaction.
		\item Use unique names for the savepoints within the transaction.
		\item It is useful when you want to undo part of a transaction without rolling back the entire transaction. 
		\item Create a savepoint:
		\syntaxblock{
			SAVEPOINT savepoint\_name;
			
		}
		\item Rollback to a Savepoint:
		\bigskip
		\syntaxblock{
			ROLLBACK TO savepoint\_name;
		}
		
		Eg:
		\commandblock{
			INSERT INTO product VALUES(101,'Cadbury'); \\
			INSERT INTO product VALUES(102,'Munch'); \\
			savepoint s1; \\
			DELETE FROM product WHERE no=102; \\
			SAVEPOINT s2; \\
			INSERT INTO product VALUES(103,'5-star'); \\
			SAVEPOINT s3; \\
			ROLLBACK TO s1;
		}
		\bigskip
		\outputblock{
			SELECT * FROM product; \\
			NO NAME \\
			---------- ----------------- \\
			101 Cadbury \\
			102 Munch
		}
		\bigskip
		\noteblock{
			There is no way of seeing all savepoint names, so you need to remember the same.
		}
		\newpage
		\item Release a Savepoint: 
		\begin{itemize}
			\item If you decide that you no longer need a savepoint, you can release it using the RELEASE SAVEPOINT statement.
			\bigskip
			\syntaxblock{
				RELEASE SAVEPOINT savepoint\_name;
			}
		\end{itemize}
	\end{itemize}

\end{flushleft}

