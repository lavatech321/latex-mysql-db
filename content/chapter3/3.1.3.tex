\setlength{\columnsep}{3pt}
\begin{flushleft}
	
	The ALTER command modify existing tables, indexes, constraints, etc. 
	
	\syntaxblock{
		ALTER TABLE table\_name ADD/MODIFY/DROP/RENAME/SET UNUSED/ENABLE/DISABLE;
	}
	\begin{itemize}
		\item \textbf{Rename column}
		
		\syntaxblock{
			ALTER TABLE table\_name \textbf{RENAME} column old\_column TO new\_column;	
		}

		Eg:
		\commandblock{
			CREATE TABLE product(no NUMBER); \\
			ALTER TABLE product \textbf{RENAME} column no TO id;
		}
	
		\item \textbf{Add single column}
		\bigskip
		\syntaxblock{
			ALTER TABLE table\_name \textbf{ADD} column\_name datatype;	 \\
			or \\
			ALTER TABLE table\_name \textbf{ADD} (column\_name datatype);
		}
		Eg:
		\commandblock{
			CREATE TABLE product(no NUMBER); \\
			ALTER TABLE product \textbf{ADD} name VARCHAR2(30); \\
			or \\
			ALTER TABLE product ADD (name VARCHAR2(30));
		}
		\bigskip
		\outputblock{
			DESC product; \\
			Column \s	Null? \s	Type \\
			NO	\s  - \s 	NUMBER \\
			NAME	\s  - \s	VARCHAR2(30)
		}
		\item \textbf{Add multiple column}
		\bigskip
		\syntaxblock{
			ALTER TABLE table\_name \textbf{ADD}(column1\_name datatype,column2\_name datatype); 
		}

		Eg:
		\commandblock{
			CREATE TABLE product(no NUMBER); \\ 
			ALTER TABLE product ADD(name VARCHAR2(20), grade CHAR(3));
		}
		\bigskip
		\outputblock{
			DESC product; \\
			Column \s	Null? \s	Type \\
			NO \s	 - \s 	NUMBER \\
			NAME	\s - \s 	VARCHAR2(20) \\
			GRADE \s	 - \s 	CHAR(3)
		}
	
		\bigskip
		\item \textbf{Modify single column size}
		\syntaxblock{
			ALTER TABLE table\_name \textbf{MODIFY} column\_name DATATYPE(SIZE);  \\
			or \\
			ALTER TABLE table\_name \textbf{MODIFY}(column\_name DATATYPE(SIZE)); 
		}
		Eg:
		\commandblock{
			CREATE TABLE product(name VARCHAR2(10)); \\
			ALTER TABLE product MODIFY(name VARCHAR2(25)); 
		}
		\bigskip
		\outputblock{
			DESC product; \\
			Column \s	Null? \s	Type   \\
			NAME  \s  -  \s 	VARCHAR2(25)   
		}

		\item \textbf{Modify multiple column size}
		\syntaxblock{
			ALTER TABLE table\_name \textbf{MODIFY}(column1\_name DATATYPE(SIZE), 
			column2\_name DATATYPE(SIZE)); 
		}
		Eg:
		\commandblock{
			CREATE TABLE product(no NUMBER(2), name VARCHAR2(10)); \\
			ALTER TABLE product MODIFY(no NUMBER(3), name VARCHAR2(25)); 
		}
		\bigskip
		\outputblock{
			DESC product; \\
			Column \s 	Null? \s	Type \\
			NO	 \s - \s 	NUMBER(3,0)  \\
			NAME \s	 - \s 	VARCHAR2(25)
		}
		
		\item \textbf{Drop single column}
		\syntaxblock{
			ALTER TABLE table\_name \textbf{DROP} COLUMN column\_name;  \\
			or \\
			ALTER TABLE table\_name \textbf{DROP}(column\_name);
		}
		Eg:
		\commandblock{
			CREATE TABLE product(no NUMBER(2), name VARCHAR2(10)); \\
			ALTER TABLE product DROP COLUMN name;
		}

		\outputblock{
			DESC product; \\
			Column \s	Null? \s	Type \\
			NO	\s -  \s	NUMBER(2,0)
		}
		\item \textbf{Drop multiple column}
		\syntaxblock{
			ALTER TABLE table\_name \textbf{DROP}(column1\_name, column2\_name);
		}
		Eg:
		\commandblock{
			CREATE TABLE product(no NUMBER(2), name VARCHAR2(10), grade CHAR(2)); \\
			ALTER TABLE product DROP(name, grade);
		}
		\bigskip
		\outputblock{
			DESC product; \\
			Column \s	Null? \s	Type \\
			NO \s	 - \s 	NUMBER(2,0)
		}
		\item \textbf{Hide single column}
		\syntaxblock{
			ALTER TABLE product SET UNUSED COLUMN column\_name; \\
			or \\
			ALTER TABLE product SET UNUSED(column\_name);
		}

		Eg:
		\commandblock{
			CREATE TABLE product(no NUMBER(2), name VARCHAR2(10), grade CHAR(2)); \\
			insert into product values(11, 'cadbury','AB'); \\
			insert into product values(12, 'munch','AC'); \\
			alter table product set unused column grade; 
		}
		\bigskip
		\outputblock{
			select * from product; \\
			NO \s	NAME \\
			11 \s	cadbury \\
			12 \s	munch
		}
	
		\item \textbf{Hide multiple column}
		\syntaxblock{
			ALTER TABLE product SET UNUSED(column1\_name, column2\_name);
		}
		Eg:
		\commandblock{
			CREATE TABLE product(no NUMBER(2), name VARCHAR2(10), grade CHAR(2)); \\
			insert into product values(11, 'cadbury','AB'); \\
			insert into product values(12, 'munch','AC'); \\
			alter table product set unused(name,grade); 
		}
		\bigskip
		\outputblock{
			select * from product; \\
			NO \\
			11 \\
			12 
		}

		\quest{Why should we hide the column and not drop it directly?}{
			\begin{itemize}
				\item Dropping column with high volume of data during working hours can consume a lot of time.
				\item Hence, during working hours, we can hide the column instead of dropping it
				\item In non-working hours, we can remove the column.
			\end{itemize}
		}
		\bigskip
		\quest{After hiding a column, can we add a new column with same name?}{
			\begin{itemize}
				\item Yes, as once the column is hidden, we cannot unhide it again. We can only drop the column after hiding it.
				\item Hence we can add a colum with same name.
			\end{itemize}
		}
	
	\end{itemize}

\end{flushleft}


\newpage

