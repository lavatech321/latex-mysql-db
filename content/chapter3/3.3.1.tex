
\begin{flushleft}
	
	\begin{itemize}
		\item GRANT statement give specific privileges or permissions to users or roles, allowing them to perform operations on database objects such as tables, views, procedures, and more.
		
		\syntaxblock{
			GRANT privilege\_list ON object\_name TO user;
		}
		where,
		\begin{itemize}
			\item \textbf{privilege\_list}: Include \textbf{SELECT, INSERT, UPDATE, DELETE, EXECUTE, and ALL(which grants all available privileges)}
			
			\item \textbf{object\_name}: Name of the database object (e.g, table, view, procedure) on which you are granting privileges.
			
			\item \textbf{user}: User to which you are granting the privileges.
		\end{itemize}

		Eg:
		\commandblock{
			GRANT SELECT ON employees TO user1; \\
			GRANT ALL ON orders TO user2;  \\
			GRANT ALL ON orders TO user2, user1;
		}
		\item GRANT statement with the \textbf{"WITH GRANT OPTION"} clause means to grant a privilege to a user along with the permission to further grant that same privilege to others. 
		\newline
		Eg:
		\bigskip
		\commandblock{
			GRANT SELECT ON employees TO user1 \textbf{WITH GRANT OPTION};
		}
		\newpage
		\noteblock{
			\begin{itemize}
				\item You cannot execute grant on multiple objects in single grant statement.
				\item User can grant privileges only those privileges he/she has to other users.
				\item But not those privilege that he/she has not to other users.
				\item Person who have granted access can only revoke the permission.
			\end{itemize}
		}
		
	\end{itemize}
	
\end{flushleft}
