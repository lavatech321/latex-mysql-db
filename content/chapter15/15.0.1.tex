\setlength{\columnsep}{3pt}
\begin{flushleft}
		File class provides methods for creating, deleting, renaming, and obtaining information about files and directories. 
		\begin{itemize}
			\item Create file object in current working directory:
			\bigskip
			\syntaxblock{
				File f  = new File(String name);	
			}
			
			\item Creates a file object for a specified sub directory:
			\syntaxblock{
				File f = new File(String subdirname, String name); \\
				File f = new File(File subdir, String name);
			}
		\item \textbf{Methods}
		
		\begin{itemize}
			\item Check if file/directory is available:
			\bigskip
			\syntaxblock{
				boolean exits();
			}
		
			\item Create a new file if not present:
			\bigskip
			\syntaxblock{
				boolean createNewFile();
			}
			
			\item Create a new directory if not present:
			\bigskip
			\syntaxblock{
				boolean mkdir();
			}
			
			\item Check if file object points to given file:
			\bigskip
			\syntaxblock{
				boolean isFile();
			}
			
			\item Check if file object points to given directory:
			\bigskip
			\syntaxblock{
				boolean isDirectory();
			}
		
			\item List the names of all file/diretory present in a directory:
			\bigskip
			\syntaxblock{
				Stringp[] list();
			}
		
			\item Returns number of characters present in specified file:
			\bigskip
			\syntaxblock{
				long length();
			}
		
			\item Delete specified file or directory:
			\bigskip
			\syntaxblock{
				boolean delete();
			}
			
		\end{itemize}
	
		\end{itemize}
		
	\textbf{Example programs:}
	\begin{itemize}
		\item Create a file named "abc.txt" and a folder named "abc" in current working directory:
		
		\codeblockfull{Test.java}{
			\textbf{import java.io.File;} \\
			class Test \{ \\
			public static void main(String[] args) \textbf{throws Exception} \{ \\
			\s \s		\textbf{File f1 = new File("abc");} \\
			\s \s		\textbf{f1.mkdir();} // Create a directory \\
			\s \s		\textbf{File f = new File("abc.txt");} \\
			\s \s		\textbf{f.createNewFile();} // Create a file \\
			\s	\} \} 
		}
		\item Create a directory named "inner" in folder named "outer", assuming "outer" already exists:
		
		\codeblockfull{Test.java}{
			import java.io.File; \\
			class Test \{ \\
			public static void main(String[] args) throws Exception \{ \\
			\s \s		\textbf{File f1 = new File("outer");} \\
			\s \s		\textbf{if (f1.exists())} \{ 		 \\
			\s \s \s			\textbf{File f2 = new File(f1, "inner");} \\
			\s \s \s			\textbf{f2.mkdir();} \\
			\s \s		\} \\
			\s \s		else \\
			\s			System.out.println("Outer directory does not exist"); \\
			\s 	\} \\
			\}
		}
		\bigskip
		\item Create a file named "abc.txt" only if it exists else display message "File exits".
		\bigskip
		\codeblockfull{Test.java}{
		import java.io.File; \\
		class B \{ \\
		 public static void main(String[] args) throws Exception \{ \\
		\s \s		\textbf{File f = new File("abc.txt");} \\
		\s \s		\textbf{if (f.exists())} \\
		\s \s \s		System.out.println("File exists"); \\
		\s \s		else  \\
		\s \s \s			\textbf{f.createNewFile();} \\
		\s	\} \\
		\}
		}
	
		\item Display all files present in folder "outer/inner" folder:
		\bigskip
		\codeblockfull{Test.java}{
			import java.io.File; \\
			class Test \{ \\
			\s	public static void main(String[] args) throws Exception \{ \\
			\s \s	\textbf{File f1 = new File("outer", "inner");} \\
			\s \s	\textbf{String[] files = f1.list();} \\
			\s \s	\textbf{for (String x: files)}  \\
			\s \s \s	\textbf{System.out.println(x);} \\
			\s \} \\
			\}
		}
		\newpage
		\item Display only files present in "lavatech" folder, but not directory present under "lavatech" folder:
		\codeblockfull{Test.java}{
			import java.io.File; \\
			class Test \{ \\
			\s	public static void main(String[] args) throws Exception \{ \\
			\s \s		\textbf{File f1 = new File("lavatech");} \\
			\s \s		\textbf{String[] files = f1.list();} \\
			\s \s		\textbf{for (String x: files) \{} \\
			\s \s \s			\textbf{File f = new File(f1,x);}  \\
			\s \s \s			\textbf{if (f.isFile())} \\
			\s \s \s \s				\textbf{System.out.println(x);} \\
			\s \s 		\} \\
			\s	\} \\
			\}
		}
		
	\end{itemize}		
	
\end{flushleft}
