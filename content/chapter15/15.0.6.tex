\setlength{\columnsep}{3pt}
\begin{flushleft}	
	
	\begin{itemize}
		\item BufferedReader \textbf{reads character data line-by-line or character-to-character} from the file. 
		
		\item \textbf{Constructor}:
		\bigskip
		\syntaxblock{
			BufferedReader br = new BufferedReader(Reader r); \\
			BufferedReader br = new BufferedReader(Reader r , in buffersize);
		}
	
		\item \textbf{Method}:
		\begin{itemize}
			\item Read character in unicode format from file:
			\bigskip
			\syntaxblock{
				int read();
			}
			\item Read character in character format from file:
			\bigskip
			\syntaxblock{
				int read(char[] ch);
			}
		
			\item To read next line from the file and returns it. If the next line not available, then this method returns null:
			\bigskip
			\syntaxblock{
				String readLine();
			}
		
			\item To close the reader:
			\bigskip
			\syntaxblock{
				void close();
			}
		
		\end{itemize}
	\end{itemize}
	
	\newpage
	
	\textbf{Example Program:}
	\begin{itemize}
		\item Read all lines in file "abc.txt":
		\bigskip
		\codeblockfull{Test.java}{
			import java.io.*; \\
			class Test \{ \\
			\s	public static void main(String[] args) \\ throws Exception \{  \\
			\s \s		\textbf{FileReader fr = new  FileReader("abc.txt");} \\
			\s \s		\textbf{BufferedReader br = new BufferedReader(fr);} \\
			\s \s		\textbf{String line;} \\
			\s \s		\textbf{while( (line = br.readLine()) != null ) \{} \\
			\s \s \s			\textbf{System.out.println(line);} \\
			\s \s		\textbf{\}} \\
			\s \s		\textbf{br.close();} \\
			\s	\} \}
		}
	
	\end{itemize}
	
	
\end{flushleft}
