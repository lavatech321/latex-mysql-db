\setlength{\columnsep}{3pt}
\begin{flushleft}

\bigskip

\begin{itemize}
	\item \textbf{iostat}: Reports "Central Processing Unit" (CPU) statistics and input/output statistics for devices and partitions.
	\newline
	Options with \textbf{iostat} command:
	\begin{itemize}
		\item \textbf{-c}: Displays CPU utilization report.
		\bigskip
		\begin{tcolorbox}[breakable,notitle,boxrule=0pt,colback=pink,colframe=pink]
			\color{black}
			\fontdimen2\font=1em
			Syntax: iostat -c
			\fontdimen2\font=4pt
		\end{tcolorbox}
		Eg:
		\begin{figure}[h!]
			\centering
			\includegraphics[scale=0.2]{content/chapter15/images/iostat.png}
			\caption{Sample output}
			\label{fig:output}
		\end{figure}
	
		Output explaination:
		\begin{itemize}
			\item \textbf{\%user}: Show CPU utilization of user application.
			\item \textbf{\%nice}: Show CPU utilization of user application with nice priority.
			\item \textbf{\%system}: Show CPU utilization of the system kernel.
			\item \textbf{\%iowait}: Show CPUs idle percentage during outstanding disk I/O request.
			\item \textbf{\%steal}: Show wait by the virtual CPUs while the hypervisor was servicing another virtual processor.
			\item \textbf{\%ideal}: Show CPUs idle percentange when the	system did not have an outstanding disk I/O request.
		\end{itemize}
	\newpage
		\item \textbf{-d}: Shows device utilization report.
		\newline
		\begin{tcolorbox}[breakable,notitle,boxrule=0pt,colback=pink,colframe=pink]
			\color{black}
			\fontdimen2\font=1em
			Syntax: iostat -d
			\fontdimen2\font=4pt
		\end{tcolorbox}
		Eg:
		\begin{figure}[h!]
			\centering
			\includegraphics[scale=0.25]{content/chapter15/images/iostatd.png}
			\caption{Sample output}
			\label{fig:output2}
		\end{figure}
		\newline
		Output explaination:
		\newline
		All the values in the output are in percentage format.
		\begin{itemize}
			\item \textbf{tps}: The "tps" stands for data \textbf{Transfers Per Second} of device.
			\item \textbf{Blk\_read/s}: Data read in number of blocks (kilobytes, megabytes) per second.
			\item \textbf{Blk\_wrtn/s}: Data written in number of blocks (kilobytes, megabytes) per second.
			\item \textbf{Blk\_read}: Number of blocks (kilobytes, megabytes) read.
			\item \textbf{Blk\_wrtn}: Number of blocks (kilobytes, megabytes) written.
		\end{itemize}
		
	\end{itemize}
	\newpage
	\item \textbf{vmstat}: 
	\begin{itemize}
		\item Stands for \textbf{v}irtual \textbf{m}emory \textbf{stat}stics.
		\item It is a computer monitoring tool that collects and displays system memory, processes, interrupts, pagging and block I/O.
	\end{itemize}
	\bigskip
	Option with \textbf{vmstat} command:
	\newline
	\textbf{-s}: Displays a table of event counters and memory statistics.
	\newline
	\begin{tcolorbox}[breakable,notitle,boxrule=0pt,colback=pink,colframe=pink]
		\color{black}
		\fontdimen2\font=1em
		Syntax: vmstat -s
		\fontdimen2\font=4pt
	\end{tcolorbox}
	Eg:
	\begin{figure}[h!]
		\centering
		\includegraphics[scale=0.4]{content/chapter15/images/vmstat.png}
		\caption{Sample output}
		\label{fig:output3}
	\end{figure}

	\newpage
	\item \textbf{sar}: 
	\begin{itemize}
		\item The \textbf{sar} command is a performance monitoring uitility.
		\item It collects reports ongoing basis and saves the performance data.
		\begin{tcolorbox}[breakable,notitle,boxrule=0pt,colback=pink,colframe=pink]
			\color{black}
			\fontdimen2\font=1em
			Syntax: sar
			\fontdimen2\font=4pt
		\end{tcolorbox}
		Eg:
		\begin{figure}[h!]
			\centering
			\includegraphics[scale=0.25]{content/chapter15/images/sar.png}
			\caption{Sample output}
			\label{fig:output4}
		\end{figure}
		
		Refer \textbf{iostat} command for output explaination.	
			
	\end{itemize}
	
	
\end{itemize}

\end{flushleft}
\newpage


