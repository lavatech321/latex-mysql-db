\setlength{\columnsep}{3pt}
\begin{flushleft}

\bigskip
\paragraph{SSH server settings}
\begin{itemize}
	\item Install SSH server package.
	\bigskip
	\begin{tcolorbox}[breakable,notitle,boxrule=-0pt,colback=black,colframe=black]
		\color{green}
		\fontdimen2\font=9pt
		\# yum install openssh-server -y
		\fontdimen2\font=4pt
	\end{tcolorbox}
	\bigskip
	\item The package installation provides configuration file named \textbf{/etc/ssh/sshd\_config}.
	\bigskip
	\item Important entries in \textbf{/etc/ssh/sshd\_config} configuration file:
	\begin{itemize}
		\item \textbf{Line 17 - "Port 22"}: You can change port on this line, default SSH port is 22.
		\item \textbf{Line 43 - "PermitRootLogin yes"}: You can disable/enable root login on this line.
		\item \textbf{Line 70 - "PasswordAuthentication no"}: You can disable/enable password authentication on this line.
	\end{itemize}
	\bigskip
	\item Changes in the configuration file are reflected only after the restarting SSH daemon named \textbf{"sshd"}:
	\bigskip
	\begin{tcolorbox}[breakable,notitle,boxrule=-0pt,colback=black,colframe=black]
		\color{green}
		\fontdimen2\font=9pt
		\# systemctl restart sshd
		\fontdimen2\font=4pt
	\end{tcolorbox}
\end{itemize}

\bigskip

\paragraph{SSH client settings}
\begin{itemize}
	\item Install SSH client package.
	\bigskip
	\begin{tcolorbox}[breakable,notitle,boxrule=-0pt,colback=black,colframe=black]
		\color{green}
		\fontdimen2\font=9pt
		\# yum install openssh-client -y
		\fontdimen2\font=4pt
	\end{tcolorbox}
\end{itemize}


\end{flushleft}
\newpage


