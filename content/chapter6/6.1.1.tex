\setlength{\columnsep}{3pt}
\begin{flushleft}
	
	\begin{itemize}
		\item \textbf{ABS}: Returns the positive value of the number, whether the original number is positive or negative.
		\bigskip
		\outputblock{
			SELECT ABS(-123), ABS(123) from dual; \\
			ABS(-123) \s	ABS(123)	\\
			----------------------------- \\
			123		\s\s	123	
		}
		\item \textbf{CEIL}: Round a numeric value up to the nearest integer that is greater than or equal to the original number.
		\bigskip
		\outputblock{
			SELECT CEIL(-123.345), CEIL(123.567) from dual; \\
			CEIL(-123.345) \s	CEIL(123.567)	\\
			----------------------------- \\
			-123 \s\s	124
		}
		\newpage
		\item \textbf{POWER}: Calculate a number raised to a specified power or exponent.
		\outputblock{
			SELECT POWER(2,4) from dual; \\
			POWER(2,4)	\\
			16
		}
		\item \textbf{MOD}: 
		\outputblock{
			SELECT MOD(4,2) from dual;	\\
			MOD(4,2)	\\
			0
		}
		\item \textbf{SQRT}:
		\outputblock{
			SELECT SQRT(64) from dual; \\
			SQRT(64) \\
			8
		}
		
		\item \textbf{TRUNC}: Rounds down the number to the nearest integer less than or equal to the original number.
		\outputblock{
			SELECT TRUNC(34.89) from dual; \\
			TRUNC(34.89) \\
			34
		}
		
	\end{itemize}
	
	\textbf{GROUP/AGGREGATE function}
	\begin{itemize}
		\item \textbf{MIN}
		\outputblock{
			select min(id) from employee; \\
			MIN(ID) \\
			101
		}
		\item \textbf{MAX}
		\outputblock{
			select max(id) from employee; \\
			MAX(ID) \\
			104
		}
		\item \textbf{SUM}
		\outputblock{
			select sum(id) from employee; \\
			SUM(ID) \\
			410
		}
		\item \textbf{AVG}
		\outputblock{
			select avg(id) from employee; \\
			AVG(ID) \\
			102.5
		}
		\item \textbf{COUNT}
		\outputblock{
			select count(id) from employee; \\
			COUNT(ID) \\
			4
		}
		\item \textbf{DISTINCT}: DISTINCT keyword is used to eliminate duplicate rows from the result set.
		\outputblock{
			select distinct first\_name from employee; \\
			FIRST\_NAME	\\
			Kavi	\\
			Jimmy	\\
			Ravi	\\
			Josh
		}
		\newpage
		\item \textbf{UNIQUE}: Used to enforce that the values in a column or a set of columns must be unique across all rows in a table. 
		\outputblock{
			select unique first\_name from employee; \\
			FIRST\_NAME	\\
			Kavi	\\
			Jimmy	\\
			Ravi	\\
			Josh	
		}
		
		
	\end{itemize}
\end{flushleft}


