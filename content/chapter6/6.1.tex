\setlength{\columnsep}{3pt}
\begin{flushleft}
	
	Consider below table:
	\commandblock{
		create table employee(id number, first\_name varchar2(20), second\_name varchar2(20)); \\
		insert into employee values(101, 'Ravi', 'Sharma');	\\
		insert into employee values(102, 'Kavi', 'Verma');	\\
		insert into employee values(103, 'Jimmy', 'Kumar');	\\
		insert into employee values(104, 'Josh', 'Kumar');		
	}
	
	\textbf{Case manipulation functions}
	\begin{itemize}
		\item \textbf{UPPER(col/ char\_string)}
		\outputblock{
			SELECT UPPER(first\_name) as fname from employee; \\
			FNAME	\\
			-------------- \\
			RAVI	\\
			KAVI	\\
			JIMMY	\\
			JOSH
		}
		\item \textbf{LOWER(col/ char\_string)}
		\outputblock{
			SELECT LOWER(first\_name) as fname from employee; \\
			FNAME	\\
			--------------- \\
			ravi	\\
			kavi	\\
			jimmy	\\
			josh
		}
		\newpage
		\item \textbf{INITCAP(col/ char\_string)}:  Capitalize the first letter of each word.
		\outputblock{
			SELECT INITCAP(first\_name) as fname from employee; \\
			FNAME	\\
			-------------\\
			Ravi	\\
			Kavi	\\
			Jimmy	\\
			Josh
		}
	\end{itemize}

	\textbf{String Manipulation functions}
	\begin{itemize}
		\item \textbf{CONCAT()}
		\outputblock{
			SELECT CONCAT(first\_name, second\_name) as fname from employee; \\
			FNAME \\
			------------------------\\
			RaviSharma	\\
			KaviVerma	\\
			JimmyKumar	\\
			JoshKumar	\\
			\\
			SELECT CONCAT(CONCAT(first\_name,' '),second\_name) as fname from employee;	 \\
			FNAME	\\
			---------------------------\\
			Ravi Sharma	\\
			Kavi Verma	\\
			Jimmy Kumar	\\
			Josh Kumar 	
		}
		\newpage
		You can also use "||" operator to concatenate multiple strings. Performance-wise, operator || is more faster than concat function.
		\outputblock{
			SELECT first\_name || ' ' || second\_name as fname from employee; \\
			FNAME	\\
			Ravi Sharma	\\
			Kavi Verma	\\
			Jimmy Kumar	\\
			Josh Kumar
		}
		
		\item \textbf{LENGTH()}
		\outputblock{
			SELECT LENGTH(first\_name) as fname from employee; \\
			FNAME	\\
			4	\\
			4	\\
			5	\\
			4
		}
		\item \textbf{SUBSTR()}: Return a set of string. 
		\syntaxblock{
			SUBSTR(STRING,start,end) \\
			where, \\
			\s start and end  can be positive or negative.
		}
		Note on start:
		\begin{itemize}
			\item Start is optional.
			\item Can be 0 or 1, which means it will start from beginning.
			\item If no start is specified, then it will start from beginning.
		\end{itemize}
		Note on end:
		\begin{itemize}
			\item End is not optional.
			\item If no end is specified, the end will be end of string.
			\item End specifies the total number of characters to select.
		\end{itemize}
		\outputblock{
			SELECT SUBSTR('ORACLE',1,3) as Example from dual; \\
			EXAMPLE \\
			-----------\\
			ORA	 \\
			\\
			SELECT SUBSTR('ORACLE',0,3) as Example from dual; \\
			EXAMPLE \\
			-----------\\
			ORA	 \\
			\\
			SELECT SUBSTR('ORACLE',3,3) as Example from dual; \\
			EXAMPLE \\
			-----------\\
			ACL	 \\
			\\
			SELECT SUBSTR('ORACLE',3) as Example from dual; \\
			EXAMPLE \\
			-----------\\
			ACLE	 \\
		}
		\bigskip
		\outputblock{
			SELECT SUBSTR('ORACLE',-2) as Example from dual;	\\
			SELECT SUBSTR('ORACLE',-3) as Example from dual;	\\
			SELECT SUBSTR('ORACLE',-4) as Example from dual;
			
		}
		\newpage
		\item \textbf{INSTR()}: Used to find the position of a substring within a string. 
		\syntaxblock{
			INSTR(source\_string, search\_string, [start\_position], [occurrence])
		}
		\bigskip
		\outputblock{
			SELECT INSTR('LAVATECH TECHNOLOGY','A',1,2) as Example from dual; \\
			EXAMPLE \\
			-----------\\
			4 \\
			\\
			SELECT INSTR('LAVATECH TECHNOLOGY','O',1,2) as Example from dual; \\
			EXAMPLE	\\
			-----------\\
			17	\\
			\\
			SELECT INSTR('LAVATECH TECHNOLOGY','O',-1,2) as Example from dual; \\
			EXAMPLE	\\
			-----------\\
			15
		}

		\item \textbf{REVERSE()}
		\outputblock{
			SELECT REVERSE('ORACLE') as Example from dual; \\
			EXAMPLE	\\
			-----------\\
			ELCARO
		}
		\newpage
		\item \textbf{LPAD()}
		\outputblock{
			SELECT LPAD(first\_name, 20, ' ') as fname from employee; \\
			\s FNAME	\\
			-------------------\\
			\s \s \s Ravi	\\
			\s \s \s Kavi	\\
			\s \s \s Jimmy	\\
			\s \s \s Josh 	\\
			\\
			SELECT LPAD(first\_name, 20, '*') as fname from employee; \\
			\s \s FNAME	\\
			--------------------------\\
			****************Ravi	\\
			****************Kavi	\\
			***************Jimmy	\\
			****************Josh	\\
			\\
			SELECT LPAD(first\_name, 10) as fname from employee;	\\
			\s FNAME	\\
			--------------------------\\
			\s Ravi	\\
			\s Kavi	\\
			\s Jimmy	\\
			\s Josh	
		}
		\newpage
		\item \textbf{RPAD()}
		\outputblock{
			SELECT RPAD(first\_name, 10, '*') as fname from employee;	\\
			FNAME	\\
			--------------------------\\
			Ravi******	\\
			Kavi******	\\
			Jimmy*****	\\
			Josh******
		}
		\bigskip
		\item \textbf{LTRIM()}
		\outputblock{
			SELECT LTRIM('\%\%\%ORACLE','\%') as Example from dual; \\
			EXAMPLE \\
			---------------------\\
			ORACLE
		}
		\bigskip
		\item \textbf{RTRIM()}
		\outputblock{
			SELECT RTRIM('ORACLE\%\%\%','\%') as Example from dual; \\
			EXAMPLE \\
			---------------------\\
			ORACLE
		}
		\newpage
		\item \textbf{TRIM()}
		\outputblock{
			SELECT TRIM('\%' FROM '\%\%\%ORACLE\%\%\%') as Example from dual;\\
			EXAMPLE \\
			--------------------------\\
			ORACLE \\
			\\
			SELECT TRIM(' ORACLE ') as Example from dual; \\
			EXAMPLE \\
			--------------------------\\
			ORACLE 	\\
			\\
			SELECT TRIM(LEADING '\%' FROM '\%\%\%ORACLE\%\%\%') as Example from dual; \\
			EXAMPLE	\\
			--------------------------\\
			ORACLE\%\%\%	\\
			\\
			SELECT TRIM(TRAILING '\%' FROM '\%\%\%ORACLE\%\%\%') as Example from dual; \\
			EXAMPLE	\\
			--------------------------\\
			\%\%\%ORACLE	
		}

		\item TRANSLATE()
		\outputblock{
			SELECT TRANSLATE('ORACLE', 'OC','12') as Example from dual; \\
			EXAMPLE	\\
			1RA2LE
		}
		\newpage
		\item \textbf{REPLACE()}
		\outputblock{
			SELECT REPLACE(second\_name, 'Kumar','Kumaar') as Example from employee; \\
			EXAMPLE	\\
			--------------------------\\
			Sharma	\\
			Verma	\\
			Kumaar	\\
			Kumaar
		}
	
	\end{itemize}
	
\end{flushleft}

