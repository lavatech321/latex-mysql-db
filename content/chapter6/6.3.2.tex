\setlength{\columnsep}{3pt}
\begin{flushleft}
	\textbf{/etc/sudoers}: This file is used to create sudo users.
	\begin{itemize}
		\item This file can be edited only using \textbf{visudo} command:
		\begin{tcolorbox}[breakable,notitle,boxrule=-0pt,colback=black,colframe=black]
			\color{green}
			\fontdimen2\font=9pt
			\# visudo
			\fontdimen2\font=4pt
		\end{tcolorbox}
		\item Syntax of lines in this file:
		\bigskip
		\begin{tcolorbox}[breakable,notitle,boxrule=1pt,colback=pink,colframe=pink]
			\color{black}
			\fontdimen2\font=9pt
			users hosts=(user:group) commands
			\newline
			OR
			\newline
			users hosts=(user) commands
			\newline
			OR
			\newline
			users hosts= commands
			\newline
			OR
			\newline
			\color{blue}
			\# To avoid password prompt everytime user runs admin commands
			\newline
			\color{black}
			users hosts=(user:group) NOPASSWD: commands  
			\fontdimen2\font=4pt
			
		\end{tcolorbox}
		\item Sample entry: Root can run all commands as per below entry in \textbf{/etc/sudoers}. Do not comment out this line.
		\bigskip
		\begin{tcolorbox}[breakable,notitle,boxrule=-0pt,colback=black,colframe=black]
			\color{green}
			\fontdimen2\font=9pt
			root ALL=(ALL:ALL) ALL
			\fontdimen2\font=4pt
		\end{tcolorbox}
		This means, \textbf{root} user can run \textbf{ALL commands} on \textbf{ALL hosts} as \textbf{ALL user} and \textbf{ALL group}.
		
		
	\end{itemize}
	
	\newpage
	
	\paragraph{Creating sudo user with all command access}
	This can be done in 2 ways:
	\begin{itemize}
		\item Way I:
		\begin{itemize}
			\item Find below line in the file \textbf{/etc/sudoers} \& remove comment\textbf{(\#)}.
			\begin{tcolorbox}[breakable,notitle,boxrule=-0pt,colback=black,colframe=black]
				\color{green}
				\fontdimen2\font=9pt
				\# visudo
				\newline
				\color{white}
				\%wheel ALL=(ALL) ALL
				\fontdimen2\font=4pt
			\end{tcolorbox}
			Line explaination:
			\begin{enumerate}
				\item \%wheel: "\%" indicates group is wheel.
				\item The line means, \textbf{ALL members of wheel group} can run \textbf{ALL commands} on \textbf{ALL hosts} as \textbf{ALL user} and \textbf{ALL group}. 
			\end{enumerate}
			\item Add the user you want to give sudo access to (eg. neo) to the wheel group.
			\newline
			Eg:
			\bigskip
			\begin{tcolorbox}[breakable,notitle,boxrule=-0pt,colback=black,colframe=black]
				\color{green}
				\fontdimen2\font=9pt
				\# usermod -aG wheel neo
				\fontdimen2\font=4pt
			\end{tcolorbox}
		\end{itemize}
		\bigskip
		\item Way II:
		\newline
		To give user \textbf{bob} and \textbf{bunny} sudo access without password prompt:
		\bigskip
		\begin{tcolorbox}[breakable,notitle,boxrule=-0pt,colback=black,colframe=black]
			\color{green}
			\fontdimen2\font=9pt
			\# visudo
			\newline
			\color{white}
			bob,bunny ALL=(ALL) NOPASSWD: ALL
			\fontdimen2\font=4pt
		\end{tcolorbox}
		\end{itemize}
	
	\newpage
	
	\paragraph{Granting specific command's sudo access to user}
	\bigskip
	\begin{itemize}
		\item Give user \textbf{Peter} access to \textbf{/bin/kill, /usr/bin/kill \& /usr/bin/pkill} command:
			\begin{tcolorbox}[breakable,notitle,boxrule=-0pt,colback=black,colframe=black]
			\color{green}
			\fontdimen2\font=9pt
			\# visudo
			\newline
			\color{white}
			Peter ALL=/bin/kill, /usr/bin/kill, /usr/bin/pkill
			\fontdimen2\font=4pt
		\end{tcolorbox}
	
	\end{itemize}
	
	
	\paragraph{Using aliases in the sudoers file}
	\begin{itemize}
		\item Users can be grouped and assigned a nickname or alias which is used throughout the \textbf{/etc/sudoers} file. 
		\item Commands can also be grouped and assigned an aliases.
		\bigskip
		\begin{tcolorbox}[breakable,notitle,boxrule=-0pt,colback=black,colframe=black]
			\color{green}
			\fontdimen2\font=9pt
			\# visudo
			\newline
			\color{white}
			User\_Alias ADMINS = peter, bob
			\newline
			Cmnd\_Alias CMND = /usr/bin/useradd, /usr/bin/userdel 
			\newline
			ADMINS ALL=CMND
			\fontdimen2\font=4pt
		\end{tcolorbox}	
	\end{itemize}
	
	
	\paragraph{sudo command}
	\begin{itemize}
		\item Sudo users can run admin commands by using \textbf{sudo} command in front of every command.
		\newline
		Eg:
		\bigskip
		\begin{tcolorbox}[breakable,notitle,boxrule=-0pt,colback=black,colframe=black]
			\color{green}
			\fontdimen2\font=9pt
			\$ sudo useradd testuser
			\fontdimen2\font=4pt
		\end{tcolorbox}
	\end{itemize}
	

	
\end{flushleft}

\newpage

