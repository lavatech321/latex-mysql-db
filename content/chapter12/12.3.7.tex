\setlength{\columnsep}{3pt}
\begin{flushleft}
	
	\begin{itemize}
		\item A marker interface is an interface that does not declare any methods. 
		\item Its sole purpose is to mark or tag a class, indicating that the class has a certain property or behavior.
		
		\item A marker interface is essentially an empty interface, but by implementing that interface, a class can convey some additional information to the compiler.
		
		\item Here's an example of a marker interface named Serializable in Java:
		\bigskip
		\codeblock{
			public interface Serializable \{ \\
			\s	// Empty interface \\
			\}
		}
		\item Below are some markers interfaces, these are marked for some ability:
		\begin{itemize}
			\item Serialisable (I)
			\item Cloneable (I)
			\item RandomAccess (I)
		\end{itemize}
	\end{itemize}
	\quest{Without having any methods, how the objects will get some ability in marker interfaces?}{
		Internally JVM is responsible to provide required ability.	
	}
	
	\quest{Why JVM is providing required ability in marker interfaces?}{
		To reduce complexity of programming \& to make java language as simple.
	}
	
	\quest{Is it possible to create our own marker interface?}{
			Yes, but customisation of JVM  is required. For this we will have to design our own JVM.
	}
	
\end{flushleft}

