\setlength{\columnsep}{3pt}
\begin{flushleft}
	
	\begin{itemize}
		\item Every class in Java extends \textbf{Object} class.
		\item Object class is the mother of all classes.
	\end{itemize}	
	\textbf{So what’s in this Object class?}
	\newimage{0.9}{content/chapter12/images/img5.png}

 	\textbf{equals(Object o)}: Used to check if 2 objects are considered "equal".
			\codeblockfull{Test.java}{
			class Dog \{\} \\
			class Cat \{\} \\
			class Test \{ \\
			\s public static void main(String[] args) \{ \\
			\s \s		Dog a = new Dog(); \\
			\s \s		Cat b = new Cat(); \\
			\s \s		if (a.equals(b)) \\
			\s \s \s			System.out.println("true"); \\
			\s \s		 else \\
			\s \s			System.out.println("false");  // Output: false \\
			\s	\} \}
			}
	\textbf{getClass()}: Used to display the class of the object.
			\codeblockfull{Test.java}{
			class Cat \{\} \\
			class Test \{ \\
			\s	public static void main(String[] args) \{ \\
			\s \s		Cat a = new Cat(); \\
			\s 	System.out.println(a.getClass()); // Output: class Cat \\
			\s	\} \}
			}
	
			\textbf{hashCode()}: Prints out a hashcode for the object (think of it as a unique code).
			\codeblockfull{Test.java}{
				class Cat \{\} \\
				class Test \{ \\
				\s	public static void main(String[] args) \{ \\
					\s \s		Cat a = new Cat(); \\
					\s 	System.out.println(a.hashCode()); // Output: 117845 \\
					\s	\} \\
					\}	
			}
			\textbf{toString()}: Prints out a string message with the name of the class and random number.
			\codeblockfull{Test.java}{
				class Cat \{\} \\
				class Test \{ \\
				\s	public static void main(String[] args) \{ \\
				\s \s		Cat a = new Cat(); \\
				\s 	System.out.println(a.toString()); // Output: Cat@515f550a \\
				\s	\} \\
				\}
			}

	
	
\end{flushleft}

\newpage

