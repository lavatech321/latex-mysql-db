\setlength{\columnsep}{3pt}
\begin{flushleft}
	
	\paragraph{}
	\bigskip
	
	\begin{figure}[h!]
		\centering
		\includegraphics[scale=.2]{content/practise.jpg}
	\end{figure}	
	\begin{enumerate}
		
		\item \textbf{Which of the following is true about CPU? (Select all that applies.)}
		\begin{enumerate}[label=(\alph*)]
			\item CPU stands for Central Processing Unit.  %correct
			\item CPU is also known as processor. %correct
			\item CPU perform arithematic \& logical operations of computer. %correct
			\item CPU has CPU cores \& threads. %correct
		\end{enumerate}
		\bigskip
		\bigskip
		
		\item \textbf{How many CPU cores are present in Hexa core processor? }
		\begin{enumerate}[label=(\alph*)]
			\item 6 cores  %correct
			\item 4 cores
			\item 2 cores
			\item 8 cores
		\end{enumerate}
		\bigskip
		\bigskip	
		
		\item \textbf{Which of the following formula calculates total number of logical CPU?}
		\begin{enumerate}[label=(\alph*)]
			\item No. of cores X No. of threads
			\item No. of socket X No. of threads
			\item No. of socket X No. of cores
			\item No. of socket X No. of cores X No. of threads  %correct
		\end{enumerate}
		\bigskip
		\bigskip	


		\item \textbf{Which of the following command is used to display CPU details?}
		\begin{enumerate}[label=(\alph*)]
			\item free -w 
			\item lscpu    %correct
			\item fdisk -l
			\item top
		\end{enumerate}
		\bigskip
		\bigskip	
		
		
		\item \textbf{Which of the following file stores CPU details?}
		\begin{enumerate}[label=(\alph*)]
			\item /proc/meminfo
			\item /proc/cpu
			\item /proc/cpuinfo   %correct
			\item /proc/processor
		\end{enumerate}
		\bigskip
		\bigskip	
		
	\end{enumerate}
	
	
\end{flushleft}

\newpage

