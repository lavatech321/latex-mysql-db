
\begin{flushleft}
	
	\begin{itemize}
		\item The StringBuffer class is a \textbf{mutable} sequence of characters present in \textbf{java.lang} package. 
		\item It modifies string content without creating a new object. 
		
		\item \textbf{capacity}: Maximum number of characters that can hold without resizing.
		\begin{itemize}
			\item Capacity of empty StringBuffer object:
			\begin{itemize}
				\item Default capacity: 16 chars.
				\item Once 16th character is added, new capacity will be:
				\bigskip
				\syntaxblock{
					\textbf{New capacity = (current capacity + 1) * 2}}
				\item Eg:
				\bigskip
				\codeblock{
					StringBuffer s1 = new StringBuffer(); \\
					System.out.println(\textbf{s1.capacity()}); // Output: 16 \\
					s1.append("hello"); \\
					System.out.println(s1.capacity()); // Output: 16 \\
					s1.append("world by Java and testing");  \\
					System.out.println(s1.capacity()); // Output: 34
				}
				
			\end{itemize}
			
			
			\item Capacity of non-empty StringBuffer object:
			\bigskip
			\syntaxblock{
				\textbf{capacity = current length of characters + 16}
			}
			\bigskip
			\item Eg:
			\bigskip
			\codeblock{
			StringBuffer s2 = new StringBuffer("Lavatech Technology"); \\
			System.out.println(s2.capacity()); // Output: 35
			}
			
		\end{itemize}
	\end{itemize}
	
\end{flushleft}

