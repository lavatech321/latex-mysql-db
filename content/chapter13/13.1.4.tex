\setlength{\columnsep}{3pt}
\begin{flushleft}
	
	\begin{itemize}
		\item The string constant pool (SCP) (or string pool), is a \textbf{special area of memory in the Java heap} that stores string objects. 
		\item Using SCP, Java compiler checks if an new string is already exists in the pool. 
		\item If not, a new string object is created and added to SCP.
		\item If it does, the reference to the existing string is returned. 
		\item Eg:
		\bigskip
		\codeblock{
		String str1 = "lava"; // Added to the string constant pool \\
		String str2 = "lava"; // Reuses the existing string from the pool \\
		\\
		System.out.println(str1 == str2); // Output: true (both strings refer to the same object)
		}
	
		\newimage{0.3}{content/chapter13/images/heap.png}
		
		\newpage
		
		\item Use intern() method to explicitly add string to the string constant pool:
		\item Eg:
		\bigskip
		\codeblock{
			String str3 = new String("Hello").intern(); // Explicitly interns the string  \\
			String str4 = "Hello"; \\
			\\
			System.out.println(str3 == str4); // Output: true
		}
	
		\bigskip
		In this case, str3.intern() adds the string to the string constant pool, allowing str3 and str4 to refer to the same string object.
	
		\bigskip
		\noteblock{
		Upto Java1.6, scp is in method area. After java1.7, scp is present in heap itself.
		}	
	
	\end{itemize}
	
\end{flushleft}
\newpage
