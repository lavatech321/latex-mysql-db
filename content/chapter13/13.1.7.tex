
\begin{flushleft}
	\begin{tabulary}{1.0\textwidth}{|p{5em}|p{22em}|}
		\toprule
		\textbf{Function} & \textbf{Syntax \& Example} \\
		\midrule
		charAt() &  public char charAt(int index) \newline Eg:
		\codeblock{
			String s1 = "hello world"; \\
			System.out.println(s1.charAt(3));  // Output: l
		}   \\
		\hline
		
		concat() & public String concat(String ) \newline Eg:
		\codeblock{
			String s1 = "Lava"; \\
			s1 = s1.concat("tech"); \\
			System.out.println(s1); // Output: Lavatech \\
			s1 = s1 + " Techo"; \\
			System.out.println(s1); // Output: Lavatech Techo \\
			s1 += "logy"; \\
			System.out.println(s1); // Output: Lavatech Technology
		} \\
		\hline
		equals() & public boolean equals(Object o) \newline Eg:
		\codeblock{
			String s1 = "Lava"; \\
 			System.out.println(s1.equals("lava")); // Output: false
		} \\
	
		\bottomrule
	\end{tabulary}
	
	\newpage

		\begin{tabulary}{1.0\textwidth}{|p{9em}|p{18em}|}
		\toprule
		\textbf{Function} & \textbf{Syntax \& Example} \\
		\midrule
				toCharArray() & public char[] toCharArray()
				 \newline Eg:
		\codeblock{
			String result = "Lava"; \\
			char[] ch2 = result.toCharArray(); \\
			System.out.println(ch2);  // Output: Lava
		} \\
		\hline
		isempty(): & public boolean isEmpty() \newline Eg:
		\codeblock{
			String s1 = "Lava"; \\
			System.out.println(s1.isEmpty()); // false
		} \\
		\hline
		length() & public int length() \newline Eg:
		\codeblock{
			String s1 = "Lava"; \\ 
			System.out.println(s1.length()); // Output: 4
		} \\
		\hline
		replace() & public String replace(char old, char new) \newline Eg:
		\codeblock{
			String s1 = "Lava"; \\
			System.out.println(s1.replace('a', 'A')); // Output: LAvA
		} \\
	
		\bottomrule
	\end{tabulary}
			
	
	\newpage
	
	\begin{tabulary}{1.0\textwidth}{|p{5em}|p{22em}|}
		\toprule
		\textbf{Function} & \textbf{Syntax \& Example} \\
		\midrule
		substring() & public String substring(int begin) \newline Eg:
		\codeblock{
			String s1 = "Lava";
			System.out.println(s1.substring(1)); // Output: ava
		} 
		 public String substring(int begin, int end) \newline 
		 \codeblock{
		 	String s1 = "Lava"; \\
		 	System.out.println(s1.substring(0,2)); // Output: La
		 } \\
		\hline
		indexof() & public int indexof(char ch);
		 \newline Eg:
		\codeblock{
			String s1 = "Lava"; \\
			System.out.println(s1.indexOf("L")); // Output: 0
			System.out.println(s1.indexOf("z")); // Output: -1
		} 
		It always will return first occurrence index only. Returns -1 if charachter not found. \\
		\midrule
		lastIndexOf() & public int lastIndexOf(char ch) \newline Eg:
		\codeblock{
			String s1 = "Lava";
			System.out.println(s1.lastIndexOf("L")); // Output: ava
		}  \\
		
		\bottomrule
	\end{tabulary}

	\newpage
	
	\begin{tabulary}{1.0\textwidth}{|p{7em}|p{20em}|}
		\toprule
		\textbf{Function} & \textbf{Syntax \& Example} \\
		\midrule
		toUpperCase() & public String toUpperCase() \newline Eg:
		\codeblock{
			String s1 = "Lava"; \\
			System.out.println(s1.toUpperCase()); // Output: LAVA
		}  \\
		\midrule
		toLowerCase() & public String toLowerCase() \newline Eg:
		\codeblock{
			String s1 = "Lava"; \\
			System.out.println(s1.toLowerCase()); // Output: lava
		}  \\
		\midrule
		trim() & public String trim() \newline 
		Used to remove space from beginning and end of a string \newline
		Eg:
		\codeblock{
			String s1 = "   Lava  "; \\
			System.out.println(s1.trim()); // Output: Lava
		}  \\		
		\bottomrule
	\end{tabulary}



\end{flushleft}

\newpage

