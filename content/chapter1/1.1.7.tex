\begin{flushleft}
	
	\newimage{0.64}{content/chapter1/images/2.png}
	
	\textbf{OLTP}
	\begin{itemize}
		\item Stands for \textbf{O}n\textbf{L}ine \textbf{T}ransaction \textbf{P}rocessing.
		\item It is database system to manage and process day-to-day data transactions.
		\item It involve activities characterized by:
		\begin{itemize}
			\item Process transactions quickly.
			\item It involve transaction for insert, update, or delete small amounts of data in database.
			\item Break data into smaller, related tables to avoid data duplication.
			\item Users expect quick responses for applications like e-commerce, banking systems etc.
		\end{itemize}
	\end{itemize}
	
	\bigskip
	\textbf{Data warehouse}
	\begin{itemize}
		\item A data warehouse is a centralized repository for storing large volumes of data from various sources. 
		\item Data once entered into data warehouse cannot be modified.
		\item Data in data warehouse is:
		\begin{itemize}
			\item Consolidated
			\item Historical Data
			\item Non-volatile
		\end{itemize}
		\bigskip
	\end{itemize}

	\textbf{OLAP}
	\begin{itemize}
		\item Stands for \textbf{O}n\textbf{L}ine \textbf{A}nalytical \textbf{P}rocessing
		\item OLAP system are used for:
		\begin{itemize}
			\item Data analysis
			\item Reporting
			\item Facilitate decision-making 
			\item Business intelligence activities
		\end{itemize}
	\end{itemize}

	\textbf{ETL}
	\begin{itemize}
		\item ETL stands for Extract, Transform, Load.
		\item It is iterative process that runs on a scheduled basis to keep the data warehouse updated. 
		\item Data warehousing performs below task:
		\begin{itemize}
			\item Collect data from various sources
			\item Transform it into a suitable format
			\item Load it into data warehouse
		\end{itemize}
	\end{itemize}
	
\end{flushleft}

\newpage