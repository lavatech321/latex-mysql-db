\setlength{\columnsep}{5pt}

\begin{flushleft}

\begin{itemize}
	\item Code structure in Java:
	\newimage{0.35}{content/chapter0/images/new06.png}
	
	\quest{What goes in a source file?}{
		\begin{itemize}
			\item A source code file with the \textbf{“.java”} extension holds one class definition. 
			\item The source file name should be \textbf{"classname.java"}.
			\item The class represents a piece of your program. 
			\item The class must go within a pair of curly braces.
			\item Eg: Below code name will be Dog.java and class name will be Dog - \\
			\boximage{0.3}{content/chapter0/images/new07.png}
		\end{itemize}
	}
	\bigskip
	\quest{What goes in a class?}{
		\begin{itemize}
			\item A class has one or more methods. 
			\item Your methods must be declared inside a class.
			\item Eg: The class Dog contains a method called bark()
			\\
			\boximage{0.3}{content/chapter0/images/new08.png}
		\end{itemize}	
	}
	\bigskip
	\quest{What goes in a method?}{
		\begin{itemize}
			\item Method code is basically a set of statements.
			\item Method kind of like a function or procedure.
			\item When the JVM starts running, it looks for the method inside the class that looks exactly like:
			\codeblock{
				public static void main (String[] args) \{ \\
				\s 	// your code goes here \\
				\}
			}
			\item JVM runs everything inside \{ \} of your main method. 
			\item Every Java application has to have at least one class, and at least one main method (not one main per class; just one main per application).
		\end{itemize}
	}

	\textbf{Overall, a basic Java program would look something like below:}
	\newimage{0.35}{content/chapter0/images/new09.png}
	
	\bigskip
	\textbf{Running your Java Code:}
	\newimage{0.3}{content/chapter0/images/new10.png}
	
\end{itemize}

				
\end{flushleft}

