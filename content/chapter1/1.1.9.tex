\begin{flushleft}
	
	\newimage{0.4}{content/chapter1/images/4.png}
	
	Create a local user in Oracle 21c.
	\begin{itemize}
		\item Login as system DBA:
		
		\syntaxblock{
			sqlplus sys/<password>@<TNS service name> as sysdba
		}
		where,
		\begin{itemize}
			\item <password> is the password that you set during oracle installation.
			\item <TNS service name> where Transparent Network Substrate (TNS) operates for connection to Oracle databases.
		\end{itemize}
		\noteblock{
			You can find the current database name using below query:
			\commandblock{
				SELECT ora\_database\_name FROM dual;
			}	
		}
		Eg:
		\commandblock{
		>sqlplus sys/Admin12345@orcl as sysdba
		}
		
		Optionally, you can connect using command prompt with username/password:
		\commandblock{
		> sqlplus / as sysdba
		}
		
		
		\item Once connected to database, alter the session:	
		\commandblock{
		SQL> alter session set "\_oracle\_script"=true;
		}
	
		\item Create a local user using below syntax:
		\syntaxblock{
			SQL> create user <username> identified by <password>; \\
			SQL> grant resource,connect,dba to <username>;
		}
		\item Eg:
		\commandblock{
			SQL> create user jack identified by Admin12345; \\
			SQL> grant resource,connect,dba to jack;
		}
		
		\item Login as local user:
		\syntaxblock{
			sqlplus user/password	
		}
		Eg:
		\commandblock{
			C:$\backslash$Users$\backslash$Admin> \textbf{sqlplus jack/Admin12345}
		}
		
		\newpage
		\item You can check current user using below command:
		\commandblock{
		SQL> \textbf{show user;} \\
		USER is "Jack"
		}
	
		\item You can switch user using below command:
		\commandblock{
			SQL> \textbf{conn jack} \\
			Enter password: \\
			Connected.
		}
	
		\item You can check all user details in "ALL\_USER" table:
		\commandblock{
			SQL> \textbf{select username from ALL\_USER;}
		}
		
		\item You can clear the screen using below command:
		\commandblock{
			SQL> \textbf{cl scr;}
		}
	
		\item You can drop user account using below command:
		\bigskip
		\syntaxblock{
			drop user username; \\
			or \\
			drop user username CASCADE;  
		}
		Eg:
		\commandblock{
			SQL> drop user jack;
		}
	\end{itemize}
	
	\newpage
	
	\textbf{Connecting to Oracle SQL developer}
	
	\begin{itemize}
		\item Start the Oracle SQL developer
		\item Create a new connection by clicking the "+" symbol as shown in screenshot below:

		\newimage{0.4}{content/chapter1/images/image1.png}
		
		\item Enter the connection details as shown and click "Test". 
		\newimage{0.4}{content/chapter1/images/image2.png}
		
		Once the test is successful, select "Connect".
		
		\item Enlarge the "NewConnection" dropdown and notice all the tables are displayed:
		\newpage
		\newimage{0.4}{content/chapter1/images/image3.png}
		
		\item Create a new table by executing below query in worksheet. Click the green run logo on top of worksheet to execute the query.

		\commandblock{
			create table employee(no int, name varchar2(20));
		}
		
		\item Refresh the table to see new table entry:
		\newimage{0.4}{content/chapter1/images/image4.png}
		
		\item Optionally you can execute below query to display all tables:
		\commandblock{
			select * from tab;
		}
		
		
		
	\end{itemize}
	
\end{flushleft}

\newpage