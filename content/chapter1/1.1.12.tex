\begin{flushleft}
	
	\begin{itemize}
		\item An ERD is a visual representation of the data model depicting entities and their relationship with each other.
		
		\item ERD components:
		
		\begin{itemize}
			\item \textbf{Entities}: 
			\begin{itemize}
				\item Entities represents real world object having some data.
				\item Eg: Customer, Employee
			\end{itemize}
			
			
			\item \textbf{Attributes}: 
			\begin{itemize}
				\item Attributes are characteristics of entities.
				\item Eg: name, age, phone-number
			\end{itemize}
			
			
			\item \textbf{Relationships}: 
			\begin{itemize}
				\item Relationships define how entities are connected to each other.
				\item Eg: Child and Parent have filiation relationship
			\end{itemize}
			
			\newpage
			\newimage{0.45}{content/chapter1/images/eg1.png}
			
			
			\item \textbf{Cardinality}: 
			\begin{itemize}
				\item Cardinality specify how many instances of one entity are associated with instances of another entity in a relationship. 
				\item 3 types of cardinalities:
				\newimage{0.55}{content/chapter1/images/test100.png}
			\end{itemize}
			
			\newpage
			\item \textbf{Primary Key}: 
			\begin{itemize}
				\item A primary key is an attribute that uniquely identifies an entity. 
				\item It ensures that there are no duplicate records.
			\end{itemize}
			
			
			\item \textbf{Foreign Key}: 
			\begin{itemize}
				\item A foreign key is an attribute in one entity that refers to the primary key of another entity. 
				\item It establishes a link between the two entities.
			\end{itemize}
			
			\bigskip
			\newimage{0.75}{content/chapter1/images/test200.png}
			
			\newpage
			\item \textbf{Diagram Symbols}: Peter Chen invented Chen ERD notation. Symbols used to represent ERD are:
			\bigskip
			\begin{itemize}
				\item \textbf{Entity:}
				\newimage{0.4}{content/chapter1/images/1.png}
				\item \textbf{Attribute:}
				\newimage{0.4}{content/chapter1/images/5.png}
				\item \textbf{Multi-valued attribute:} An attribute can have more than one value. 
				\newimage{0.4}{content/chapter1/images/8.png}
				\newline
				Eg: A student can have more than one phone number.
				\newimage{0.5}{content/chapter1/images/10.png}
				\item \textbf{Derived Attribute:} An attribute that can be derived from other attribute is known as a derived attribute. 
				\newimage{0.6}{content/chapter1/images/11.png}
				\newline
				Eg:  A person's age can be derived from Date of birth.
				\newimage{0.5}{content/chapter1/images/12.png}
				\newpage
				\item \textbf{Primary Key attribute:}
				\newimage{0.4}{content/chapter1/images/6.png}
				\newline
				Eg: Id is unqiue for every student
				\newimage{0.55}{content/chapter1/images/13.png}
				\newpage
				\item \textbf{Relationship:}
				\newimage{0.4}{content/chapter1/images/7.png}			
				\item \textbf{Cardinality:}
				\newimage{0.4}{content/chapter1/images/new.png}
			\end{itemize}
		\end{itemize}
	\end{itemize}
	
	
		
	
	
\end{flushleft}

\newpage