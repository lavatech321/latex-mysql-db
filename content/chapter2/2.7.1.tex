
\begin{flushleft}
	\begin{itemize}
		\item \textbf{Ordered collection of bytes}, enclosed in "bytearray()".
		\begin{tcolorbox}[breakable,notitle,boxrule=1pt,colback=pink,colframe=pink]
			\color{black}
			\fontdimen2\font=8pt
			Syntax: a=bytearray('characters','ascii')
			\newline
			Syntax: a=bytearray('characters','utf-8')
			\fontdimen2\font=4pt
		\end{tcolorbox}
		
		Eg:	
		\begin{tcolorbox}[breakable,notitle,boxrule=-0pt,colback=code,colframe=code]
			\color{white}
			\fontdimen2\font=8pt
			a=bytearray('apple','utf-8') \newline
			print(a,type(a))
			\fontdimen2\font=4pt
		\end{tcolorbox}
		
		Output:
		\begin{tcolorbox}[breakable,notitle,boxrule=-0pt,colback=output,colframe=output]
			\color{black}
			bytearray(b'apple') <class 'bytearray'>
			\fontdimen2\font=4pt
		\end{tcolorbox}
		
		\item Bytearray datatypes stores ascii and unicode characters.
		\newline
		Eg:			
		\begin{tcolorbox}[breakable,notitle,boxrule=-0pt,colback=code,colframe=code]
			\color{white}
			\fontdimen2\font=8pt
			a=bytearray('apple','utf-8') \newline
			print(a,a[0],a[1])
			\fontdimen2\font=4pt
		\end{tcolorbox}

		Output:
		\begin{tcolorbox}[breakable,notitle,boxrule=-0pt,colback=output,colframe=output]
			\color{black}
			bytearray(b'apple') 97 112
			\fontdimen2\font=4pt
		\end{tcolorbox}
		
		\item Bytearray is mutable datatype. You can add, remove or change existing bytes characters.
		\newline
		Eg:			
		\begin{tcolorbox}[breakable,notitle,boxrule=-0pt,colback=code,colframe=code]
			\color{white}
			\fontdimen2\font=8pt
			a=bytearray('apple','utf-8') \newline
			a[0]=99 \newline
			print(a)
			\fontdimen2\font=4pt
		\end{tcolorbox}
		
		Output:
		\begin{tcolorbox}[breakable,notitle,boxrule=-0pt,colback=output,colframe=output]
			\color{black}
			bytearray(b'cpple')
			\fontdimen2\font=4pt
		\end{tcolorbox}
		
	\end{itemize}
	
\end{flushleft}

\newpage

