
\begin{flushleft}
		
		\begin{tabulary}{1.0\textwidth}{|p{7em}|p{10em}|p{7em}|}
			\toprule
			\textbf{Operation} & \textbf{Sample code}  & \textbf{Output} \\
			\midrule
			Indexing & a=["raj",[4.5, 8.9]] \newline print(a[0]) \newline print(a[-1])  \newline print(a[-1][-1]) & raj \newline [4.5, 8.9] \newline 8.9 \\
			\hline
			Slice Operator & a=["raj","ravi","raman"] \newline print(a[:2]) \newline print(a[-2:]) & ['raj', 'ravi'] \newline ['ravi', 'raman']  \\
			\hline
			Change Member & a=["raj","ravi","raman"] \newline a[-1]="jack" \newline
			print(a) & ["jack", "ravi", "raman"]  \\
			\hline
			Change Multiple Member & a=["raj","ravi","raman"] \newline a[0:2]=["apple","mango"]  \newline print(a) & ['apple', 'mango', 'raman']  \\
			\hline
			"+" Operator & a=["jack"] \newline b=["jill"] \newline print(a) & ["jack","jill"] \\
			\hline
			"*" Operator & a=["jack"] \newline print(a*2) & ["jack","jack"] \\
			\hline
			Delete a Member & a=["raj","ravi","raman"] \newline del a[0] \newline print(a) & ["ravi","raman"] \\
			\hline
			Delete Multiple Member & a=["raj","ravi","raman"] \newline del a[:2] \newline
			print(a) \newline a[:2]=[] \newline print(a) & ["raman"] \newline ["raman"] \\
			\hline
			Membership operator & a=["raj","ravi","raman"] \newline 
			print("raj" in a) & True  \\
			\bottomrule
		\end{tabulary}

	
\end{flushleft}

