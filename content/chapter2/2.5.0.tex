

\begin{flushleft}

	\begin{itemize}

		\item \textbf{Unordered collection of datatypes}, enclosed in "{}", wherein duplicate members are not allowed.

		\begin{tcolorbox}[breakable,notitle,boxrule=1pt,colback=pink,colframe=pink]

			\color{black}

			\fontdimen2\font=8pt

			Syntax: \{"string",int,float,complex,....\}

			\fontdimen2\font=4pt

		\end{tcolorbox}

		
		Eg:	

		\begin{tcolorbox}[breakable,notitle,boxrule=-0pt,colback=code,colframe=code]

			\color{white}

			\fontdimen2\font=8pt

			a=\{"ravi@gmail.com",34,4.5,True,34,True\}

			\newline

			print(a,type(a))

			\fontdimen2\font=4pt

		\end{tcolorbox}

		
		Output:

		\begin{tcolorbox}[breakable,notitle,boxrule=-0pt,colback=output,colframe=output]

			\color{black}

			\{True, 34, 'ravi@gmail.com', 4.5\} <class 'set'>

			\fontdimen2\font=4pt

		\end{tcolorbox}

		
		\bigskip

		
		\item Set is mutable datatype. You can add, remove or change existing set members.

		\newline
		Eg:
		\begin{tcolorbox}[breakable,notitle,boxrule=-0pt,colback=code,colframe=code]

			\color{white}

			\fontdimen2\font=8pt

			s1=\{"Ram","Raman","Ravi","Ram"\} \newline

			s1.add("Sham") \newline

			print(s1)

			\fontdimen2\font=4pt

		\end{tcolorbox}

		
		Output:

		\begin{tcolorbox}[breakable,notitle,boxrule=-0pt,colback=output,colframe=output]

			\color{black}

			{'Raman', 'Ram', 'Sham', 'Ravi'}			

			\fontdimen2\font=4pt

		\end{tcolorbox}

		
		\bigskip

		\item Being mutable datatype, memory address of similar set datatype is same.

		
	\end{itemize}

	
\end{flushleft}


\newpage


