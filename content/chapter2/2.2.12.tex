
\begin{flushleft}

	\begin{itemize}
		
		\item By default, floating-point numbers are represented in double form.
		\item So below declaration will result in error: 
		\bigskip		
		\codeblock{
				float f = 1.0 \xmark
		}
		\bigskip		
		
		\item Correct way to represent float data type is by suffixing \textbf{"F"} or \textbf{"f"} to the floating-point number as shown:
		\bigskip
		\codeblock{
			float f = 1.6F; \cmark \newline
			float f = 7.8f;		\cmark
		}

		\item Double data type can be represented using suffix \textbf{"D"} or \textbf{"d"} or no suffix as below:
		\bigskip
		\codeblock{
		    double a = 12.67; \cmark \newline
			double b = 13.7d; \cmark \newline
			double c = 123.456D; \cmark 
		}
	
		\item Assign integral literal directly to floating-point variables:
		\bigskip
		\codeblock{
			double a = 0456; \cmark \newline
			float b = 0XFace; \cmark 
		}
			
		\item \textbf{Expontential format:} This is scientific notation to represent very large or small floating-point values. Use the letter “e” or “E” to indicate the exponent:
				\bigskip
		\codeblock{
			double a = 1.2e3; \cmark \newline
			float b = 1.3e4F; \cmark
		}
		
		\bigskip
		\item \textbf{Usage of \_ in floating literal}:
		\begin{itemize}
			\item From Java 1.7 version, "\_" in middle of big numbers increase floating-point's readability:
			\codeblock{
				double y = 12\_45\_23\_\_23\_2323.90;  \cmark
			}
						
		\end{itemize}	
		
	\end{itemize}
	
\end{flushleft}


