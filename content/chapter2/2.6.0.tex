
\begin{flushleft}
	\begin{itemize}
		\item \textbf{Unordered collection of key and value pair}, enclosed in "\{\}", wherein key is unique and values can be duplicate. Key and value can be any datatype.
		\begin{tcolorbox}[breakable,notitle,boxrule=1pt,colback=pink,colframe=pink]
			\color{black}
			\fontdimen2\font=8pt
			Syntax: \{key:value, key:value, ...\}
			\fontdimen2\font=4pt
		\end{tcolorbox}
		
		Eg:	
		\begin{tcolorbox}[breakable,notitle,boxrule=-0pt,colback=code,colframe=code]
			\color{white}
			\fontdimen2\font=8pt
			a=\{ "name":"Raman" , "age":56 , "skill":"Python" \}
			\newline
			print(a,type(a))
			\fontdimen2\font=4pt
		\end{tcolorbox}
		
		Output:
		\begin{tcolorbox}[breakable,notitle,boxrule=-0pt,colback=output,colframe=output]
			\color{black}
			\{'name': 'Raman', 'age': 56, 'skill': 'Python'\} <class 'dict'>
			\fontdimen2\font=4pt
		\end{tcolorbox}
		
		\bigskip
		
		\item Dictionary is mutable datatype. You cannot add, remove or change existing list members.
		\newline
		Eg:			
		\begin{tcolorbox}[breakable,notitle,boxrule=-0pt,colback=code,colframe=code]
			\color{white}
			\fontdimen2\font=8pt
			info=\{ "name":"Raman" , "age":56 , "skill":"Python" \} \newline
			info['name']="Ram" \newline
			info['city']="Pune" \newline
			print(info)
			\fontdimen2\font=4pt
		\end{tcolorbox}
		
		Output:
		\begin{tcolorbox}[breakable,notitle,boxrule=-0pt,colback=output,colframe=output]
			\color{black}
			{'name': 'Ram', 'age': 56, 'skill': 'Python', 'city': 'Pune'}
			\fontdimen2\font=4pt
		\end{tcolorbox}
		
		\bigskip
		\item Being mutable datatype, memory address of similar tuple datatype is different.
		
	\end{itemize}
	
\end{flushleft}

\newpage

