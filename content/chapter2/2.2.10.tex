
\begin{flushleft}
	
	\begin{itemize}
		\item For integral datatypes like byte, short, int \& long, we can specify literal value in the following base:
		\begin{itemize}
			\item \textbf{Decimal literal (base-10)}:
			\begin{itemize}
				\item Allowed digits are 0-9
				\codeblock{
					int x = 10;
				}
			\end{itemize}

		\item \textbf{Binary literal (base-2)}:
		\begin{itemize}
			\item From Java 1.7 version, integral literal can be represented as binary value.
			\item Allowed digits are 0 and 1
			\item Literal value should be pre-fixed with "0B" or "0b"
			\codeblock{
				int x = 0B10; \newline
				int y = 0b10101;
			}
		\end{itemize}

			
		\item \textbf{Octal literal (base-8)}:
		\begin{itemize}
			\item Allowed digits are 0-7
			\item Literal value should be pre-fixed with 0
			\codeblock{
					int x = 017;
			}
		\end{itemize}
	
		\item \textbf{Hexadecimal literal (base-16)}:
		\begin{itemize}
			\item Allowed digits are 0-9, a-f or A-F
			\item Literal value should be pre-fixed with "0X" or "0x"
			\codeblock{
				int x = 0X13aA; \\
				int x = 0x45Fe;
			}
		\end{itemize}

		\item \textbf{Usage of \_ in numeric literal}:
		\begin{itemize}
			\item From Java 1.7 version, we can use "\_" in middle of big numbers to increase integer's readability.
			\item At the time of compilation, these "\_" symbols will be removed automatically.
			\bigskip
			\codeblock{
				int x = 78\_32\_34\_23; \newline
				int y = 0Xaa\_ff\_11; \newline
				int z = 034\_12\_10; \newline
				int a = 0B11\_\_00\_10\_11;
			}
			\item "\_" symbol cannot be used in the starting or end of integer. Below are \textbf{invalid} declarations:
			\bigskip
			\codeblock{
				int x = \_78\_32\_34\_23; \newline
				int y = 0Xaa\_ff\_11\_; 
			}
						
			
		\end{itemize}
	\end{itemize}
		
	\end{itemize}

	\newpage
	\textbf{Program to convert octal and hexadecimal form of integer to decimal form:}

	\codeblockfull{test.java}{
		package Starter; \newline
		class test \newline
		\{ \newline
		\s	public static void main (String[] args) \newline
		\hphantom{} \hphantom{}	\{ \newline
		\hphantom{} \hphantom{} \hphantom{} \hphantom{}	int x = 10; \newline
		\hphantom{} \hphantom{} \hphantom{} \hphantom{}	int y = 061; \newline
		\hphantom{} \hphantom{} \hphantom{} \hphantom{} int z = 0x9a; \newline
		\hphantom{} \hphantom{} \hphantom{}	\hphantom{}	System.out.println(x+","+y+","+z); \newline
		\hphantom{} \hphantom{}	\} \newline
		\} 	
	}
	\outputblock{
		10,49,154
	}

	\textbf{Output Explaination:}
	
	In Java, integeral literals are always represents in decimal literals forms: 
			\begin{itemize}	
				\item Octal to decimal form: 
					\[ (61)_8 = (?)_{10} \]
					\[ 6 \times 8^{1} + 1 \times 8^{0} = 49  \]
				\item Hexadecimal to decimal form:
					\[ (9a)_{16} = (?)_{10} \]
					\[ 9 \times 16^{1} + 10 \times 16^{0} = 154  \]
			\end{itemize}
			
\end{flushleft}

\newpage

