

\begin{flushleft}
	
	\begin{itemize}
		\item \textbf{Encapsulation:}
		\begin{itemize}
			\item Technique of bundling data and methods in a class. 
			\item With encapsulation, data cannot be accessed from outside the class. 
			\item This protects the data from accidental modification.
			
			\newimage{0.55}{content/chapter0/images/new25.png}
		\end{itemize}
		
		\item \textbf{Inheritance:}
		\begin{itemize}
			\item Inheritance is one class acquires the properties (methods and fields) of another class.
			\item It allows code reusability and saves time and effort.
			\newimage{0.55}{content/chapter0/images/new26.png}
		\end{itemize}
	
		\newpage
		\item \textbf{Polymorphism: }
		\begin{itemize}
			\item Polymorphism is ability of objects of different classes to be treated as if they belong to a common superclass. 
			\item Polymorphism allows methods to be written that can work with objects of many different classes, as long as they share a common interface. 
			\item This enables code to be more flexible and adaptable to changing requirements.
			\newimage{0.4}{content/chapter0/images/poly.png}
			
		\end{itemize}
	
		\item \textbf{Abstraction:}
		\begin{itemize}
			\item Abstraction hides implementation details.
			\item It shows only the essential features of an object. 
			\newimage{0.55}{content/chapter0/images/new27.png}
		\end{itemize}
		
	
	\end{itemize}
		
	
\end{flushleft}




