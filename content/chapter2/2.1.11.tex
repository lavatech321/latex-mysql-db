

\begin{flushleft}
	
	\textbf{O}bject-\textbf{O}riented \textbf{P}rogramming (OOP) is a programming paradigm that revolves around the concept of "objects". 
	\newline
	\newline
	In Java, OOP is implemented through the use of classes and objects.
	\newline \newline
	\textbf{Class \& Object}	
	\begin{itemize}
		\item A class is a blueprint for an \textbf{object}. 
		\item It tells the JVM how to make an object of that particular type.
		\item An Object is real world entity. It is an instance of class.
		
		\newimage{0.5}{content/chapter0/images/new20.png}
		
		\item A class consists of instance \textbf{variable and methods}:
		\begin{itemize}
			\item \textbf{Instance variable:} Represents the object data.
			\item \textbf{Methods:} Things an object can do are called methods.
			\newimage{0.38}{content/chapter0/images/new21.png}
		\end{itemize}
		
		\newpage
		\item Java code for class and object would look something like below:
		
		\newimage{0.5}{content/chapter0/images/new22.png}
		
		\newimage{0.5}{content/chapter0/images/new23.png}	
		
	\end{itemize}
	
\end{flushleft}
