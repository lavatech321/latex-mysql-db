\setlength{\columnsep}{3pt}
\begin{flushleft}

	An identifier refers to a name used to identify database objects such as tables, columns, indexes, views, procedures, and other database elements.
	
	\newline
	Rules for Identifiers:
	\begin{itemize}
		\item Identifiers can consist of \textbf{letters}, \textbf{numbers}, and \textbf{underscore character} (\_).
		\item They must \textbf{start with a letter}.
		\item They can be up to \textbf{30 characters} in length.
		\item Identifiers are \textbf{case-insensitive}. However, you can enclose identifiers in double quotes to make them case-sensitive. For example, "My\_Table" and "my\_table" would be treated as different identifiers.
		\item You \textbf{cannot use reserved words as identifiers} without enclosing them in double quotes.
		\item Common naming conventions is \textbf{camelCase or underscores} to separate words in identifiers (e.g., employee\_id, customerName).
	\end{itemize}


		
\end{flushleft}

\newpage





