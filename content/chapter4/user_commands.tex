\setlength{\columnsep}{3pt}
\begin{flushleft}
	
	Let's see some commands to create, update, delete users in Linux OS.
	
\paragraph{How to creating a new user?}
\bigskip
\textbf{useradd}: Used to create new user.
		\begin{tcolorbox}[breakable,notitle,boxrule=1pt,colback=pink,colframe=pink]
			\color{black}
			Syntax:  useradd username
		\end{tcolorbox}
		On adding new user, some actions are performed:	
		\begin{itemize}
			\item The home directory of this user will be \textbf{/home/username}
			\item The default shell assigned to thus user will be \textbf{/bin/bash}
			\item \textbf{Group named "username" is created}.
			\item \textbf{User "username" is then made a member of the group "username".}
		\end{itemize}

	\paragraph{Command options}
		\begin{enumerate}[label=(\alph*)]
			\item \textbf{–g}: Assign primary group.
			\bigskip
			\begin{tcolorbox}[breakable,notitle,boxrule=0pt,colback=pink,colframe=pink]
				\color{black}
				\fontdimen2\font=1em
				Syntax: useradd -g primary\_group\_name username
				\fontdimen2\font=4pt
			\end{tcolorbox}
			Eg:
			\bigskip
			\begin{tcolorbox}[breakable,notitle,boxrule=-0pt,colback=black,colframe=black]
				\color{green}
				\fontdimen2\font=1em
				\$ useradd –g javadevl shekhar
				\fontdimen2\font=4pt
			\end{tcolorbox}
			
			\item \textbf{–G}: Assign secondary groups.
			\bigskip
			\begin{tcolorbox}[breakable,notitle,boxrule=0pt,colback=pink,colframe=pink]
				\color{black}
				\fontdimen2\font=1em
				Syntax: useradd -G secondary\_group\_name username
				\fontdimen2\font=4pt
			\end{tcolorbox}
			Eg:
			\bigskip
			\begin{tcolorbox}[breakable,notitle,boxrule=-0pt,colback=black,colframe=black]
				\color{green}
				\fontdimen2\font=1em
				\$ useradd –G cdevl,perldevl shekhar
				\fontdimen2\font=4pt
			\end{tcolorbox}
			

			\item \textbf{–u}: Assign specific UID. UID should be more than 500 for normal users.
			\bigskip
			\begin{tcolorbox}[breakable,notitle,boxrule=0pt,colback=pink,colframe=pink]
				\color{black}
				\fontdimen2\font=1em
				Syntax: useradd –u UID username
				\fontdimen2\font=4pt
			\end{tcolorbox}
			Eg:
			\bigskip
			\begin{tcolorbox}[breakable,notitle,boxrule=-0pt,colback=black,colframe=black]
				\color{green}
				\fontdimen2\font=1em
				\$ useradd –u 601 shekhar
				\fontdimen2\font=4pt
			\end{tcolorbox}
			

			\item \textbf{–s}: Change user shell.
			\bigskip
			\begin{tcolorbox}[breakable,notitle,boxrule=0pt,colback=pink,colframe=pink]
				\color{black}
				\fontdimen2\font=1em
				Syntax: useradd –s shell\_name username
				\fontdimen2\font=4pt
			\end{tcolorbox}
			Eg:
			\bigskip
			\begin{tcolorbox}[breakable,notitle,boxrule=-0pt,colback=black,colframe=black]
				\color{green}
				\fontdimen2\font=1em
				\$ useradd –s /sbin/nologin shekhar
				\fontdimen2\font=4pt
			\end{tcolorbox}
			\bigskip
			\begin{tcolorbox}[breakable,notitle,boxrule=-0pt,colback=yellow,colframe=yellow]
				\color{black}
				Note: Shell /sbin/nologin do not allow shekhar to log in. Usually used for system users like ftp, squid etc.
			\end{tcolorbox}
			

			\item \textbf{–h}: Change user home directory.
			\bigskip
			\begin{tcolorbox}[breakable,notitle,boxrule=0pt,colback=pink,colframe=pink]
				\color{black}
				\fontdimen2\font=1em
				Syntax: useradd –h directory\_name –m username
				\fontdimen2\font=4pt
			\end{tcolorbox}
			Eg:
			\bigskip
			\begin{tcolorbox}[breakable,notitle,boxrule=-0pt,colback=black,colframe=black]
				\color{green}
				\fontdimen2\font=1em
				\$ useradd –h /mnt/shekar –m shekhar
				\fontdimen2\font=4pt
			\end{tcolorbox}
		\end{enumerate}

		\bigskip
		\bigskip

\newpage

\paragraph{How to assign or change user’s password?}

\bigskip
	\textbf{passwd}: Change shekhar’s password.
	
	\begin{tcolorbox}[breakable,notitle,boxrule=0pt,colback=pink,colframe=pink]
		\color{black}
		\fontdimen2\font=1em
		Syntax: passwd username
		\fontdimen2\font=4pt
	\end{tcolorbox}
	On executing the command, you will be asked to set password twice.
	
	\begin{tcolorbox}[breakable,notitle,boxrule=0pt,colback=yellow,colframe=yellow]
		\color{black}
		Note: Only root or super user can set the password of any user.
	\end{tcolorbox}
	
	\paragraph{Command options}
	
	\begin{itemize}
		\item \textbf{- -stdin}: Change password by providing on command line itself.
		\begin{tcolorbox}[breakable,notitle,boxrule=0pt,colback=pink,colframe=pink]
			\color{black}
			\fontdimen2\font=1em
			Syntax: echo “new\_password” | passwd ---stdin username
			\fontdimen2\font=4pt
		\end{tcolorbox}
		Eg:
		\bigskip
		\begin{tcolorbox}[breakable,notitle,boxrule=-0pt,colback=black,colframe=black]
			\color{green}
			\fontdimen2\font=1em
			\$ echo “shekhar@12345” | passwd ---stdin shekhar
			\fontdimen2\font=4pt
		\end{tcolorbox}
	\end{itemize}
	


\newpage

\paragraph{How to modify an existing user?}
\bigskip
	\textbf{usermod}: Modify existing user.
	\paragraph{Command options}
	\begin{enumerate}[label=(\alph*)]
		\item \textbf{–g}: Change user's primary group.
		\bigskip
		\begin{tcolorbox}[breakable,notitle,boxrule=0pt,colback=pink,colframe=pink]
			\color{black}
			\fontdimen2\font=1em
			Syntax: usermod –g primary\_group username
			\fontdimen2\font=4pt
		\end{tcolorbox}
		Eg:
		\bigskip
		\begin{tcolorbox}[breakable,notitle,boxrule=-0pt,colback=black,colframe=black]
			\color{green}
			\fontdimen2\font=1em
			\$ usermod –g cdevl shekhar
			\fontdimen2\font=4pt
		\end{tcolorbox}
		
		\item \textbf{–G}: Change user's secondary group.
		\bigskip
		\begin{tcolorbox}[breakable,notitle,boxrule=0pt,colback=pink,colframe=pink]
			\color{black}
			\fontdimen2\font=1em
			Syntax: usermod -G secondary\_group\_name username
			\fontdimen2\font=4pt
		\end{tcolorbox}
		Eg:
		\bigskip
		\begin{tcolorbox}[breakable,notitle,boxrule=-0pt,colback=black,colframe=black]
			\color{green}
			\fontdimen2\font=1em
			\$ usermod –G javadevl shekhar
			\fontdimen2\font=4pt
		\end{tcolorbox}
		
		
		\item \textbf{–L}: Lock (i.e. disable) user.
		\bigskip
		\begin{tcolorbox}[breakable,notitle,boxrule=0pt,colback=pink,colframe=pink]
			\color{black}
			\fontdimen2\font=1em
			Syntax: usermod -L username
			\fontdimen2\font=4pt
		\end{tcolorbox}		
		
		\item \textbf{–U}: Unlock use.
		\bigskip
		\begin{tcolorbox}[breakable,notitle,boxrule=0pt,colback=pink,colframe=pink]
			\color{black}
			\fontdimen2\font=1em
			Syntax: usermod –U username
			\fontdimen2\font=4pt
		\end{tcolorbox}
		
		\item \textbf{–s}: Change user's login shell.
		\bigskip
		\begin{tcolorbox}[breakable,notitle,boxrule=0pt,colback=pink,colframe=pink]
			\color{black}
			\fontdimen2\font=1em
			Syntax: usermod –s shell\_name username
			\fontdimen2\font=4pt
		\end{tcolorbox}
		Eg:
		\bigskip
		\begin{tcolorbox}[breakable,notitle,boxrule=-0pt,colback=black,colframe=black]
			\color{green}
			\fontdimen2\font=1em
			\$ usermod –s /bin/ksh shekhar
			\fontdimen2\font=4pt
		\end{tcolorbox}
	
	
		\item \textbf{–d}: Change user’s home directory.
		\bigskip
		\begin{tcolorbox}[breakable,notitle,boxrule=0pt,colback=pink,colframe=pink]
			\color{black}
			\fontdimen2\font=1em
			Syntax: usermod –d directory\_name username
			\fontdimen2\font=4pt
		\end{tcolorbox}
		Eg:
		\bigskip
		\begin{tcolorbox}[breakable,notitle,boxrule=-0pt,colback=black,colframe=black]
			\color{green}
			\fontdimen2\font=1em
			\$ usermod –d /opt/java shekhar
			\fontdimen2\font=4pt
		\end{tcolorbox}
	\end{enumerate}

\bigskip
\bigskip

\newpage
	
\paragraph{How to delete a user?}

\bigskip
\begin{tcolorbox}[breakable,notitle,boxrule=-0pt,colback=red,colframe=red]
	\color{white}
	Standard Practice: You should not delete a user if he/she leaves the organization. You
	should lock him/her.
\end{tcolorbox}
	


\textbf{userdel}: Delete user \& entry is removed from /etc/passwd and /etc/shadow files.
	\begin{tcolorbox}[breakable,notitle,boxrule=0pt,colback=pink,colframe=pink]
		\color{black}
		\fontdimen2\font=1em
		Syntax: userdel username
		\fontdimen2\font=4pt
	\end{tcolorbox}
	\paragraph{Command options}
	\begin{itemize}
		\item \textbf{-r}: Delete user along with home directory.
		\begin{tcolorbox}[breakable,notitle,boxrule=0pt,colback=pink,colframe=pink]
			\color{black}
			\fontdimen2\font=1em
			Syntax: userdel -r username
			\fontdimen2\font=4pt
		\end{tcolorbox}
	\end{itemize}

\end{flushleft}

\newpage

