\setlength{\columnsep}{3pt}
\begin{flushleft}
		
	\textbf{default}:
	\begin{itemize}
		\item DEFAULT constraint specify default value for column in table. 
		\item Used when INSERT statement does not provide a value for that column. 

		\syntaxblock{
			CREATE TABLE table\_name ( \\
			\s column1 datatype DEFAULT default\_value, \\
			\s column2 datatype DEFAULT default\_value, \\
			\s -- other columns \\
			);
		}
		Eg: 
		\commandblock{
			CREATE TABLE employees ( \\
			\s employee\_id NUMBER(5), \\
			\s first\_name VARCHAR2(50) DEFAULT 'John', \\
			\s dob date DEFAULT SYSTIMESTAMP, \\
			\s joining\_date date DEFAULT SYSDATE, ); \\
			\\
			insert into employees(employee\_id, first\_name) values (101, 'Ravi'); \\
			\\
			insert into employees(employee\_id, first\_name, dob\_date, joining\_date ) values (101, default, default, default); \\
			\\
			CREATE TABLE employees ( \\
			\s first\_name VARCHAR2(50) DEFAULT 'John', \\
			\s dob date DEFAULT SYSTIMESTAMP, \\
			\s joining\_date date DEFAULT SYSDATE, \\
			); 
		}
		\newpage
		\codecontinue{
			insert into employees(first\_name, dob\_date, joining\_date ) values (default, default, default);
		}
	\end{itemize}


	\textbf{not null}:
	\begin{itemize}
		\item NOT NULL ensures a column cannot contain NULL values. 
		\bigskip
		\noteblock{
			Default has highest priority than Not NULL, so first define default than not null.
		}
		\syntaxblock{
			CREATE TABLE table\_name ( \\
			\s column\_name datatype NOT NULL, \\
			\s -- other columns \\
			); \\
			or \\
			CREATE TABLE table\_name ( \\
			\s column\_name datatype CONSTRAINT constraint\_name NOT NULL, \\
			\s -- other columns \\
			); 
		}
		Eg:
		\commandblock{
			create table student( \\
			\s no number CONSTRAINT c1 not null \\
			); \\
			CREATE TABLE students ( \\
			\s student\_id NUMBER(5) NOT NULL, \\
			\s first\_name VARCHAR2(50) NOT NULL,	\\
			); 	
		}
		\newpage
		\codecontinue{
			create table students( \\
			\s no number DEFAULT 0 NOT NULL,  \\
			); 
		}
	\end{itemize}

	\textbf{check}
	\begin{itemize}
		\item CHECK constraint enforces a condition on column or entire table. 
		\item It ensures data entered into the column meets specific criteria or rules. 
		\item Column level check constraints:
		\bigskip
		\syntaxblock{
			CREATE TABLE table\_name ( \\
			column\_name datatype CONSTRAINT constraint\_name CHECK (condition), \\
			-- other columns \\
			);	 \\
			or \\
			CREATE TABLE table\_name ( \\
			column\_name datatype CHECK (condition), \\
			-- other columns \\
			);	 
		}
		\item Table level check constraints:
		\bigskip
		\syntaxblock{
			CREATE TABLE table\_name ( \\
			column\_name datatype, \\
			CONSTRAINT constraint\_name CHECK (condition) \\
			);	 
		}
		Eg: 
		\commandblock{
			CREATE TABLE orders ( \\
			\s order\_id NUMBER(5), \\
			\s order\_total NUMBER(10, 2) CHECK (order\_total > 0) \\
			);
		}

		\commandblock{
			CREATE TABLE emp( \\
			\s eno number,  \\
			\s CONSTRAINT con1 CHECK(eno IN (1,2,3)) \\
			); \\
			CREATE TABLE emp1( \\
			\s eno number  \\
			\s DEFAULT 10 CHECK(eno in (10,20,30))  \\
			\s NOT NULL \\
			);
		}
		
	\end{itemize}	
\end{flushleft}
\newpage