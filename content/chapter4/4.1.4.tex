\setlength{\columnsep}{3pt}
\begin{flushleft}
	
	\textbf{foreign key}
	\begin{itemize}
		\item A foreign key establishes link between two tables.
		\item It ensures one table's column(foreign key) correspond to another table's primary/unique key(referenced key).
		\item Reference column may be present in same/different table.
		\item It allows:
		\begin{itemize}
			\item Duplicate values
			\item NULL values	
		\end{itemize}
		\item It can be defined on single/multiple column (compsite key).
		\item Can be defined at table level or column level.
		
		\newimage{0.4}{content/chapter4/images/new.png}
	
		\newpage
		\syntaxblock{
			// Table level \\
			CREATE TABLE child\_table ( \\
			\s column1 datatype, \\
			\s column2 datatype, \\
			\s referenced\_column datatype, \\
			\s CONSTRAINT fk\_name FOREIGN KEY \\ 
			\s (referenced\_column) \\ 
			\s REFERENCES parent\_table (primary\_key\_column) \\
			); \\
			// Column level \\
			CREATE TABLE child\_table ( \\
			\s column1 datatype, \\
			\s column2 datatype, \\
			\s referenced\_column datatype CONSTRAINT fk\_name \\
			\s REFERENCES parent\_table(primary\_key\_column) \\
			);
		}
		Eg:
		\commandblock{
			CREATE TABLE product( \\
			\s pno number PRIMARY KEY, \\
			\s pname varchar(20), \\
			); \\
			CREATE TABLE productOrder( \\
			\s no number, \\
			\s pno number CONSTRAINT fk  \\
			\s REFERENCES product(pno) \\
			); 
		}	
		\newpage
		\codecontinue{
			CREATE TABLE productOrder( \\
			\s no number, \\
			\s CONSTRAINT fk FOREIGN KEY(pno)  \\
			\s REFERENCES product(pno) \\
			);
		}
	\end{itemize}
	
	\textbf{ON DELETE SET NULL}
	\begin{itemize}
		\item Specify a foreign key column should be set NULL when corresponding row in parent table is deleted. 
		\bigskip
		\syntaxblock{
			CREATE TABLE child\_table ( \\
			\s column1 datatype, \\
			\s parent\_reference datatype, \\
			\s CONSTRAINT fk\_parent\_reference  \\
			\s FOREIGN KEY (parent\_reference) \\
			\s REFERENCES parent\_table(parent\_id) \\
			\s \textbf{ON DELETE SET NULL} \\
			);
		}
		\item Eg:
		\commandblock{
			CREATE TABLE product( \\
			\s pno number PRIMARY KEY, \\
			\s pname varchar(20) \\
			); 
		}
		\newpage
		\codecontinue{
			create table productOrder( \\
			\s id number, \\
			\s pno number, \\
			\s odate date, \\
			\s constraint fk foreign key(pno)  \\
			\s references product(pno) \\
			\s \textbf{on delete set null} \\
			); \\
			insert into product values(101,'cadbury'); \\
			insert into product values(102,'5-star'); \\
			insert into product values(103,'kitkat'); \\
			insert into productOrder values(201,101,SYSDATE); \\
			insert into productOrder values(202,101,SYSDATE); \\
			insert into productOrder values(203,103,SYSDATE); \\
			insert into productOrder values(204,102,SYSDATE);  \\
			delete from product where pno=101; 
		}
		\bigskip
		\outputblock{
		select * from productOrder; \\
		ID 	\s	PNO \s	ODATE \\
		201	\s  - 	\s	06-OCT-23 \\
		202	\s -  	\s	06-OCT-23 \\
		203 \s	103 \s	06-OCT-23 \\
		204	\s 	102 \s	06-OCT-23 
		}
	\end{itemize}
	
	\newpage
	
	\textbf{ON DELETE CASCADE}
	\begin{itemize}
		\item Specify when a row in the parent table is deleted, all related rows in the child table should be deleted. 
		\bigskip
		\syntaxblock{
			CREATE TABLE child\_table ( \\
			\s column1 datatype, \\
			\s parent\_reference datatype, \\
			\s CONSTRAINT fk\_parent\_reference  \\
			\s FOREIGN KEY (parent\_reference) \\
			\s REFERENCES parent\_table(parent\_id) \\
			\s \textbf{ON DELETE CASCADE} \\
			);
		}
		\item Eg:
		\commandblock{
			CREATE TABLE product( \\
			\s pno number PRIMARY KEY, \\
			\s pname varchar(20) \\
			); \\ 
			create table productOrder( \\
			\s id number, \\
			\s pno number, \\
			\s odate date, \\
			\s constraint fk foreign key(pno)  \\
			\s references product(pno) \\
			\s \textbf{on delete cascade} \\
			); \\
			insert into product values(101,'cadbury'); \\
			insert into product values(102,'5-star'); \\
			insert into product values(103,'kitkat'); \\
			insert into productOrder values(201,101,SYSDATE); 

		}
		\newpage
		\codecontinue{
			insert into productOrder values(202,101,SYSDATE); \\
			insert into productOrder values(203,103,SYSDATE); \\
			insert into productOrder values(204,102,SYSDATE);  \\
			delete from product where pno=101; 
		}	
		\bigskip
		\outputblock{
			select * from productOrder; \\
			ID \s	PNO \s	ODATE \\
			203 \s	103 \s	06-OCT-23 \\
			204 \s	102 \s	06-OCT-23
		}
	\end{itemize}
	
\end{flushleft}


