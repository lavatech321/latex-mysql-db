\setlength{\columnsep}{3pt}
\begin{flushleft}
	
	\textbf{length variable}:
	\begin{itemize}
		\item length is final variable applicable for arrays.
		\item length variable is used to display size of an array.
		\item Value returned by length is fixed as array once created cannot change it's size.
		\bigskip
		\codeblock{
			int[] x = new int[6]; \\
			System.out.println(x.length);  \cmark
		}
		\bigskip
		\outputblock{
			6
		}
		
		\item length variable is \textbf{not} applicable on string objects.
		\bigskip
		\codeblock{
			String s="lavatech"; \\
			System.out.println(s.length); \xmark
		}
		
		\item In multi-dimensional arrays, length variable represents only base size, but not total size.
		\codeblock{
			int[][] x = new int[6][3]; \\
			System.out.println(x.length);
		}
		\bigskip
		\outputblock{
			6
		}
		
	\end{itemize}
	
	\textbf{length()}:
	\begin{itemize}
		\item length() is present in String class.
		\item length() method is final variable applicable for string objects.
		\item It returns number of characters present in the string.
		\bigskip
		\codeblock{
			String s="lavatech"; \\
			System.out.println(s.length()); \cmark
		}
		\bigskip
		\outputblock{
			8
		}
		\bigskip
		
		\item length variable is applicable for arrays, but not for string objects.
		\item length() is applicable for string objects, but not for arrays.
		Example:
		\codeblock{
			String[] s= \{"A","AA","AAA"\}; \\
			System.out.println(s.length); \cmark \\
			System.out.println(s.length()); \xmark  \\
			System.out.println(s[0].length); \xmark \\
			System.out.println(s[0].length()); \cmark
		}
		\bigskip	
		\outputblock{
			3 \\
			error \\
			error \\
			1
		}
		\bigskip
		\item There is no direct way to find total length of multi-dimensional array. Total length of multi-dimensional array can be found as follows:
		\codeblock{
			int[][] x = new int[3][3]; \\
			System.out.println(x.length); \\
			System.out.println(x[0].length+x[1].length+x[2].length);
		}
		\outputblock{
			3 \\
			9
		}
	\end{itemize}
\end{flushleft}
