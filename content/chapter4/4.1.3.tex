\setlength{\columnsep}{3pt}
\begin{flushleft}

	\textbf{unique}
	\begin{itemize}
		\item \textbf{Unique} does not allow \textbf{duplicate} values
		\item It allows \textbf{NULL} values (Any number of NULLs)
		\item Can be defined at single or multiple \textbf{columns} or at \textbf{table level}
		\syntaxblock{
			// Column level \\
			CREATE TABLE table\_name ( \\
			\s column1 datatype UNIQUE, \\
			\s column2 datatype CONSTRAINT constraint\_name UNIQUE \\
			);  \\
			// Table level \\
			CREATE TABLE table\_name ( \\
			\s column1 datatype, \\
			\s column2 datatype, \\
			CONSTRAINT constraint\_name UNIQUE(column1, column2,..)\\
			); 
		}
		Eg: Column level
		\commandblock{
			CREATE TABLE product( \\
			\s	id number UNIQUE, \\
			\s	name varchar(50) UNIQUE \\
			); \\
			CREATE TABLE product( \\
			name varchar(50) CONSTRAINT con1 UNIQUE(column\_name) \\
			);
		}
		\newpage
		Eg: Table level
		\commandblock{
			create table product1( \\
			\s no number, \\
			\s constraint constraint\_name unique\_no unique(no) \\
			); \\
			create table product2( \\
			\s no number, \\
			\s name varchar2(20), \\
			\s unique(no,name) \\
			);
		}
	\end{itemize}

	\textbf{Primary key}
	\begin{itemize}
		\item A primary key(PK) constraint can:
		\begin{itemize}
			\item Enforces data uniqueness.
			\item Prevent duplicate or NULL values in column
		\end{itemize}
		\item Other tables can reference the PK column(s) as foreign keys.
		\item PK on multiple columns is called composite primary key.
		\item Oracle database automatically creates a unique index on the primary key column(s). 
		\newimage{0.3}{content/chapter4/images/new.png}
		\syntaxblock{
			// Single column primary key \\
			CREATE TABLE example ( \\
			\s column1 datatype PRIMARY KEY, \\
			\s column2 datatype, \\
			); \\	 
 			// Multiple column primary key \\
			CREATE TABLE example ( \\
			\s column1 datatype, \\
			\s column2 datatype, \\
			\s CONSTRAINT pk\_example PRIMARY KEY(column1, column2...) \\
			);	
		}
		Eg:
		\commandblock{
			create table product( \\
			\s no number primary key, \\
			\s name varchar2(30) \\
			); \\
			CREATE TABLE sample( \\
			\s no number, \\
			\s name varchar2(30), \\
			\s CONSTRAINT pk PRIMARY KEY(no, name) \\
			);
		}
			
	\end{itemize}
 	
\end{flushleft}

\newpage















