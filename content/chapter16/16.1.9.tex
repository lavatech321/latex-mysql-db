\setlength{\columnsep}{3pt}
\begin{flushleft}
	
	Throwable class defines the following methods to print exception information:
	\tabletwo{
			\textbf{Method} & \textbf{Printable format} \\
			\hline
			printStackTrace() & Name of exception: Description  \newline
			Stack Trace \newline
			\newline
			Internally, default exception handler will use  printStackTrace() method to print exception information to the console \\
			\hline
			toString() & Name of exception: Description \\
			\hline
			getMessage() & Description \\
		}
		
		 Eg:
		\codeblockfull{demo.java}{
			public class demo \{ \\
			\s public static void main(String[] args) \{ \\
			\s \s try \{ \\
			\s \s \s System.out.println(5/0); \\
			\s \s		\} \\
			\s \s		catch (ArithmeticException e) \{ \\
			\s \s		// toString() \\
			\s \s \s 	System.out.println("Output for:  toString()"); \\
			\s \s \s 	System.out.println(e); \\
			\s \s \s 	System.out.println(e.toString()); \\
			\s \s 	//printStackTrace() \\
			\s \s \s 	System.out.println("Output for:  printStackTrace()"); \\
			\s \s \s 	e.printStackTrace(); \\
			\s \s 	//getMessage() \\
			\s \s \s 	System.out.println("Output for: getMessage()"); \\
			\s \s \s 	System.out.println(e.getMessage()); \\
			\s \s		\} \} \}
		}

		\outputblock{
			Output for: toString() \\
			java.lang.ArithmeticException: / by zero \\
			java.lang.ArithmeticException: / by zero \\
			Output for: printStackTrace() \\
			java.lang.ArithmeticException: / by zero \\
			at demo.main(demo.java:5) \\
			Output for: getMessage() \\
			/ by zero 
		}

	
\end{flushleft}

\newpage


