\setlength{\columnsep}{3pt}
\begin{flushleft}
	
	\begin{itemize}
		\item Exception propagation is process by which an exception is passed from one method to another in the call stack until:
		\begin{itemize}
			\item It is either caught and handled
			\item Or causes the program to terminate if left uncaught
		\end{itemize}
		\item When a method throws an exception but does not catch it, the exception is propagated up the call stack to the calling method. 
		
		\item Consider below code where exception is handled by main() ultimately:
		\bigskip
		\codeblockfull{demo.java}{
			public class demo \{ \\
			\s	public static void main(String[] args) \{ \\
			\s \s		try \{ \\
			\s \s \s    	test1();   \\
			\s \s		\} \\
			\s \s		catch(Exception e) \{ \\
			\s \s \s			System.out.println("Caught exception:"); \\
			\s \s \s			e.printStackTrace(); \\
			\s \s		\} \\
			\s \s	\} 
		}
		\newpage
		\codecontinue{
			\s  	public static void test1() \{ \\
			\s \s		test2(); \\
			\s	\} \\
			\\
			\s	public static void test2() \{ \\
			\s \s		test3(); \\
			\s	\} \\
			\\
			\s 	static void test3() \{ \\
			\s \s		int arr[] = new int[2]; \\
			\s \s		System.out.println(arr[9]); \\
			\s	\} \\
			\}
		}
		\bigskip
		\outputblock{
			Caught exception: \\
			java.lang.ArrayIndexOutOfBoundsException: Index 9 out of bounds for length 2 \\
			at demo.test3(demo.java:20) \\
			at demo.test2(demo.java:16) \\
			at demo.test1(demo.java:13) \\
			at demo.main(demo.java:5)
		}
		\bigskip
		\textbf{Explanation:}
		\begin{itemize}
			\item When method3() is called, it tries to access an element in the array with an index that is out of bounds. 
			\item This results in an ArrayIndexOutOfBoundsException. 
			\item Since method3() does not catch this exception, it is propagated up the call stack through method2() to method1() and finally caught in the main() method.
		\end{itemize}
	\end{itemize}
	
\end{flushleft}
\newpage





