\setlength{\columnsep}{3pt}
\begin{flushleft}
	
	Based on who is raising an exception, all exceptions are divided into 2 categories:

	\textbf{JVM exceptions}
	\begin{itemize}
		\item These are raised automatically by JVM, whenever a particular event occurs.
	\end{itemize}

	\item \textbf{Programmatic Exceptions}
	\begin{itemize}
		\item These are raised explicitly either by programmer or API developer to handle exceptional situations.
	\end{itemize}

	Below are some classifications of exception:
	
	\tabletwo{
		\textbf{JVM Exceptions} & \textbf{Programmatic Exceptions} \\
		\hline
		ArrayIndexOutofBoundsException & IllegalArgumentException \\
		\hline
		NullPointerException & NumberFormatException \\
		\hline
		ClassCastException & IllegalStateException \\
		\hline
		StackOverflowException & AssertionException \\
		\hline
		& NoClassDefFoundError \\
		\hline
		& ExceptionInInitializerError \\
	}
	
	\newpage
	\textbf{ArrayIndexOutofBoundsException}
	\begin{itemize}
		\item Parent Class: \textbf{RuntimeClassException}
		\item Type: \textbf{Unchecked Exception}
		\item Raised By: \textbf{JVM}
		\item Reason: Trying to access array element with out of range index.
		\item Eg:
		\bigskip
		\codeblockfull{demo.java}{
			public class demo \{ \\
			\s	public static void main(String[] args) \{ \\
			\s \s		int[] arr = new int[4]; \\
			\s \s		System.out.println(arr[0]); \\
			\s \s		System.out.println(arr[5]); \\
			\s	\} \\
			\}
		}
		\bigskip
		\outputblock{
			0 \\
			Exception in thread "main" \\
			java.lang.\textbf{ArrayIndexOutOfBoundsException}: Index 5 out of bounds for length 4 \\
			at demo.main(demo.java:7)
			
		}
	\end{itemize}
	
	\textbf{NullPointerException}
	\begin{itemize}
		\item Parent Class: \textbf{RuntimeClassException}
		\item Type: \textbf{Unchecked Exception}
		\item Raised By: \textbf{JVM}
		\item Reason: Trying to perform any operation on null.
		\item Eg:
		\newpage
		\codeblockfull{demo.java}{
			public class demo \{ \\
			\s public static void main(String[] args) \{ \\
			\s \s	String s=null; \\
			\s \s	System.out.println(s.length()); \\
			\s	\} \\
			\}
		}
		\bigskip
		\outputblock{
			Exception in thread "main" java.lang.\textbf{NullPointerException}
			at demo.main(demo.java:6)
		}
	\end{itemize}

	\textbf{ClassCastException:}
	\begin{itemize}
		\item Parent Class: \textbf{RuntimeClassException}
		\item Type: \textbf{Unchecked Exception}
		\item Raised By: \textbf{JVM}
		\item Reason: Trying to type cast parent object to child type.
		\item Eg:
		\codeblockfull{demo.java}{
			public class demo \{ \\
			\s	public static void main(String[] args) \{ \\
			\s \s		Object o = new Object(); \\
			\s \s		String s = (String)o; \\
			\s	\} \\
			\}
		}
		\bigskip
		\outputblock{
			Exception in thread "main" java.lang.\textbf{ClassCastException}...
			
		}
	\end{itemize}
	\newpage
	\textbf{StackOverflowException:}
	\begin{itemize}
		\item Parent Class: \textbf{Error}
		\item Type: \textbf{Unchecked Exception}
		\item Raised By: \textbf{JVM}
		\item Reason: Trying to perform recursive method class.
		\bigskip
		\codeblockfull{demo.java}{
			public class demo \{ \\
			\s public static void test1() \{ \\
			\s \s	test2(); \\
			\s	\} \\
			\s	public static void test2() \{ \\
			\s \s	test1(); \\
			\s	\} \\
			\s	public static void main(String[] args) \{ \\
			\s \s	test1(); \\
			\s	\} \\
			\}
		}
		\bigskip
		\outputblock{
			Exception in thread "main" java.lang.StackOverflowError \\
			at demo.test2(demo.java:7) \\
			at demo.test1(demo.java:4) \\
			at demo.test2(demo.java:7) \\
			.. \\
			.. \\
			.. \\
			.. \\
			..
		}
	\end{itemize}

	\newpage
	\textbf{ClassNotFoundException:}
	\begin{itemize}
		\item Parent Class: \textbf{Error}
		\item Type: \textbf{Unchecked Exception}
		\item Raised By: \textbf{JVM}
		\item Reason: Raised when JVM unable to be find required "\textbf{.class}" file.
		\item Eg:
		\bigskip
		\codeblock{
			\$ java Demo \\
			Error: Could not find or load main class Demo \\
			Caused by: java.lang.ClassNotFoundException: Demo
		}
	\end{itemize}
	
	\textbf{ExceptionInInitialiserError:}
	\begin{itemize} 
		\item Parent Class: \textbf{Error}
		\item Type: \textbf{Unchecked Exception}
		\item Raised By: \textbf{JVM}
		\item Reason: Raised when exception occurs while executing static variable assignments and static blocks.		
		\item Eg:
		\codeblockfull{demo.java}{
			public class demo \{ \\
			\s	static int i = 10/0; \\
			\}
		}
		\bigskip
		\outputblock{
			Error: Main method not found in class demo, please define the main method as: \\
			public static void main(String[] args) \\
			or a JavaFX application class must extend \\ javafx.application.Application
		}
	\end{itemize}
	
	\textbf{IllegalArgumentException}
	\begin{itemize}
		\item Parent Class: \textbf{RuntimeException}
		\item Type: \textbf{Unchecked Exception}
		\item Raised By: \textbf{Programmer or API developer}
		\item Reason: Indicate that a method has been invoked with illegal argument.
		\item Eg: The valid range of thread priorities is 1-10. Invalid thread priority will result in IllegalArgument exception:
		\bigskip
		\codeblockfull{demo.java}{
			public class demo \{ \\
			\s public static void main(String[] args) \{ \\
			\s \s		Thread t1 = new Thread(); \\
			\s \s		t1.setPriority(7); \\
			\s \s		t1.setPriority(15); \\
			\s	\} \\
			\}
		}
		\bigskip
		\outputblock{
			Exception in thread "main" java.lang.IllegalArgumentException \\
			at java.base/java.lang.Thread.setPriority(Thread.java:1141) \\
			at demo.main(demo.java:7)
		}
	\end{itemize}
	
	\textbf{NumberFormatException:}
	\begin{itemize}
		\item Parent Class: \textbf{IllegalArgumentException}
		\item Type: \textbf{Unchecked Exception}
		\item Raised By: \textbf{Programmer or API developer}
		\item Reason: Trying to convert string to number and the string is not properly formatted.
		\bigskip
		\codeblockfull{demo.java}{
			public class demo \{ \\
			\s public static void main(String[] args) \{ \\
			\s \s	int i1 = Integer.parseInt("67"); \\
			\s \s	int i2 = Integer.parseInt("one"); \\
			\s	\} \\
			\}	
		}
		\bigskip
		\outputblock{
			Exception in thread "main" java.lang.\textbf{NumberFormatException}: at java.base/java.lang.Integer.parseInt(Integer.java:652) \\
			at java.base/java.lang.Integer.parseInt(Integer.java:770) \\
			at demo.main(demo.java:6)
		}
	\end{itemize}
	
	
	\textbf{IllegalThreadStateException:}
	\begin{itemize}
		\item Parent Class: \textbf{RuntimeException}
		\item Type: \textbf{Unchecked Exception}
		\item Raised By: \textbf{Programmer or API developer}
		\item Reason: Indicate that a method has been invoked at a wrong time.
		\codeblockfull{demo.java}{
			public class demo \{ \\
			\s public static void main(String[] args) \{ \\
			\s \s		Thread t1 = new Thread(); \\
			\s \s		t1.start(); \\
			\s \s		t1.start(); // Output: \textbf{IllegalThreadStateException} \\
			\s	\} \\
			\}
		}
	\end{itemize}
	
	\newpage
	\textbf{AssertionError}
	\begin{itemize}
		\item Parent Class: \textbf{Error}
		\item Type: \textbf{Unchecked Exception}
		\item Raised By: \textbf{Programmer or API developer}
		\item Reason: 
		\begin{itemize}
			\item Used for debugging and testing purposes. 
			\item Indicates that certain condition is true during execution. 
			\item If the condition is false, the assert statement will throw an AssertionError.
			\item To enable the assertion checks, JVM with the \textbf{-ea} or \textbf{-enableassertions} command-line option.
		\end{itemize}

		\item Eg:
		\codeblockfull{demo.java}{
			public class demo \{	 \\
			\s	public static void main(String[] args) \{ \\
			\s \s 	int num = 14; \\
			\s	// Example 1: Simple assertion check \\
			\s		assert num >= 0; 
			// No AssertionError as num > 0 \\				
			\s	// Example 2: Assertion \& error message \\
			\s \s	int age = -5; \\
			\s \s	assert age >= 0 : "Age cannot be negative"; \\
			\s	\} \\
			\}	
		}
		\bigskip
		\outputblock{
			\$ javac demo.java  \\
			\$ java -ea demo \\
			Exception in thread "main" java.lang.\textbf{AssertionError}: Age cannot be negative 
		}
		
	\end{itemize}
	
\end{flushleft}
